\section{متطلبات الموارد لنظام الاستشعار عن بعد وإدارة المزارع}

\subsection{متطلبات الأجهزة}

\subsubsection{معدات الاستشعار عن بعد}
\begin{itemize}
    \item \textbf{أسطول الطائرات بدون طيار:}
    \begin{itemize}
        \item طائرتان بدون طيار ثابتة الجناح بوقت طيران 120 دقيقة وقدرة تغطية 120 هكتار
        \item 3 طائرات بدون طيار متعددة الدوارات مع أنظمة حمولة قابلة للتبديل
        \item 8 أنظمة كاميرا متخصصة (RGB، متعددة الأطياف، حرارية، وفائقة الطيفية)
        \item 12 مجموعة بطاريات و4 محطات شحن سريع
        \item محطتا تحكم أرضية محمولة مع أجهزة لوحية مقواة
    \end{itemize}
    
    \item \textbf{شبكة أجهزة الاستشعار الأرضية:}
    \begin{itemize}
        \item 150 جهاز استشعار لرطوبة التربة ودرجة الحرارة مع مسابر متعددة الأعماق
        \item 25 محطة طقس مع مراقبة كاملة للمعايير الجوية
        \item 50 مقياس تدفق للري وأجهزة استشعار الضغط
        \item 30 جهاز استشعار متخصص للمحاصيل (تدفق النسغ، درجة حرارة الأوراق، إلخ)
        \item 10 محطات مراقبة جودة المياه لأنظمة الري
    \end{itemize}
    
    \item \textbf{البنية التحتية للاتصالات:}
    \begin{itemize}
        \item 15 بوابة LoRaWAN بنطاق 5-10 كم
        \item 20 عقدة شبكة Wi-Fi متداخلة للمناطق ذات النطاق الترددي العالي
        \item محطتا إنترنت عبر الأقمار الصناعية للاتصال الاحتياطي
        \item 10 عقد حوسبة طرفية لمعالجة البيانات المحلية
        \item 25 نظام طاقة شمسية لتركيبات أجهزة الاستشعار البعيدة
    \end{itemize}
\end{itemize}

\subsection{متطلبات البرمجيات والبيانات}

\subsubsection{أنظمة البرمجيات الأساسية}
\begin{itemize}
    \item \textbf{منصة إدارة البيانات:}
    \begin{itemize}
        \item بنية بحيرة بيانات سحابية قابلة للتوسع
        \item قاعدة بيانات السلاسل الزمنية لتخزين بيانات أجهزة الاستشعار
        \item قاعدة بيانات مكانية لنظم المعلومات الجغرافية وبيانات الاستشعار عن بعد
        \item برمجيات خط أنابيب ETL لتكامل البيانات
        \item نظام إدارة جودة البيانات والتحقق منها
    \end{itemize}
    
    \item \textbf{برمجيات التحليلات:}
    \begin{itemize}
        \item بيئة تطوير التعلم الآلي والذكاء الاصطناعي
        \item برمجيات معالجة صور الاستشعار عن بعد
        \item برمجيات التحليل المكاني ونظم المعلومات الجغرافية
        \item أدوات التحليل الإحصائي والنمذجة
        \item منصة إنشاء التصور ولوحات المعلومات
    \end{itemize}
    
    \item \textbf{برمجيات التطبيقات:}
    \begin{itemize}
        \item نظام معلومات إدارة المزارع (FMIS)
        \item منصة تطوير تطبيقات الجوال
        \item إطار عمل نظام دعم القرار
        \item منصة إدارة وتكامل واجهات برمجة التطبيقات (API)
        \item نظام الإخطار والتنبيه
    \end{itemize}
\end{itemize}

\subsubsection{متطلبات البيانات}
\begin{itemize}
    \item \textbf{مصادر البيانات الخارجية:}
    \begin{itemize}
        \item اشتراكات صور الأقمار الصناعية (Sentinel-2، Landsat، مزودون تجاريون)
        \item تغذيات بيانات توقعات الطقس الإقليمية
        \item البيانات المناخية والزراعية التاريخية للمنطقة
        \item خرائط مسح التربة والجيولوجيا
        \item تغذيات بيانات أسعار السوق والسلع
    \end{itemize}
    
    \item \textbf{التخزين والمعالجة:}
    \begin{itemize}
        \item تخزين البيانات الأولي: 50 تيرابايت
        \item نمو البيانات السنوي: 25-30 تيرابايت
        \item موارد الحوسبة عالية الأداء لمعالجة الصور
        \item أنظمة النسخ الاحتياطي والتعافي من الكوارث
        \item نظام أرشفة البيانات وإدارة دورة الحياة
    \end{itemize}
\end{itemize}

\subsection{الموارد البشرية}

\subsubsection{الفريق الأساسي (15 موظف بدوام كامل)}
\begin{itemize}
    \item مدير وحدة واحد مع خبرة في تكنولوجيا الزراعة
    \item 3 علماء بيانات متخصصين في التحليلات الزراعية
    \item متخصصان زراعيان مع خبرة في الزراعة الدقيقة
    \item 3 مهندسي برمجيات لتطوير المنصة وصيانتها
    \item 4 فنيين ميدانيين لنشر الأجهزة وصيانتها
    \item 2 مديري نجاح العملاء لتقديم الخدمات الخارجية
\end{itemize}

\subsubsection{الفريق الموسع (8 موظفين بدوام كامل)}
\begin{itemize}
    \item 2 مشغلي طائرات بدون طيار معتمدين للعمليات الزراعية
    \item 2 متخصصي نظم معلومات جغرافية مع خبرة في رسم الخرائط الزراعية
    \item 2 مهندسي إنترنت الأشياء لإدارة شبكة أجهزة الاستشعار
    \item مصمم واجهة مستخدم/تجربة مستخدم واحد لتطوير واجهة المستخدم
    \item متخصص تدريب واحد لبرامج بناء القدرات
\end{itemize}

\subsubsection{وظائف الدعم (موارد مشتركة)}
\begin{itemize}
    \item دعم إداري (1 مخصص، والباقي مشترك)
    \item إدارة مالية (مورد مشترك)
    \item شؤون قانونية وامتثال (مورد مشترك)
    \item تسويق واتصالات (مورد مشترك)
\end{itemize}

\subsection{المرافق والبنية التحتية}

\subsubsection{المرافق المادية}
\begin{itemize}
    \item \textbf{مركز العمليات الرئيسي (300 م²):}
    \begin{itemize}
        \item مركز بيانات وغرفة خوادم (50 م²)
        \item غرفة تحكم مع شاشات مراقبة (60 م²)
        \item مكاتب الموظفين ومحطات العمل (120 م²)
        \item غرف اجتماعات وتدريب (50 م²)
        \item منطقة تخزين وصيانة المعدات (20 م²)
    \end{itemize}
    
    \item \textbf{المرافق الميدانية:}
    \begin{itemize}
        \item منشأة عمليات وصيانة الطائرات بدون طيار (80 م²)
        \item منطقة معايرة أجهزة الاستشعار محمية من الطقس (40 م²)
        \item تخزين المعدات الميدانية ومحطات الشحن (60 م²)
        \item قطع عرض واختبار (1 هكتار)
    \end{itemize}
\end{itemize}

\subsubsection{المرافق والخدمات}
\begin{itemize}
    \item إمداد طاقة موثوق مع نظام UPS واحتياطي مولد
    \item اتصال إنترنت عالي السرعة (1 جيجابت في الثانية كحد أدنى)
    \item تحكم في المناخ لمناطق المعدات الحساسة
    \item أنظمة الأمن المادي والسيبراني
    \item إمداد مياه لتنظيف المعدات وصيانتها
\end{itemize}

\subsection{الموارد المالية}

\subsubsection{متطلبات الاستثمار الرأسمالي}
\begin{itemize}
    \item معدات الاستشعار عن بعد: 4.2 مليون جنيه مصري
    \item شبكة أجهزة الاستشعار الأرضية: 2.8 مليون جنيه مصري
    \item البنية التحتية للاتصالات: 1.5 مليون جنيه مصري
    \item أجهزة الحوسبة ومركز البيانات: 3.2 مليون جنيه مصري
    \item تراخيص البرمجيات والتطوير: 2.5 مليون جنيه مصري
    \item المرافق والبنية التحتية: 3.8 مليون جنيه مصري
    \item اكتساب البيانات الأولية: 1.0 مليون جنيه مصري
    \item إجمالي الاستثمار الرأسمالي: 19.0 مليون جنيه مصري
\end{itemize}

\subsubsection{الميزانية التشغيلية السنوية}
\begin{itemize}
    \item رواتب الموظفين والمزايا: 4.8 مليون جنيه مصري
    \item صيانة المعدات واستبدالها: 1.2 مليون جنيه مصري
    \item اشتراكات البرمجيات والتحديثات: 0.8 مليون جنيه مصري
    \item اشتراكات البيانات والخدمات: 0.6 مليون جنيه مصري
    \item المرافق والاتصالات: 0.4 مليون جنيه مصري
    \item التدريب وبناء القدرات: 0.3 مليون جنيه مصري
    \item البحث والتطوير: 0.9 مليون جنيه مصري
    \item إجمالي الميزانية التشغيلية السنوية: 9.0 مليون جنيه مصري
\end{itemize}

\subsection{الجدول الزمني للتنفيذ}

\subsubsection{جدول اكتساب الموارد}
\begin{itemize}
    \item \textbf{المرحلة 1 (الأشهر 1-6):}
    \begin{itemize}
        \item توظيف الفريق الأساسي وتأهيله
        \item شراء الأجهزة الأولية (25\% من الإجمالي)
        \item اكتساب منصة البرمجيات الأساسية
        \item إنشاء مركز العمليات الرئيسي
    \end{itemize}
    
    \item \textbf{المرحلة 2 (الأشهر 7-12):}
    \begin{itemize}
        \item توظيف الفريق الموسع
        \item نشر شبكة أجهزة الاستشعار (تغطية 50\%)
        \item اكتساب أسطول الطائرات بدون طيار واختباره
        \item تطوير منصة تكامل البيانات
    \end{itemize}
    
    \item \textbf{المرحلة 3 (الأشهر 13-24):}
    \begin{itemize}
        \item نشر الأجهزة بالكامل
        \item تطوير قدرات التحليلات المتقدمة
        \item إكمال تطبيقات الجوال وواجهات المستخدم
        \item إعداد عرض الخدمات الخارجية
    \end{itemize}
\end{itemize}

\subsection{استراتيجيات تحسين الموارد}

\begin{itemize}
    \item اكتساب المعدات على مراحل بما يتماشى مع الجدول الزمني للتنفيذ
    \item البنية التحتية المشتركة مع الوحدات الأخرى حيثما أمكن
    \item موارد قائمة على السحابة لاحتياجات الحوسبة القابلة للتوسع
    \item استخدام البرمجيات مفتوحة المصدر حيثما كان ذلك مناسبًا
    \item تدريب وتطوير المواهب المحلية
    \item برامج الصيانة الوقائية لإطالة عمر المعدات
    \item تركيبات موفرة للطاقة وتعمل بالطاقة الشمسية في المواقع البعيدة
    \item شراكات استراتيجية للقدرات المتخصصة
\end{itemize} 