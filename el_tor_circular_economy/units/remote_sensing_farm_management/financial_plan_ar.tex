\section{الخطة المالية}

\subsection{متطلبات الاستثمار}

\subsubsection{النفقات الرأسمالية الأولية (CAPEX)}
\begin{table}[H]
\centering
\begin{tabular}{lrr}
\toprule
\textbf{الفئة} & \textbf{المبلغ (جنيه مصري)} & \textbf{ملاحظات} \\
\midrule
معدات الاستشعار عن بعد & 1,250,000 & طائرات بدون طيار، كاميرات، أجهزة استشعار \\
شبكة أجهزة استشعار إنترنت الأشياء & 850,000 & أجهزة استشعار ميدانية، محطات طقس \\
البنية التحتية للحوسبة & 650,000 & خوادم، أجهزة طرفية، شبكات \\
البرمجيات والتراخيص & 450,000 & نظم المعلومات الجغرافية، التحليلات، إدارة البيانات \\
المركبات والمعدات الميدانية & 350,000 & مركبات ميدانية، أدوات صيانة \\
تجهيز المكاتب & 150,000 & محطات عمل، أثاث، معدات \\
\midrule
\textbf{إجمالي النفقات الرأسمالية} & \textbf{3,700,000} & \\
\bottomrule
\end{tabular}
\caption{تفصيل النفقات الرأسمالية الأولية}
\end{table}

\subsubsection{خطة الاستثمار المرحلية}
\begin{table}[H]
\centering
\begin{tabular}{lrrr}
\toprule
\textbf{مرحلة الاستثمار} & \textbf{الإطار الزمني} & \textbf{المبلغ (جنيه مصري)} & \textbf{مجالات التركيز} \\
\midrule
المرحلة 1 (الأولية) & السنة 1 & 2,200,000 & البنية التحتية الأساسية، القدرات الأساسية \\
المرحلة 2 (التوسع) & السنة 2 & 1,500,000 & توسيع الشبكة، التحليلات المتقدمة \\
المرحلة 3 (التحسين) & السنة 3 & 800,000 & تحسين النظام، التكامل \\
\midrule
\textbf{إجمالي الاستثمار} & 3 سنوات & \textbf{4,500,000} & \\
\bottomrule
\end{tabular}
\caption{خطة الاستثمار المرحلية}
\end{table}

\subsection{النفقات التشغيلية (OPEX)}

\subsubsection{تكاليف التشغيل السنوية}
\begin{table}[H]
\centering
\begin{tabular}{lrr}
\toprule
\textbf{الفئة} & \textbf{التكلفة السنوية (جنيه مصري)} & \textbf{ملاحظات} \\
\midrule
الموظفون & 1,450,000 & الرواتب، المزايا، التدريب \\
خدمات البيانات & 350,000 & صور الأقمار الصناعية، خدمات سحابية \\
صيانة المعدات & 280,000 & إصلاحات، معايرة، استبدالات \\
اشتراكات البرمجيات & 220,000 & تراخيص سنوية، تحديثات \\
المرافق & 120,000 & الكهرباء، الإنترنت، الاتصالات \\
المواد الاستهلاكية & 80,000 & بطاريات، قطع غيار، مستلزمات \\
\midrule
\textbf{إجمالي النفقات التشغيلية السنوية} & \textbf{2,500,000} & \\
\bottomrule
\end{tabular}
\caption{تفصيل النفقات التشغيلية السنوية}
\end{table}

\subsubsection{توقعات النفقات التشغيلية لخمس سنوات}
\begin{table}[H]
\centering
\begin{tabular}{lrrrrr}
\toprule
\textbf{الفئة} & \textbf{السنة 1} & \textbf{السنة 2} & \textbf{السنة 3} & \textbf{السنة 4} & \textbf{السنة 5} \\
\midrule
الموظفون & 1,450,000 & 1,595,000 & 1,754,500 & 1,929,950 & 2,122,945 \\
خدمات البيانات & 350,000 & 367,500 & 385,875 & 405,169 & 425,427 \\
صيانة المعدات & 280,000 & 308,000 & 338,800 & 372,680 & 409,948 \\
اشتراكات البرمجيات & 220,000 & 231,000 & 242,550 & 254,678 & 267,411 \\
المرافق & 120,000 & 126,000 & 132,300 & 138,915 & 145,861 \\
المواد الاستهلاكية & 80,000 & 84,000 & 88,200 & 92,610 & 97,241 \\
\midrule
\textbf{إجمالي النفقات التشغيلية} & \textbf{2,500,000} & \textbf{2,711,500} & \textbf{2,942,225} & \textbf{3,194,002} & \textbf{3,468,833} \\
\bottomrule
\end{tabular}
\caption{توقعات النفقات التشغيلية لخمس سنوات}
\end{table}

\subsection{مصادر الإيرادات}

\subsubsection{خلق القيمة الداخلية}
\begin{table}[H]
\centering
\begin{tabular}{lrr}
\toprule
\textbf{فئة القيمة} & \textbf{القيمة السنوية (جنيه مصري)} & \textbf{طريقة القياس} \\
\midrule
تحسينات كفاءة المياه & 850,000 & تقليل الاستهلاك × تكلفة المياه \\
تحسين استخدام الأسمدة & 650,000 & تقليل المدخلات × تكلفة الأسمدة \\
تحسينات المحصول & 1,200,000 & زيادة المحصول × قيمة المحصول \\
كفاءة العمالة & 450,000 & تقليل ساعات العمل × تكلفة العمالة \\
منع الخسائر & 750,000 & الخسائر التي تم منعها × قيمة المحصول \\
\midrule
\textbf{إجمالي القيمة الداخلية} & \textbf{3,900,000} & \\
\bottomrule
\end{tabular}
\caption{خلق القيمة الداخلية السنوية}
\end{table}

\subsubsection{إمكانات الإيرادات الخارجية (المستقبلية)}
\begin{table}[H]
\centering
\begin{tabular}{lrrr}
\toprule
\textbf{فئة الخدمة} & \textbf{السنة 3} & \textbf{السنة 4} & \textbf{السنة 5} \\
\midrule
خدمات المراقبة الأساسية & 250,000 & 500,000 & 750,000 \\
التحليلات المتقدمة & 150,000 & 350,000 & 600,000 \\
الخدمات الاستشارية & 100,000 & 200,000 & 350,000 \\
\midrule
\textbf{إجمالي الإيرادات الخارجية} & \textbf{500,000} & \textbf{1,050,000} & \textbf{1,700,000} \\
\bottomrule
\end{tabular}
\caption{توقعات الإيرادات الخارجية (السنوات 3-5)}
\end{table}

\subsection{التحليل المالي}

\subsubsection{العائد على الاستثمار (ROI)}
\begin{table}[H]
\centering
\begin{tabular}{lrrrrr}
\toprule
\textbf{الفئة} & \textbf{السنة 1} & \textbf{السنة 2} & \textbf{السنة 3} & \textbf{السنة 4} & \textbf{السنة 5} \\
\midrule
إجمالي الاستثمار (التراكمي) & 2,200,000 & 3,700,000 & 4,500,000 & 4,500,000 & 4,500,000 \\
النفقات التشغيلية السنوية & 2,500,000 & 2,711,500 & 2,942,225 & 3,194,002 & 3,468,833 \\
القيمة الداخلية المنشأة & 2,500,000 & 3,250,000 & 3,900,000 & 4,290,000 & 4,719,000 \\
الإيرادات الخارجية & 0 & 0 & 500,000 & 1,050,000 & 1,700,000 \\
صافي القيمة السنوية & 0 & 538,500 & 1,457,775 & 2,145,998 & 2,950,167 \\
صافي القيمة التراكمية & 0 & 538,500 & 1,996,275 & 4,142,273 & 7,092,440 \\
\midrule
\textbf{العائد على الاستثمار (التراكمي)} & \textbf{-100\%} & \textbf{-85\%} & \textbf{-56\%} & \textbf{-8\%} & \textbf{58\%} \\
\bottomrule
\end{tabular}
\caption{تحليل العائد على الاستثمار لخمس سنوات}
\end{table}

\subsubsection{فترة الاسترداد}
بناءً على توقعات صافي القيمة التراكمية، من المتوقع استرداد الاستثمار الأولي بالكامل خلال السنة الخامسة، مع فترة استرداد تبلغ حوالي 4.5 سنوات.

\subsection{استراتيجية التمويل}

\subsubsection{مصادر التمويل}
\begin{itemize}
    \item \textbf{التخصيص الداخلي للمشروع:} 60\% من النفقات الرأسمالية الأولية من ميزانية مشروع الاقتصاد الدائري في الطور الإجمالية
    
    \item \textbf{منح التكنولوجيا:} 25\% من منح ابتكار التكنولوجيا الزراعية ومبادرات الزراعة المستدامة
    
    \item \textbf{الشراكات الاستراتيجية:} 15\% من خلال المساهمات العينية والتطوير المشترك مع مزودي التكنولوجيا
\end{itemize}

\subsubsection{إدارة المخاطر المالية}
\begin{itemize}
    \item \textbf{التنفيذ المرحلي:} نهج استثماري مرحلي للتحقق من خلق القيمة قبل النشر الكامل
    
    \item \textbf{استئجار التكنولوجيا:} النظر في خيارات الاستئجار للمعدات عالية القيمة لتقليل متطلبات رأس المال الأولية
    
    \item \textbf{البنية التحتية المشتركة:} الاستفادة من البنية التحتية الحالية للحوسبة والشبكات حيثما أمكن
    
    \item \textbf{مقاييس قائمة على القيمة:} تتبع واضح لخلق القيمة لتبرير الاستثمار المستمر
    
    \item \textbf{احتياطي الطوارئ:} تخصيص 10\% من إجمالي الميزانية كاحتياطي للتكاليف غير المتوقعة
\end{itemize}

\subsection{خطة الاستدامة المالية}
\begin{itemize}
    \item \textbf{المدى القصير (السنوات 1-2):} التركيز على خلق القيمة الداخلية من خلال تحسين الموارد وتحسينات المحصول
    
    \item \textbf{المدى المتوسط (السنوات 3-4):} بدء تقديم الخدمات الخارجية للمزارع المجاورة والمشاريع الزراعية
    
    \item \textbf{المدى الطويل (السنة 5+):} تطوير عرض شامل للزراعة كخدمة (FaaS) مع نماذج اشتراك متدرجة
    
    \item \textbf{استراتيجية إعادة الاستثمار:} تخصيص 15\% من خلق القيمة السنوية لترقيات التكنولوجيا وتوسيع القدرات
    
    \item \textbf{تحسين التكلفة:} التحسين المستمر للكفاءة التشغيلية لتقليل النفقات التشغيلية كنسبة مئوية من القيمة المنشأة
\end{itemize} 