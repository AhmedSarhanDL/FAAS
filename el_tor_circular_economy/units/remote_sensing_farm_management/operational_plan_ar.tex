\section{الخطة التشغيلية}

\subsection{الهيكل التشغيلي}
ستعمل وحدة الاستشعار عن بعد وإدارة المزرعة بفريق متخصص منظم في المجالات الوظيفية التالية:

\begin{itemize}
    \item \textbf{عمليات الاستشعار عن بعد:} مسؤولة عن الحصول على بيانات الأقمار الصناعية، وعمليات تحليق الطائرات بدون طيار، ومعالجة الصور
    
    \item \textbf{البنية التحتية لإنترنت الأشياء:} تدير نشر وصيانة وجمع البيانات من شبكات أجهزة الاستشعار
    
    \item \textbf{إدارة البيانات:} تتعامل مع تخزين البيانات ومعالجتها وتكاملها ومراقبة جودتها
    
    \item \textbf{التحليلات ودعم القرار:} تطور وتحافظ على النماذج التحليلية وأدوات دعم القرار
    
    \item \textbf{العمليات الميدانية:} تجري التحقق الميداني وصيانة المعدات والتنفيذ على مستوى الحقل
    
    \item \textbf{التدريب ونقل المعرفة:} توفر بناء القدرات والدعم الفني للوحدات الأخرى
\end{itemize}

\subsection{بنية النظام}

\subsubsection{طبقة اكتساب البيانات}
\begin{itemize}
    \item \textbf{صور الأقمار الصناعية:} اشتراك في Sentinel-2 (دقة 10 متر، إعادة زيارة كل 5 أيام) ومزودين تجاريين (دقة دون المتر) للمراقبة المنتظمة للغطاء الأرضي
    
    \item \textbf{أسطول الطائرات بدون طيار:} 
    \begin{itemize}
        \item طائرتان بدون طيار ثابتة الجناح لرسم خرائط المناطق الكبيرة (120 هكتار/رحلة)
        \item 3 طائرات بدون طيار متعددة الدوارات مع مستشعرات قابلة للتبديل (RGB، متعددة الأطياف، حرارية)
        \item برمجيات تخطيط الرحلات الآلية وتنفيذ المهام
    \end{itemize}
    
    \item \textbf{شبكة أجهزة الاستشعار الأرضية:}
    \begin{itemize}
        \item 150 جهاز استشعار لرطوبة التربة ودرجة الحرارة على أعماق متعددة
        \item 25 محطة طقس تقيس درجة الحرارة، والرطوبة، والرياح، والإشعاع الشمسي، وهطول الأمطار
        \item 50 مقياس تدفق للري وأجهزة استشعار الضغط
        \item 30 جهاز استشعار متخصص للمحاصيل (تدفق النسغ، درجة حرارة الأوراق، إلخ)
    \end{itemize}
    
    \item \textbf{جمع البيانات المتنقل:}
    \begin{itemize}
        \item تزويد فريق العمل الميداني بتطبيقات جوال للملاحظات المرجعية جغرافيًا
        \item نظام QR/RFID لتحديد الأصول والقطع
        \item قدرات تحويل الصوت إلى نص لإدخال البيانات بدون استخدام اليدين
    \end{itemize}
\end{itemize}

\subsubsection{طبقة نقل البيانات}
\begin{itemize}
    \item \textbf{الشبكة الأساسية:} شبكة LoRaWAN على مستوى الموقع للاتصال منخفض الطاقة وطويل المدى لأجهزة الاستشعار
    
    \item \textbf{الشبكات الثانوية:}
    \begin{itemize}
        \item شبكة Wi-Fi متداخلة تغطي المرافق المركزية ومناطق النشاط العالي
        \item اتصال 4G/LTE للتطبيقات ذات النطاق الترددي العالي والوصول عن بعد
        \item نسخة احتياطية من الإنترنت عبر الأقمار الصناعية للأنظمة الحرجة
    \end{itemize}
    
    \item \textbf{الحوسبة الطرفية:}
    \begin{itemize}
        \item عقد معالجة موزعة للتحليلات الحساسة للوقت
        \item تخزين البيانات محليًا للتعامل مع انقطاعات الاتصال
        \item معالجة أولية على مستوى أجهزة الاستشعار لتقليل متطلبات النطاق الترددي
    \end{itemize}
\end{itemize}

\subsubsection{طبقة معالجة البيانات}
\begin{itemize}
    \item \textbf{منصة البيانات المركزية:}
    \begin{itemize}
        \item بنية سحابية قابلة للتوسع مع نسخة احتياطية محلية
        \item بحيرة بيانات لتخزين البيانات الخام وإدارتها
        \item خطوط أنابيب ETL لتنظيف البيانات والتحقق منها ودمجها
    \end{itemize}
    
    \item \textbf{محرك التحليلات:}
    \begin{itemize}
        \item خط أنابيب التعلم الآلي لتدريب النماذج ونشرها
        \item تحليل السلاسل الزمنية لاكتشاف الأنماط الزمنية
        \item تحليلات مكانية للتعرف على الأنماط الجغرافية المكانية
        \item نمذجة تنبؤية للتوقع وتحليل السيناريوهات
    \end{itemize}
    
    \item \textbf{نظام دعم القرار:}
    \begin{itemize}
        \item محرك توصيات قائم على القواعد
        \item خوارزميات التحسين لتخصيص الموارد
        \item قدرات المحاكاة لاختبار السيناريوهات
        \item نظام التنبيه والإخطار للأحداث الحرجة
    \end{itemize}
\end{itemize}

\subsubsection{طبقة واجهة المستخدم}
\begin{itemize}
    \item \textbf{لوحة الإدارة:}
    \begin{itemize}
        \item واجهة مستندة إلى الويب مع التحكم في الوصول المستند إلى الأدوار
        \item عروض قابلة للتخصيص لمختلف أصحاب المصلحة
        \item خرائط وتصورات تفاعلية
        \item مقاييس الأداء وتتبع مؤشرات الأداء الرئيسية
    \end{itemize}
    
    \item \textbf{تطبيقات الجوال:}
    \begin{itemize}
        \item تطبيق العمليات الميدانية لجمع البيانات وإدارة المهام
        \item تطبيق المدير لدعم القرار والموافقات
        \item تطبيق الفني لصيانة المعدات واستكشاف الأخطاء وإصلاحها
        \item تطبيق العميل لمستخدمي الخدمات الخارجية
    \end{itemize}
    
    \item \textbf{نظام الإخطارات:}
    \begin{itemize}
        \item تنبيهات متعددة القنوات (تطبيق، رسائل نصية قصيرة، بريد إلكتروني)
        \item رسائل ذات أولوية بناءً على الإلحاح
        \item بروتوكولات الإقرار والتصعيد
    \end{itemize}
\end{itemize}

\subsection{العمليات الأساسية}

\subsubsection{جمع البيانات وإدارتها}
\begin{itemize}
    \item \textbf{المراقبة المجدولة:}
    \begin{itemize}
        \item جمع يومي لبيانات أجهزة الاستشعار الأرضية
        \item مسوحات أسبوعية بالطائرات بدون طيار للمناطق ذات الأولوية
        \item تغطية كاملة نصف شهرية بالطائرات بدون طيار لجميع المناطق الزراعية
        \item دمج مستمر لصور الأقمار الصناعية المتاحة
    \end{itemize}
    
    \item \textbf{إدارة جودة البيانات:}
    \begin{itemize}
        \item التحقق الآلي مقابل النطاقات المقبولة والأنماط التاريخية
        \item معايرة منتظمة لأجهزة الاستشعار والتحقق المتبادل
        \item مراقبة اكتمال البيانات وإجراءات ملء الفجوات
        \item التحكم في الإصدار ومسار التدقيق لجميع مجموعات البيانات
    \end{itemize}
    
    \item \textbf{حوكمة البيانات:}
    \begin{itemize}
        \item سياسات واضحة للملكية والوصول
        \item بروتوكولات خصوصية وأمن البيانات
        \item إجراءات الاحتفاظ والأرشفة
        \item البيانات الوصفية الموحدة والتوثيق
    \end{itemize}
\end{itemize}

\subsubsection{التحليلات وتوليد الرؤى}
\begin{itemize}
    \item \textbf{منتجات التحليلات المنتظمة:}
    \begin{itemize}
        \item توصيات الري اليومية حسب المنطقة
        \item تقييم أسبوعي لصحة المحاصيل وتوصيات التدخل
        \item تقارير شهرية عن الإنتاجية وكفاءة الموارد
        \item توقعات موسمية للمحصول وسيناريوهات التخطيط
    \end{itemize}
    
    \item \textbf{تحليلات عند الطلب:}
    \begin{itemize}
        \item نمذجة مخاطر الآفات والأمراض
        \item تخطيط التدخل المخصص
        \item تحسين تخصيص الموارد
        \item تقييم تأثير التغييرات الإدارية
    \end{itemize}
    
    \item \textbf{التحسين المستمر:}
    \begin{itemize}
        \item مراقبة أداء النموذج وإعادة التدريب
        \item جمع التغذية الراجعة من المستخدمين حول جودة التوصيات
        \item قياس منتظم مقابل النتائج الفعلية
        \item دمج طرق تحليلية جديدة ومصادر بيانات
    \end{itemize}
\end{itemize}

\subsubsection{دعم القرار والتنفيذ}
\begin{itemize}
    \item \textbf{تقديم التوصيات:}
    \begin{itemize}
        \item التوزيع الآلي لأصحاب المصلحة المعنيين
        \item عرض غني بالسياق مع أدلة داعمة
        \item قابلية واضحة للتنفيذ مع توجيهات التنفيذ
        \item آلية تغذية راجعة لجودة التوصية
    \end{itemize}
    
    \item \textbf{تتبع التنفيذ:}
    \begin{itemize}
        \item إدارة المهام الرقمية للعمليات الميدانية
        \item مراقبة التقدم في الوقت الفعلي
        \item التحقق من جودة التنفيذ
        \item قياس النتائج وتوثيقها
    \end{itemize}
    
    \item \textbf{تنسيق التدخل:}
    \begin{itemize}
        \item جدولة الموارد عبر الوحدات
        \item إطار تحديد الأولويات للاحتياجات المتنافسة
        \item بروتوكولات الاستجابة للطوارئ
        \item إجراءات إدارة التغيير
    \end{itemize}
\end{itemize}

\subsection{الخدمات الزراعية المتخصصة}

\subsubsection{إدارة الري الدقيق}
\begin{itemize}
    \item \textbf{مدخلات البيانات:}
    \begin{itemize}
        \item رطوبة التربة على أعماق متعددة
        \item بيانات الطقس وتقديرات التبخر-النتح
        \item مرحلة نمو المحصول ومتطلبات المياه
        \item قدرات وقيود نظام الري
    \end{itemize}
    
    \item \textbf{العمليات التحليلية:}
    \begin{itemize}
        \item نمذجة توازن مياه التربة
        \item حساب متطلبات المياه الخاصة بالمحصول
        \item تقييم كفاءة الري
        \item تحسين ضغط وتدفق النظام
    \end{itemize}
    
    \item \textbf{المخرجات والإجراءات:}
    \begin{itemize}
        \item جداول ري خاصة بالمنطقة
        \item خرائط تطبيق معدل متغير
        \item توصيات صيانة النظام
        \item تقارير كفاءة استخدام المياه
    \end{itemize}
\end{itemize}

\subsubsection{مراقبة صحة المحاصيل}
\begin{itemize}
    \item \textbf{مدخلات البيانات:}
    \begin{itemize}
        \item مؤشرات الغطاء النباتي متعددة الأطياف (NDVI، EVI، إلخ)
        \item التصوير الحراري لاكتشاف الإجهاد
        \item الأنماط التاريخية ونماذج النمو
        \item الملاحظات الميدانية وبيانات التحقق الأرضي
    \end{itemize}
    
    \item \textbf{العمليات التحليلية:}
    \begin{itemize}
        \item خوارزميات اكتشاف الشذوذ
        \item التعرف على الأنماط لعوامل الإجهاد المحددة
        \item مراقبة مرحلة النمو والمقارنة
        \item التنبؤ بتأثير المحصول
    \end{itemize}
    
    \item \textbf{المخرجات والإجراءات:}
    \begin{itemize}
        \item تنبيهات الكشف المبكر عن الإجهاد
        \item خرائط التدخل المرجعية جغرافيًا
        \item توصيات العلاج
        \item مراقبة الفعالية
    \end{itemize}
\end{itemize}

\subsubsection{إدارة الآفات والأمراض}
\begin{itemize}
    \item \textbf{مدخلات البيانات:}
    \begin{itemize}
        \item ظروف الطقس والتوقعات
        \item قابلية المحصول للإصابة حسب مرحلة النمو
        \item أنماط التفشي التاريخية
        \item تقارير مراقبة المصائد والاستكشاف الميداني
    \end{itemize}
    
    \item \textbf{العمليات التحليلية:}
    \begin{itemize}
        \item نمذجة تطور الآفات
        \item رسم خرائط المخاطر بناءً على الظروف البيئية
        \item محاكاة الانتشار
        \item تحسين العلاج
    \end{itemize}
    
    \item \textbf{المخرجات والإجراءات:}
    \begin{itemize}
        \item توقعات المخاطر والإنذارات المبكرة
        \item توصيات الاستكشاف المستهدف
        \item خطط التطبيق الدقيق
        \item جدولة نشر المكافحة البيولوجية
    \end{itemize}
\end{itemize}

\subsubsection{إدارة المغذيات}
\begin{itemize}
    \item \textbf{مدخلات البيانات:}
    \begin{itemize}
        \item تحليل مغذيات التربة
        \item متطلبات المغذيات للمحصول حسب مرحلة النمو
        \item توفر المادة العضوية والتعديلات
        \item التصوير متعدد الأطياف لاكتشاف النقص
    \end{itemize}
    
    \item \textbf{العمليات التحليلية:}
    \begin{itemize}
        \item نمذجة توازن المغذيات
        \item تقييم مخاطر النقص
        \item تحسين تدفق الموارد الدائري
        \item تحسين توقيت التطبيق
    \end{itemize}
    
    \item \textbf{المخرجات والإجراءات:}
    \begin{itemize}
        \item خرائط تسميد بمعدل متغير
        \item خطط تخصيص التعديلات العضوية
        \item توصيات التطبيق الورقي
        \item تقارير كفاءة استخدام المغذيات
    \end{itemize}
\end{itemize}

\subsection{تكامل الاقتصاد الدائري}

\subsubsection{مراقبة تدفق الموارد}
\begin{itemize}
    \item \textbf{تتبع دورة المياه:}
    \begin{itemize}
        \item المراقبة في الوقت الفعلي لحركة المياه بين الوحدات
        \item تقييم الجودة عند نقاط النقل
        \item مقاييس الكفاءة وتحديد الفقد
        \item توصيات التحسين
    \end{itemize}
    
    \item \textbf{تتبع دورة المغذيات:}
    \begin{itemize}
        \item تحديد كمية تدفق الكتلة الحيوية والمادة العضوية
        \item تحليل محتوى المغذيات وميزانياتها
        \item مراقبة كفاءة التحويل
        \item اكتشاف التسرب والوقاية منه
    \end{itemize}
    
    \item \textbf{تتبع تدفق الطاقة:}
    \begin{itemize}
        \item مراقبة إنتاج واستهلاك الطاقة المتجددة
        \item تتبع استخدام الديزل الحيوي والغاز الحيوي
        \item مقاييس كفاءة الطاقة حسب العملية
        \item تحديد فرص التحسين
    \end{itemize}
\end{itemize}

\subsubsection{التحسين عبر الوحدات}
\begin{itemize}
    \item \textbf{التخطيط المتكامل:}
    \begin{itemize}
        \item جدولة الإنتاج المنسقة
        \item تحسين تخصيص الموارد
        \item تخطيط استخدام البنية التحتية المشتركة
        \item جدولة الصيانة المتزامنة
    \end{itemize}
    
    \item \textbf{آليات التغذية الراجعة:}
    \begin{itemize}
        \item مشاركة مقاييس الأداء في الوقت الفعلي
        \item تقييم التأثير عبر الوحدات
        \item منصة حل المشكلات التعاونية
        \item نظام اقتراح التحسين المستمر
    \end{itemize}
    
    \item \textbf{تحليلات على مستوى النظام:}
    \begin{itemize}
        \item مقاييس الكفاءة الشاملة
        \item تحديد الاختناقات
        \item نمذجة السيناريوهات للتغييرات في النظام
        \item تقييم المرونة
    \end{itemize}
\end{itemize}

\subsection{تقديم الخدمات الخارجية (الزراعة كخدمة)}

\subsubsection{محفظة الخدمات}
\begin{itemize}
    \item \textbf{الخدمات الأساسية:}
    \begin{itemize}
        \item مراقبة المحاصيل عن بعد وتقييم الصحة
        \item جدولة الري المستندة إلى الطقس
        \item تقدير المحصول والتنبؤ به
        \item الوصول الأساسي إلى برمجيات إدارة المزارع
    \end{itemize}
    
    \item \textbf{الخدمات المتقدمة:}
    \begin{itemize}
        \item إنشاء وصفات التطبيق الدقيق
        \item إنشاء توأم رقمي شامل للمزرعة
        \item تحليلات مخصصة ودعم القرار
        \item التكامل مع معدات وأنظمة المزرعة
    \end{itemize}
    
    \item \textbf{الخدمات المتخصصة:}
    \begin{itemize}
        \item قياس عزل الكربون والتصديق عليه
        \item تحليل البصمة المائية وتحسينها
        \item تقييم تأثير التنوع البيولوجي
        \item تخطيط المرونة المناخية
    \end{itemize}
\end{itemize}

\subsubsection{نموذج تقديم الخدمة}
\begin{itemize}
    \item \textbf{خيارات الاشتراك:}
    \begin{itemize}
        \item مستويات خدمة متدرجة مع عروض قيمة واضحة
        \item حزم موسمية وسنوية ومتعددة السنوات
        \item خيارات الدفع حسب الاستخدام للخدمات المتخصصة
        \item نماذج تسعير قائمة على النتائج للخدمات المتميزة
    \end{itemize}
    
    \item \textbf{تأهيل العملاء:}
    \begin{itemize}
        \item تقييم أولي للمزرعة وإنشاء خط الأساس
        \item إعداد تكامل النظام واتصال البيانات
        \item تدريب المستخدم وبناء القدرات
        \item التكوين والمعايرة المخصصة
    \end{itemize}
    
    \item \textbf{الدعم المستمر:}
    \begin{itemize}
        \item إدارة حسابات مخصصة
        \item مراجعات منتظمة للخدمة وتحسينها
        \item الدعم الفني واستكشاف الأخطاء وإصلاحها
        \item مشاركة المعرفة وأفضل الممارسات
    \end{itemize}
\end{itemize}

\subsection{إدارة البنية التحتية}

\subsubsection{صيانة الأجهزة}
\begin{itemize}
    \item \textbf{الصيانة الوقائية:}
    \begin{itemize}
        \item تنظيف ومعايرة أجهزة الاستشعار المجدولة
        \item صيانة أسطول الطائرات بدون طيار واستبدال المكونات
        \item فحص واختبار البنية التحتية للشبكة
        \item إدارة البطاريات ونظام الطاقة
    \end{itemize}
    
    \item \textbf{إجراءات الإصلاح:}
    \begin{itemize}
        \item بروتوكولات استجابة متدرجة بناءً على الأهمية
        \item إدارة مخزون قطع الغيار
        \item توثيق الإصلاح ومراقبة الجودة
        \item تحليل الأعطال والوقاية منها
    \end{itemize}
    
    \item \textbf{إدارة دورة الحياة:}
    \begin{itemize}
        \item تخطيط تجديد التكنولوجيا
        \item جدولة الاستبدال في نهاية العمر
        \item التخلص المستدام وإعادة التدوير
        \item إدارة مسار الترقية
    \end{itemize}
\end{itemize}

\subsubsection{إدارة البرمجيات والبيانات}
\begin{itemize}
    \item \textbf{إدارة النظام:}
    \begin{itemize}
        \item التحديثات المنتظمة وإدارة التصحيحات
        \item مراقبة الأداء وتحسينه
        \item إدارة وصول المستخدم والأمان
        \item النسخ الاحتياطي والتعافي من الكوارث
    \end{itemize}
    
    \item \textbf{عمليات التطوير:}
    \begin{itemize}
        \item خط أنابيب التكامل والنشر المستمر
        \item التحكم في الإصدار وإدارة التغيير
        \item الاختبار وضمان الجودة
        \item التوثيق وإدارة المعرفة
    \end{itemize}
    
    \item \textbf{إدارة البيانات:}
    \begin{itemize}
        \item تحسين التخزين والأرشفة
        \item عمليات سلامة البيانات والتحقق منها
        \item الامتثال للخصوصية والأمان
        \item بروتوكولات مشاركة البيانات وتصديرها
    \end{itemize}
\end{itemize}

\subsection{الموارد البشرية وبناء القدرات}

\subsubsection{هيكل الفريق}
\begin{itemize}
    \item \textbf{الفريق الأساسي:}
    \begin{itemize}
        \item مدير الوحدة (1): القيادة الاستراتيجية الشاملة
        \item علماء البيانات (3): التحليلات المتقدمة وتطوير النماذج
        \item المتخصصون الزراعيون (2): الخبرة المجالية والتحقق من التوصيات
        \item مهندسو البرمجيات (3): تطوير المنصة وصيانتها
        \item الفنيون الميدانيون (4): نشر الأجهزة وصيانتها
        \item مديرو نجاح العملاء (2): تقديم الخدمات الخارجية
    \end{itemize}
    
    \item \textbf{الفريق الموسع:}
    \begin{itemize}
        \item مشغلو الطائرات بدون طيار (2): جمع البيانات الجوية
        \item متخصصو نظم المعلومات الجغرافية (2): معالجة وتحليل البيانات المكانية
        \item مهندسو إنترنت الأشياء (2): إدارة شبكة أجهزة الاستشعار
        \item مصممو واجهة المستخدم/تجربة المستخدم (1): تطوير واجهة المستخدم
        \item متخصصو التدريب (1): بناء القدرات ونقل المعرفة
    \end{itemize}
    
    \item \textbf{وظائف الدعم:}
    \begin{itemize}
        \item الدعم الإداري (1)
        \item الإدارة المالية (مورد مشترك)
        \item الشؤون القانونية والامتثال (مورد مشترك)
        \item التسويق والاتصالات (مورد مشترك)
    \end{itemize}
\end{itemize}

\subsubsection{التدريب والتطوير}
\begin{itemize}
    \item \textbf{بناء القدرات الداخلية:}
    \begin{itemize}
        \item برنامج تطوير المهارات التقنية
        \item التدريب المتبادل لمرونة العمليات
        \item تطوير القيادة والإدارة
        \item ثقافة الابتكار والتحسين المستمر
    \end{itemize}
    
    \item \textbf{نقل المعرفة الخارجية:}
    \begin{itemize}
        \item برامج تدريب المستخدمين لجميع الوحدات الزراعية
        \item تعليم العملاء لمستخدمي الخدمات الخارجية
        \item أنشطة التوعية المجتمعية والعروض التوضيحية
        \item التعاون الأكاديمي والبحثي
    \end{itemize}
    
    \item \textbf{إدارة المعرفة:}
    \begin{itemize}
        \item توثيق العمليات وأفضل الممارسات
        \item التقاط الدروس المستفادة ونشرها
        \item منصة مشاركة المعرفة الداخلية
        \item برنامج النشر والعرض الخارجي
    \end{itemize}
\end{itemize}

\subsection{ضمان الجودة والتحسين المستمر}

\subsubsection{مراقبة الأداء}
\begin{itemize}
    \item \textbf{مقاييس أداء النظام:}
    \begin{itemize}
        \item اكتمال ودقة جمع البيانات
        \item وقت تشغيل النظام وموثوقيته
        \item سرعة المعالجة وكفاءتها
        \item دقة النموذج وصحته
    \end{itemize}
    
    \item \textbf{مقاييس التأثير الزراعي:}
    \begin{itemize}
        \item تحسينات كفاءة استخدام الموارد
        \item تعزيزات المحصول والجودة
        \item فعالية منع المشكلات
        \item نجاح تكامل الاقتصاد الدائري
    \end{itemize}
    
    \item \textbf{مقاييس جودة الخدمة:}
    \begin{itemize}
        \item رضا المستخدم ومشاركته
        \item معدل تنفيذ التوصيات
        \item وقت الاستجابة للمشكلات
        \item الاحتفاظ بالعملاء وتوسيعهم
    \end{itemize}
\end{itemize}

\subsubsection{عمليات التحسين}
\begin{itemize}
    \item \textbf{دورات المراجعة المنتظمة:}
    \begin{itemize}
        \item مراجعات تشغيلية يومية
        \item تحليل أداء أسبوعي
        \item تقييم استراتيجي شهري
        \item تقييم شامل موسمي
    \end{itemize}
    
    \item \textbf{دمج التغذية الراجعة:}
    \begin{itemize}
        \item جمع منظم لتعليقات المستخدمين
        \item تحليل منهجي لبيانات الأداء
        \item تحليل السبب الجذري للمشكلات
        \item إطار تحديد أولويات التحسينات
    \end{itemize}
    
    \item \textbf{إدارة الابتكار:}
    \begin{itemize}
        \item استكشاف التكنولوجيا وتقييمها
        \item اختبار تجريبي للنهج الجديدة
        \item تنفيذ منظم للتحسينات المصادق عليها
        \item مشاركة المعرفة لنتائج الابتكار
    \end{itemize}
\end{itemize}

\subsection{الأنشطة التشغيلية الرئيسية}

\subsubsection{الحصول على بيانات الاستشعار عن بعد}
\begin{itemize}
    \item الحصول المنتظم على صور الأقمار الصناعية (أسبوعيًا للدقة المتوسطة، شهريًا للدقة العالية)
    \item رحلات طائرات بدون طيار مجدولة للمراقبة التفصيلية (كل أسبوعين خلال مواسم النمو)
    \item معالجة الصور متعددة الأطياف والحرارية لإنشاء مؤشرات زراعية
    \item إنشاء وصيانة خرائط أساسية وكشف التغيرات الزمنية
\end{itemize}

\subsubsection{إدارة شبكة أجهزة استشعار إنترنت الأشياء}
\begin{itemize}
    \item نشر أجهزة استشعار رطوبة التربة في مواقع استراتيجية عبر جميع وحدات الزراعة
    \item تركيب وصيانة محطات الطقس لمراقبة المناخ المحلي
    \item نشر أجهزة استشعار متخصصة لجودة المياه وتدفق الري وصحة النبات
    \item المعايرة والصيانة المنتظمة لجميع معدات الاستشعار
    \item إدارة نظام نقل البيانات ومراقبتها في الوقت الفعلي
\end{itemize}

\subsubsection{تكامل البيانات وإدارتها}
\begin{itemize}
    \item جمع وتخزين البيانات من جميع أنظمة المراقبة في قاعدة بيانات مركزية
    \item إجراءات تنظيف البيانات والتحقق منها ومراقبة جودتها
    \item دمج مصادر البيانات المتنوعة في مجموعات بيانات موحدة
    \item تنفيذ بروتوكولات أمن البيانات والنسخ الاحتياطي
    \item تطوير واجهات برمجة التطبيقات (APIs) لمشاركة البيانات بين الوحدات
\end{itemize}

\subsubsection{التحليلات ودعم القرار}
\begin{itemize}
    \item تطوير خوارزميات مراقبة صحة المحاصيل
    \item إنشاء أدوات جدولة الري بناءً على بيانات رطوبة التربة والطقس
    \item تنفيذ أنظمة الإنذار المبكر للكشف عن الآفات والأمراض
    \item تطوير نماذج التنبؤ بالمحصول
    \item إنشاء خوارزميات تحسين الموارد لتطبيق المياه والأسمدة
    \item صيانة وتحديث النماذج التحليلية بناءً على التحقق الميداني
\end{itemize}

\subsubsection{التنفيذ والتحقق الميداني}
\begin{itemize}
    \item التحقق الميداني المنتظم من بيانات الاستشعار عن بعد
    \item التحقق الميداني من مخرجات النموذج التحليلي
    \item تنفيذ وصفات الزراعة الدقيقة
    \item التنسيق مع الوحدات الأخرى لتنفيذ التوصيات
    \item جمع التغذية الراجعة حول أداء النظام ودقته
\end{itemize}

\subsection{متطلبات الموارد}

\subsubsection{الموارد البشرية}
\begin{itemize}
    \item مدير وحدة واحد مع خبرة في التكنولوجيا الزراعية
    \item اثنان من متخصصي الاستشعار عن بعد
    \item اثنان من فنيي شبكة إنترنت الأشياء/أجهزة الاستشعار
    \item مهندس بيانات واحد
    \item عالم بيانات زراعية واحد
    \item اثنان من الفنيين الميدانيين
    \item متخصص واحد في التدريب ونقل المعرفة
\end{itemize}

\subsubsection{المعدات والبنية التحتية}
\begin{itemize}
    \item خدمات الاشتراك في بيانات الأقمار الصناعية
    \item أسطول من الطائرات بدون طيار الزراعية مع كاميرات متعددة الأطياف وحرارية
    \item شبكة من أجهزة استشعار رطوبة التربة ومحطات الطقس وأجهزة الاستشعار الزراعية المتخصصة
    \item أجهزة حوسبة طرفية لمعالجة البيانات المحلية
    \item بنية تحتية للخادم المركزي لتخزين البيانات ومعالجتها
    \item مركبات ميدانية للتحقق الميداني وصيانة المعدات
    \item برامج متخصصة لتحليل الاستشعار عن بعد وإدارة البيانات والتحليلات
\end{itemize}

\subsection{البروتوكولات التشغيلية}

\subsubsection{جمع البيانات ومعالجتها}
\begin{itemize}
    \item إجراءات التشغيل القياسية للحصول على بيانات الأقمار الصناعية ومعالجتها
    \item بروتوكولات تخطيط رحلات الطائرات بدون طيار والسلامة
    \item جداول نشر أجهزة الاستشعار ومعايرتها وصيانتها
    \item إجراءات مراقبة جودة البيانات والتحقق منها
    \item سير عمل تكامل البيانات ومعالجتها
\end{itemize}

\subsubsection{دعم القرار والتنفيذ}
\begin{itemize}
    \item بروتوكولات لترجمة المخرجات التحليلية إلى توصيات قابلة للتنفيذ
    \item تنسيقات قياسية للتواصل مع الوحدات الأخرى بشأن التوصيات
    \item إجراءات للتنبيهات الطارئة والاستجابة السريعة للمشكلات المكتشفة
    \item عمليات جمع التغذية الراجعة وتحسين النظام
    \item بروتوكولات التوثيق وإدارة المعرفة
\end{itemize}

\subsection{التكامل مع الوحدات الأخرى}
\begin{itemize}
    \item \textbf{المشتل:} توفير بيانات المناخ المحلي ومراقبة النمو لإنتاج الشتلات
    
    \item \textbf{زراعة النخيل والزيتون والأكاسيا:} تقديم مراقبة صحة المحاصيل، وجدولة الري، والإنذار المبكر للآفات/الأمراض
    
    \item \textbf{زراعة الأزولا:} مراقبة معايير جودة المياه وظروف النمو
    
    \item \textbf{إدارة الثروة الحيوانية:} توفير تقييم جودة المراعي وتوصيات تناوب الرعي
    
    \item \textbf{التسميد الدودي والفحم الحيوي:} مراقبة ظروف العملية وتوفير بيانات عن المناطق التي تتطلب تعديلات التربة
    
    \item \textbf{إنتاج الديزل الحيوي:} تتبع نمو محاصيل المواد الأولية ومعايير الجودة
\end{itemize} 