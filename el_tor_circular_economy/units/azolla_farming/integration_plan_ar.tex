\selectlanguage{arabic}
\section{خطة التكامل لزراعة الأزولا}

\subsection{التكامل المرحلي (2026-2031)}

\subsubsection{المرحلة الأولى (2026-2027)}
\begin{itemize}
    \item \textbf{المدخلات:}
    \begin{itemize}
        \item مياه صرف معالجة (100 متر مكعب/يوم)
        \item شاي السماد الدودي الأولي
        \item بنية تحتية أساسية للبرك
        \item إمداد طاقة شمسية
    \end{itemize}
    \item \textbf{المخرجات:}
    \begin{itemize}
        \item كتلة حيوية طازجة من الأزولا (5 أطنان سنوياً)
        \item مياه غنية بالمغذيات للري
        \item إنتاج أولي للأسمدة الحيوية
        \item توليد الأكسجين
    \end{itemize}
    \item \textbf{نقاط التكامل:}
    \begin{itemize}
        \item نظام معالجة المياه
        \item إمداد علف الماشية
        \item دعم الزراعة الأولي
    \end{itemize}
\end{itemize}

\subsubsection{المرحلة الثانية (2027-2028)}
\begin{itemize}
    \item \textbf{المدخلات:}
    \begin{itemize}
        \item توسيع معالجة مياه الصرف (300 متر مكعب/يوم)
        \item تحسين دورة المغذيات
        \item نظام برك موسع
        \item استخدام محسن للطاقة
    \end{itemize}
    \item \textbf{المخرجات:}
    \begin{itemize}
        \item زيادة إنتاج الكتلة الحيوية (15 طن سنوياً)
        \item تحسين جودة المياه
        \item توسيع نطاق الأسمدة الحيوية
        \item احتجاز الكربون
    \end{itemize}
    \item \textbf{نقاط التكامل:}
    \begin{itemize}
        \item وحدات زراعية متعددة
        \item تحسين علف الماشية
        \item إمداد مواد خام للديزل الحيوي
    \end{itemize}
\end{itemize}

\subsubsection{المرحلة الثالثة (2028-2029)}
\begin{itemize}
    \item \textbf{المدخلات:}
    \begin{itemize}
        \item تكامل كامل لمياه الصرف (500 متر مكعب/يوم)
        \item استعادة كاملة للمغذيات
        \item إدارة متقدمة للبرك
        \item كفاءة قصوى للطاقة
    \end{itemize}
    \item \textbf{المخرجات:}
    \begin{itemize}
        \item ذروة إنتاج الكتلة الحيوية (25 طن سنوياً)
        \item معالجة قصوى للمياه
        \item إنتاج كامل للأسمدة الحيوية
        \item خدمات نظام بيئي محسنة
    \end{itemize}
    \item \textbf{نقاط التكامل:}
    \begin{itemize}
        \item جميع الوحدات: دورة الموارد
        \item تكامل مرفق المعالجة
        \item توليد ائتمان الكربون
    \end{itemize}
\end{itemize}

\subsubsection{المرحلة الرابعة (2029-2030)}
\begin{itemize}
    \item \textbf{المدخلات:}
    \begin{itemize}
        \item أنظمة مياه محسنة (700 متر مكعب/يوم)
        \item إدارة ذكية للمغذيات
        \item تحكم آلي في البرك
        \item تكامل الطاقة المتجددة
    \end{itemize}
    \item \textbf{المخرجات:}
    \begin{itemize}
        \item منتجات كتلة حيوية متقدمة (50 طن سنوياً)
        \item جودة مياه ممتازة
        \item أسمدة متخصصة
        \item أقصى التقاط للكربون
    \end{itemize}
    \item \textbf{نقاط التكامل:}
    \begin{itemize}
        \item تكامل كامل للنظام
        \item معالجة ذات قيمة مضافة
        \item تحسين مقاييس الاستدامة
    \end{itemize}
\end{itemize}

\subsubsection{المرحلة الخامسة (2030-2031)}
\begin{itemize}
    \item \textbf{المدخلات:}
    \begin{itemize}
        \item سعة قصوى للنظام (1000 متر مكعب/يوم)
        \item مغذيات محسنة بالكامل
        \item تحكم ذكي في النظام
        \item كفاءة قصوى للطاقة
    \end{itemize}
    \item \textbf{المخرجات:}
    \begin{itemize}
        \item أقصى إنتاج للكتلة الحيوية (65 طن سنوياً)
        \item جودة مياه مثالية
        \item مجموعة منتجات كاملة
        \item فوائد قصوى للنظام البيئي
    \end{itemize}
    \item \textbf{نقاط التكامل:}
    \begin{itemize}
        \item تكامل كامل مع الاقتصاد الدائري
        \item تحسين كامل للموارد
        \item كفاءة قصوى للنظام
    \end{itemize}
\end{itemize}
