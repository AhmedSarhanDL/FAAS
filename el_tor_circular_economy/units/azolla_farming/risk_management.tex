\section{Risk Management}

\subsection{Implementation of Unified Risk Framework}

This risk management plan follows the standardized approach defined in the project-wide Unified Risk Management Framework (see Section \ref{sec:unified_risk_management}). It applies the standardized risk assessment methodology, categorization, and management processes while focusing on Azolla farming-specific risks and mitigation strategies.

\subsection{Unit-Specific Risk Assessment}

Following the risk categories defined in Section \ref{sec:risk_categories}, we have identified the following key risks specific to the Azolla Farming Unit:

\subsubsection{Environmental Risks}

\begin{table}[h]
\centering
\begin{tabular}{|p{3cm}|p{3cm}|p{2cm}|p{2cm}|p{4cm}|}
\hline
\textbf{Risk} & \textbf{Description} & \textbf{Likelihood (1-5)} & \textbf{Impact (1-5)} & \textbf{Mitigation Strategy} \\
\hline
Climate Extremes & Temperature variations outside Azolla's optimal growth range (20-30°C) & 4 & 4 & Greenhouse cultivation for temperature-controlled environments; seasonal adjustment of production targets \\
\hline
Water Quality Degradation & Contamination of cultivation ponds affecting Azolla growth & 3 & 5 & Regular water quality monitoring; filtration systems; water source diversification \\
\hline
Pest Invasions & Insects, fungi, or other organisms attacking Azolla colonies & 4 & 3 & Integrated pest management; isolation protocols; resistant strain development \\
\hline
Algal Competition & Excessive algal growth competing with Azolla for nutrients & 3 & 3 & Balanced nutrient management; shade optimization; regular skimming \\
\hline
\end{tabular}
\caption{Environmental Risks Specific to Azolla Farming}
\end{table}

\subsubsection{Operational Risks}

\begin{table}[h]
\centering
\begin{tabular}{|p{3cm}|p{3cm}|p{2cm}|p{2cm}|p{4cm}|}
\hline
\textbf{Risk} & \textbf{Description} & \textbf{Likelihood (1-5)} & \textbf{Impact (1-5)} & \textbf{Mitigation Strategy} \\
\hline
Biomass Production Shortfall & Failure to meet target production quantities & 3 & 4 & Cultivation area buffer; strain optimization; growth condition refinement \\
\hline
Harvesting Inefficiency & Equipment failure or inefficient harvesting processes & 2 & 3 & Redundant equipment systems; preventive maintenance; manual backup procedures \\
\hline
Contamination During Processing & Cross-contamination affecting Azolla quality & 2 & 4 & Strict hygiene protocols; staff training; quality testing regimes \\
\hline
Strain Degradation & Genetic drift or loss of productive characteristics & 3 & 4 & Strain banking; regular refreshment from pure cultures; monitoring of growth metrics \\
\hline
\end{tabular}
\caption{Operational Risks Specific to Azolla Farming}
\end{table}

\subsubsection{Technical Risks}

\begin{table}[h]
\centering
\begin{tabular}{|p{3cm}|p{3cm}|p{2cm}|p{2cm}|p{4cm}|}
\hline
\textbf{Risk} & \textbf{Description} & \textbf{Likelihood (1-5)} & \textbf{Impact (1-5)} & \textbf{Mitigation Strategy} \\
\hline
Nutrient Imbalance & Sub-optimal nutrient profiles affecting Azolla growth rate & 3 & 3 & Regular nutrient testing; automated dosing systems; water chemistry expertise \\
\hline
Monitoring System Failure & Malfunction of growth and environmental monitoring equipment & 2 & 4 & Redundant sensor networks; manual verification protocols; preventive maintenance \\
\hline
Irrigation System Failure & Breakdown of water circulation or delivery systems & 2 & 5 & Backup pumps and distribution systems; emergency water supply arrangements \\
\hline
\end{tabular}
\caption{Technical Risks Specific to Azolla Farming}
\end{table}

\subsection{Integration-Specific Risks}

These risks specifically relate to the Azolla Farming Unit's integration with other units in the El Tor Circular Economy project:

\begin{table}[h]
\centering
\begin{tabular}{|p{3cm}|p{3cm}|p{2cm}|p{2cm}|p{4cm}|}
\hline
\textbf{Risk} & \textbf{Description} & \textbf{Likelihood (1-5)} & \textbf{Impact (1-5)} & \textbf{Mitigation Strategy} \\
\hline
Biodiesel Production Demand Mismatch & Production capacity misaligned with Biodiesel Unit requirements & 3 & 4 & Coordinated production planning; buffer stock management; flexible scaling capacity \\
\hline
Nutrient Recycling Disruption & Interruption in nutrient flow from Livestock and Vermicomposting Units & 2 & 3 & Nutrient stockpiling; commercial backup sources; alternative formulations \\
\hline
Water Integration Failure & Breakdown in integrated water systems from Water Management Unit & 2 & 4 & Independent water storage; alternative water sources; water recycling optimization \\
\hline
\end{tabular}
\caption{Integration Risks Specific to Azolla Farming}
\end{table}

\subsection{Unit-Specific Risk Response Protocols}

In addition to the standard risk response strategies outlined in Section \ref{sec:risk_assessment_methodology}, the Azolla Farming Unit implements the following specific response protocols:

\begin{itemize}
    \item \textbf{Production Shortage Protocol:} Tiered response based on severity:
    \begin{itemize}
        \item Level 1 (10-20\% shortfall): Increase nutrient concentration and optimize growing conditions
        \item Level 2 (20-40\% shortfall): Activate reserve cultivation areas and extend harvesting cycles
        \item Level 3 (>40\% shortfall): Implement emergency production plan and coordinate with dependent units
    \end{itemize}
    
    \item \textbf{Contamination Response Protocol:}
    \begin{itemize}
        \item Immediate isolation of affected cultivation area
        \item Root cause analysis and containment measures
        \item Decontamination procedures following established guidelines
        \item Repopulation from clean starter cultures
    \end{itemize}
    
    \item \textbf{Extreme Weather Response Protocol:}
    \begin{itemize}
        \item 72-hour advance preparation for forecasted weather events
        \item Protection measures for vulnerable cultivation areas
        \item Accelerated harvesting if crop loss is anticipated
        \item Rapid recovery procedures post-event
    \end{itemize}
\end{itemize}

\subsection{Risk Monitoring and Review}

In accordance with Section \ref{sec:risk_monitoring}, the Azolla Farming Unit implements the following unit-specific monitoring mechanisms:

\begin{itemize}
    \item \textbf{Daily Monitoring:}
    \begin{itemize}
        \item Growth rate and health indicators in each cultivation area
        \item Water quality parameters (pH, temperature, nutrient levels)
        \item Visual inspection for pests, contamination, or abnormalities
    \end{itemize}
    
    \item \textbf{Weekly Assessments:}
    \begin{itemize}
        \item Biomass production rate vs. targets
        \item Strain performance and consistency
        \item Equipment functionality and efficiency
        \item Integration points with other units
    \end{itemize}
    
    \item \textbf{Monthly Risk Review:}
    \begin{itemize}
        \item Comprehensive risk register update
        \item Effectiveness assessment of current mitigation strategies
        \item Early warning indicator review
        \item Emerging risk identification
    \end{itemize}
\end{itemize}

\subsection{Risk Management Responsibilities}

Following the responsibility structure in Section \ref{sec:risk_responsibilities}, specific roles within the Azolla Farming Unit include:

\begin{itemize}
    \item \textbf{Unit Manager:} Overall accountability for risk management implementation
    \item \textbf{Cultivation Specialist:} Monitoring and managing production-related risks
    \item \textbf{Technical Coordinator:} Overseeing systems and equipment risks
    \item \textbf{Quality Assurance Officer:} Managing contamination and quality risks
    \item \textbf{Integration Liaison:} Coordinating cross-unit risk management
\end{itemize}

This risk management plan will be reviewed quarterly and updated annually, with additional updates as significant changes occur in operations, technology, or the broader project context.
