\section{\RL{إدارة المخاطر}}

\subsection{\RL{تطبيق إطار المخاطر الموحد}}

\RL{تتبع خطة إدارة المخاطر هذه النهج الموحد المحدد في إطار إدارة المخاطر الموحد على مستوى المشروع (انظر القسم} \ref{sec:unified_risk_management_ar}\RL{). وهي تطبق منهجية تقييم المخاطر الموحدة والتصنيف وعمليات الإدارة مع التركيز على المخاطر واستراتيجيات التخفيف الخاصة بزراعة الأزولا.}

\subsection{\RL{تقييم المخاطر الخاصة بالوحدة}}

\RL{بناءً على فئات المخاطر المحددة في القسم} \ref{sec:risk_categories_ar}\RL{، حددنا المخاطر الرئيسية التالية الخاصة بوحدة زراعة الأزولا:}

\subsubsection{\RL{المخاطر البيئية}}

\begin{table}[h]
\centering
\begin{tabular}{|p{3cm}|p{3cm}|p{2cm}|p{2cm}|p{4cm}|}
\hline
\textbf{\RL{المخاطر}} & \textbf{\RL{الوصف}} & \textbf{\RL{الاحتمالية (1-5)}} & \textbf{\RL{التأثير (1-5)}} & \textbf{\RL{استراتيجية التخفيف}} \\
\hline
\RL{الظروف المناخية القصوى} & \RL{تغيرات درجة الحرارة خارج نطاق النمو الأمثل للأزولا (20-30 درجة مئوية)} & 4 & 4 & \RL{زراعة البيوت المحمية للبيئات ذات درجة الحرارة المتحكم فيها؛ تعديل أهداف الإنتاج الموسمية} \\
\hline
\RL{تدهور جودة المياه} & \RL{تلوث برك الزراعة مما يؤثر على نمو الأزولا} & 3 & 5 & \RL{المراقبة المنتظمة لجودة المياه؛ أنظمة الترشيح؛ تنويع مصادر المياه} \\
\hline
\RL{غزو الآفات} & \RL{الحشرات والفطريات أو الكائنات الأخرى التي تهاجم مستعمرات الأزولا} & 4 & 3 & \RL{الإدارة المتكاملة للآفات؛ بروتوكولات العزل؛ تطوير سلالات مقاومة} \\
\hline
\RL{منافسة الطحالب} & \RL{نمو الطحالب المفرط المنافس للأزولا على العناصر الغذائية} & 3 & 3 & \RL{إدارة متوازنة للمغذيات؛ تحسين الظل؛ الإزالة المنتظمة} \\
\hline
\end{tabular}
\caption{\RL{المخاطر البيئية الخاصة بزراعة الأزولا}}
\end{table}

\subsubsection{\RL{المخاطر التشغيلية}}

\begin{table}[h]
\centering
\begin{tabular}{|p{3cm}|p{3cm}|p{2cm}|p{2cm}|p{4cm}|}
\hline
\textbf{\RL{المخاطر}} & \textbf{\RL{الوصف}} & \textbf{\RL{الاحتمالية (1-5)}} & \textbf{\RL{التأثير (1-5)}} & \textbf{\RL{استراتيجية التخفيف}} \\
\hline
\RL{نقص إنتاج الكتلة الحيوية} & \RL{عدم تلبية كميات الإنتاج المستهدفة} & 3 & 4 & \RL{احتياطي مساحة الزراعة؛ تحسين السلالة؛ تحسين ظروف النمو} \\
\hline
\RL{عدم كفاءة الحصاد} & \RL{تعطل المعدات أو عدم كفاءة عمليات الحصاد} & 2 & 3 & \RL{أنظمة معدات احتياطية؛ صيانة وقائية؛ إجراءات يدوية احتياطية} \\
\hline
\RL{التلوث أثناء المعالجة} & \RL{التلوث المتبادل الذي يؤثر على جودة الأزولا} & 2 & 4 & \RL{بروتوكولات نظافة صارمة؛ تدريب الموظفين؛ أنظمة اختبار الجودة} \\
\hline
\RL{تدهور السلالة} & \RL{الانحراف الجيني أو فقدان الخصائص الإنتاجية} & 3 & 4 & \RL{بنك السلالات؛ التجديد المنتظم من الثقافات النقية؛ مراقبة مقاييس النمو} \\
\hline
\end{tabular}
\caption{\RL{المخاطر التشغيلية الخاصة بزراعة الأزولا}}
\end{table}

\subsubsection{\RL{المخاطر التقنية}}

\begin{table}[h]
\centering
\begin{tabular}{|p{3cm}|p{3cm}|p{2cm}|p{2cm}|p{4cm}|}
\hline
\textbf{\RL{المخاطر}} & \textbf{\RL{الوصف}} & \textbf{\RL{الاحتمالية (1-5)}} & \textbf{\RL{التأثير (1-5)}} & \textbf{\RL{استراتيجية التخفيف}} \\
\hline
\RL{عدم توازن المغذيات} & \RL{ملفات تعريف المغذيات دون المستوى الأمثل تؤثر على معدل نمو الأزولا} & 3 & 3 & \RL{اختبار المغذيات بانتظام؛ أنظمة جرعات آلية؛ خبرة في كيمياء المياه} \\
\hline
\RL{فشل نظام المراقبة} & \RL{خلل في معدات مراقبة النمو والبيئة} & 2 & 4 & \RL{شبكات استشعار احتياطية؛ بروتوكولات التحقق اليدوي؛ صيانة وقائية} \\
\hline
\RL{فشل نظام الري} & \RL{تعطل أنظمة توزيع المياه} & 2 & 5 & \RL{مضخات احتياطية وأنظمة توزيع؛ ترتيبات إمدادات المياه في حالات الطوارئ} \\
\hline
\end{tabular}
\caption{\RL{المخاطر التقنية الخاصة بزراعة الأزولا}}
\end{table}

\subsection{\RL{مخاطر محددة للتكامل}}

\RL{تتعلق هذه المخاطر تحديدًا بتكامل وحدة زراعة الأزولا مع الوحدات الأخرى في مشروع اقتصاد الطور الدائري:}

\begin{table}[h]
\centering
\begin{tabular}{|p{3cm}|p{3cm}|p{2cm}|p{2cm}|p{4cm}|}
\hline
\textbf{\RL{المخاطر}} & \textbf{\RL{الوصف}} & \textbf{\RL{الاحتمالية (1-5)}} & \textbf{\RL{التأثير (1-5)}} & \textbf{\RL{استراتيجية التخفيف}} \\
\hline
\RL{عدم تطابق الطلب على إنتاج الديزل الحيوي} & \RL{القدرة الإنتاجية غير متوافقة مع متطلبات وحدة الديزل الحيوي} & 3 & 4 & \RL{تخطيط الإنتاج المنسق؛ إدارة المخزون الاحتياطي؛ قدرة التوسع المرنة} \\
\hline
\RL{اضطراب إعادة تدوير المغذيات} & \RL{انقطاع في تدفق المغذيات من وحدات الثروة الحيوانية والتسميد الدودي} & 2 & 3 & \RL{تخزين المغذيات؛ مصادر تجارية احتياطية؛ تركيبات بديلة} \\
\hline
\RL{فشل تكامل المياه} & \RL{انهيار في أنظمة المياه المتكاملة من وحدة إدارة المياه} & 2 & 4 & \RL{تخزين المياه المستقلة؛ مصادر المياه البديلة؛ تحسين إعادة تدوير المياه} \\
\hline
\end{tabular}
\caption{\RL{مخاطر التكامل الخاصة بزراعة الأزولا}}
\end{table}

\subsection{\RL{بروتوكولات الاستجابة للمخاطر الخاصة بالوحدة}}

\RL{بالإضافة إلى استراتيجيات الاستجابة للمخاطر القياسية الموضحة في القسم} \ref{sec:risk_assessment_methodology_ar}\RL{، تنفذ وحدة زراعة الأزولا بروتوكولات الاستجابة المحددة التالية:}

\begin{itemize}
    \item \textbf{\RL{بروتوكول نقص الإنتاج:}} \RL{استجابة متدرجة حسب الشدة:}
    \begin{itemize}
        \item \RL{المستوى 1 (نقص 10-20\%): زيادة تركيز المغذيات وتحسين ظروف النمو}
        \item \RL{المستوى 2 (نقص 20-40\%): تنشيط مناطق الزراعة الاحتياطية وتمديد دورات الحصاد}
        \item \RL{المستوى 3 (نقص > 40\%): تنفيذ خطة إنتاج طارئة والتنسيق مع الوحدات المعتمدة}
    \end{itemize}
    
    \item \textbf{\RL{بروتوكول الاستجابة للتلوث:}}
    \begin{itemize}
        \item \RL{العزل الفوري لمنطقة الزراعة المتأثرة}
        \item \RL{تحليل السبب الجذري وإجراءات الاحتواء}
        \item \RL{إجراءات إزالة التلوث اتباعًا للإرشادات المعمول بها}
        \item \RL{إعادة التأهيل من مزارع بادئة نظيفة}
    \end{itemize}
    
    \item \textbf{\RL{بروتوكول الاستجابة للطقس القاسي:}}
    \begin{itemize}
        \item \RL{الاستعداد المسبق بـ 72 ساعة للأحداث الجوية المتوقعة}
        \item \RL{تدابير حماية لمناطق الزراعة المعرضة للخطر}
        \item \RL{تسريع الحصاد إذا كان فقدان المحصول متوقعًا}
        \item \RL{إجراءات التعافي السريع بعد الحدث}
    \end{itemize}
\end{itemize}

\subsection{\RL{مراقبة المخاطر ومراجعتها}}

\RL{وفقًا للقسم} \ref{sec:risk_monitoring_ar}\RL{، تنفذ وحدة زراعة الأزولا آليات المراقبة الخاصة بالوحدة التالية:}

\begin{itemize}
    \item \textbf{\RL{المراقبة اليومية:}}
    \begin{itemize}
        \item \RL{معدل النمو ومؤشرات الصحة في كل منطقة زراعة}
        \item \RL{معايير جودة المياه (الرقم الهيدروجيني، درجة الحرارة، مستويات المغذيات)}
        \item \RL{الفحص البصري للآفات والتلوث أو الشذوذ}
    \end{itemize}
    
    \item \textbf{\RL{التقييمات الأسبوعية:}}
    \begin{itemize}
        \item \RL{معدل إنتاج الكتلة الحيوية مقابل الأهداف}
        \item \RL{أداء السلالة واتساقها}
        \item \RL{وظائف المعدات وكفاءتها}
        \item \RL{نقاط التكامل مع الوحدات الأخرى}
    \end{itemize}
    
    \item \textbf{\RL{مراجعة المخاطر الشهرية:}}
    \begin{itemize}
        \item \RL{تحديث شامل لسجل المخاطر}
        \item \RL{تقييم فعالية استراتيجيات التخفيف الحالية}
        \item \RL{مراجعة مؤشرات الإنذار المبكر}
        \item \RL{تحديد المخاطر الناشئة}
    \end{itemize}
\end{itemize}

\subsection{\RL{مسؤوليات إدارة المخاطر}}

\RL{اتباعًا لهيكل المسؤولية في القسم} \ref{sec:risk_responsibilities_ar}\RL{، تشمل الأدوار المحددة داخل وحدة زراعة الأزولا:}

\begin{itemize}
    \item \textbf{\RL{مدير الوحدة:}} \RL{المساءلة الشاملة عن تنفيذ إدارة المخاطر}
    \item \textbf{\RL{أخصائي الزراعة:}} \RL{مراقبة وإدارة المخاطر المتعلقة بالإنتاج}
    \item \textbf{\RL{المنسق الفني:}} \RL{الإشراف على مخاطر الأنظمة والمعدات}
    \item \textbf{\RL{مسؤول ضمان الجودة:}} \RL{إدارة مخاطر التلوث والجودة}
    \item \textbf{\RL{مسؤول التكامل:}} \RL{تنسيق إدارة المخاطر عبر الوحدات}
\end{itemize}

\RL{ستتم مراجعة خطة إدارة المخاطر هذه كل ثلاثة أشهر وتحديثها سنويًا، مع تحديثات إضافية حسب حدوث تغييرات كبيرة في العمليات أو التكنولوجيا أو سياق المشروع الأوسع.}
