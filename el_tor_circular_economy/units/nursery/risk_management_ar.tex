\section{خطة إدارة المخاطر لوحدة المشتل}

\subsection{نهج إدارة المخاطر}
تحدد خطة إدارة المخاطر هذه وتحلل وتضع استراتيجيات استجابة للمخاطر المحتملة التي قد تؤثر على نجاح إنشاء وتشغيل وحدة المشتل ضمن مشروع الاقتصاد الدائري في الطور. تتبنى الخطة نهجًا استباقيًا لإدارة المخاطر، مع التركيز على التحديد المبكر والمراقبة المستمرة واستراتيجيات الاستجابة التكيفية لتقليل الآثار السلبية وتعظيم الفرص.

\subsection{تحديد وتقييم المخاطر}

\subsubsection{المخاطر البيئية}

\begin{table}[h]
\centering
\begin{tabular}{|p{3cm}|p{4cm}|p{2cm}|p{2cm}|p{3cm}|}
\hline
\textbf{الخطر} & \textbf{الوصف} & \textbf{الاحتمالية} & \textbf{التأثير} & \textbf{مستوى الخطر} \\
\hline
الظواهر الجوية المتطرفة & العواصف الرملية والرياح القوية أو الحرارة الشديدة التي تؤثر على هياكل البيوت المحمية وصحة النبات & عالية & عالي & حرج \\
\hline
انقطاع إمدادات المياه & انقطاع أو تلوث إمدادات المياه مما يؤثر على أنظمة الري & متوسطة & عالي & عالي \\
\hline
تلوث التربة/الوسائط & دخول مسببات الأمراض أو السموم إلى وسائط النمو & منخفضة & عالي & متوسط \\
\hline
تقلبات المناخ & تقلبات غير متوقعة في درجة الحرارة أو الرطوبة تؤثر على تطور النبات & متوسطة & متوسط & متوسط \\
\hline
غزو الآفات & دخول آفات جديدة أو مقاومة إلى البيئة المتحكم بها & متوسطة & عالي & عالي \\
\hline
\end{tabular}
\caption{تقييم المخاطر البيئية}
\end{table}

\subsubsection{المخاطر التقنية والتشغيلية}

\begin{table}[h]
\centering
\begin{tabular}{|p{3cm}|p{4cm}|p{2cm}|p{2cm}|p{3cm}|}
\hline
\textbf{الخطر} & \textbf{الوصف} & \textbf{الاحتمالية} & \textbf{التأثير} & \textbf{مستوى الخطر} \\
\hline
فشل المعدات & أعطال في الأنظمة الحرجة للتحكم في المناخ أو الري أو معدات المختبر & متوسطة & عالي & عالي \\
\hline
انقطاع التيار الكهربائي & انقطاع إمدادات الكهرباء مما يؤثر على أنظمة التحكم في المناخ والري & متوسطة & عالي & عالي \\
\hline
فشل الإكثار & معدلات نجاح منخفضة في تقنيات إكثار الأصناف الرئيسية & متوسطة & عالي & عالي \\
\hline
تفشي الأمراض & انتشار أمراض النبات داخل بيئة المشتل & متوسطة & حرج & عالي \\
\hline
فجوات المهارات التقنية & نقص الخبرة التقنية للعمليات المتخصصة & متوسطة & متوسط & متوسط \\
\hline
\end{tabular}
\caption{تقييم المخاطر التقنية والتشغيلية}
\end{table}

\subsubsection{المخاطر المالية ومخاطر الموارد}

\begin{table}[h]
\centering
\begin{tabular}{|p{3cm}|p{4cm}|p{2cm}|p{2cm}|p{3cm}|}
\hline
\textbf{الخطر} & \textbf{الوصف} & \textbf{الاحتمالية} & \textbf{التأثير} & \textbf{مستوى الخطر} \\
\hline
تجاوزات الميزانية & تكاليف تتجاوز الميزانية المخططة للبناء أو العمليات & متوسطة & عالي & عالي \\
\hline
اضطرابات سلسلة التوريد & تأخير أو عدم توفر المستلزمات والمواد الحرجة & متوسطة & متوسط & متوسط \\
\hline
دوران الموظفين & فقدان موظفين فنيين رئيسيين ذوي معرفة متخصصة & متوسطة & عالي & عالي \\
\hline
المنافسة على الموارد & المنافسة على الموارد مع وحدات المشروع الأخرى & منخفضة & متوسط & منخفض \\
\hline
تأخير التمويل & تأخير في تلقي التمويل المخطط مما يؤثر على جدول التنفيذ & متوسطة & عالي & عالي \\
\hline
\end{tabular}
\caption{تقييم المخاطر المالية ومخاطر الموارد}
\end{table}

\subsubsection{مخاطر السوق والمخاطر الاستراتيجية}

\begin{table}[h]
\centering
\begin{tabular}{|p{3cm}|p{4cm}|p{2cm}|p{2cm}|p{3cm}|}
\hline
\textbf{الخطر} & \textbf{الوصف} & \textbf{الاحتمالية} & \textbf{التأثير} & \textbf{مستوى الخطر} \\
\hline
تقلبات الطلب & تغييرات في الطلب على أصناف أو كميات محددة & متوسطة & متوسط & متوسط \\
\hline
تغييرات معايير الجودة & تطور متطلبات الجودة من وحدات الزراعة & منخفضة & متوسط & منخفض \\
\hline
ضغط المنافسة & المنافسة من المشاتل أو مرافق الإكثار الأخرى & منخفضة & متوسط & منخفض \\
\hline
الوصول إلى المادة الوراثية & صعوبات في الوصول إلى مادة وراثية عالية الجودة للإكثار & متوسطة & عالي & عالي \\
\hline
التغييرات التنظيمية & تغييرات في اللوائح التي تؤثر على الإكثار أو نقل النباتات & منخفضة & عالي & متوسط \\
\hline
\end{tabular}
\caption{تقييم مخاطر السوق والمخاطر الاستراتيجية}
\end{table}

\subsection{استراتيجيات الاستجابة للمخاطر}

\subsubsection{تخفيف المخاطر البيئية}

\begin{itemize}
    \item \textbf{الظواهر الجوية المتطرفة:}
    \begin{itemize}
        \item تصميم هياكل البيوت المحمية لتحمل ظروف الرياح المحلية
        \item تركيب حواجز واقية ومصدات للرياح حول المرافق
        \item تنفيذ بروتوكولات طوارئ للظواهر الجوية المتطرفة
        \item تطوير إجراءات إخلاء للمواد النباتية الحساسة
        \item تركيب أنظمة إنذار مبكر للظواهر الجوية
    \end{itemize}
    
    \item \textbf{انقطاع إمدادات المياه:}
    \begin{itemize}
        \item تركيب سعة تخزين مياه تكفي لـ 7-10 أيام من العمليات
        \item تنفيذ أنظمة إعادة تدوير المياه والحفاظ عليها
        \item تطوير خطط طوارئ لمصادر مياه بديلة
        \item تركيب أنظمة مراقبة ومعالجة جودة المياه
        \item إنشاء ترتيبات احتياطية لتوصيل المياه
    \end{itemize}
    
    \item \textbf{إدارة الآفات والأمراض:}
    \begin{itemize}
        \item تنفيذ بروتوكولات أمن حيوي صارمة لجميع المواد الواردة
        \item إنشاء إجراءات حجر صحي للمواد النباتية الجديدة
        \item تطوير استراتيجيات الإدارة المتكاملة للآفات
        \item تدريب الموظفين على إجراءات الكشف المبكر والاستجابة
        \item الحفاظ على التنوع في مجموعات الكائنات المفيدة
    \end{itemize}
\end{itemize}

\subsubsection{تخفيف المخاطر التقنية والتشغيلية}

\begin{itemize}
    \item \textbf{موثوقية المعدات والأنظمة:}
    \begin{itemize}
        \item تنفيذ جداول صيانة وقائية لجميع الأنظمة الحرجة
        \item تركيب أنظمة احتياطية للوظائف الحرجة
        \item الاحتفاظ بمخزون من قطع الغيار الأساسية
        \item تدريب عدة موظفين على تشغيل المعدات واستكشاف الأخطاء وإصلاحها
        \item إنشاء عقود خدمة مع موردي المعدات
    \end{itemize}
    
    \item \textbf{أمن إمدادات الطاقة:}
    \begin{itemize}
        \item تركيب أنظمة طاقة شمسية مع تخزين البطاريات
        \item الاحتفاظ بمولدات احتياطية مع تبديل تلقائي
        \item تطوير إجراءات تجاوز يدوية للأنظمة الحرجة
        \item تنفيذ تصميمات موفرة للطاقة لتقليل متطلبات الطاقة
        \item إنشاء بروتوكولات أولوية لتخصيص الطاقة أثناء النقص
    \end{itemize}
    
    \item \textbf{نجاح الإكثار:}
    \begin{itemize}
        \item تطوير واختبار طرق إكثار متعددة لكل نوع
        \item الاحتفاظ بسجلات مفصلة لعوامل نجاح الإكثار
        \item تنفيذ تحسين مستمر في بروتوكولات الإكثار
        \item إنشاء شراكات مع مؤسسات بحثية للدعم الفني
        \item الحفاظ على مصادر متنوعة للمادة الوراثية
    \end{itemize}
\end{itemize}

\subsubsection{تخفيف المخاطر المالية ومخاطر الموارد}

\begin{itemize}
    \item \textbf{إدارة الميزانية:}
    \begin{itemize}
        \item تنفيذ نهج تطوير مرحلي مع معالم واضحة
        \item الاحتفاظ باحتياطيات طوارئ (15\% من إجمالي الميزانية)
        \item إجراء مراجعات منتظمة للميزانية والتنبؤ
        \item تطوير ترتيبات تقاسم التكاليف مع وحدات المشروع الأخرى
        \item تحديد المجالات المحتملة لتخفيض التكاليف إذا لزم الأمر
    \end{itemize}
    
    \item \textbf{أمن سلسلة التوريد:}
    \begin{itemize}
        \item تطوير علاقات مع موردين متعددين للعناصر الحرجة
        \item الاحتفاظ بمخزون من المستلزمات الأساسية لمدة 3-6 أشهر
        \item تحديد بدائل محلية للمواد المستوردة حيثما أمكن
        \item تنفيذ إدارة مخزون في الوقت المناسب للعناصر غير الحرجة
        \item تطوير القدرة على إنتاج بعض المستلزمات داخليًا
    \end{itemize}
    
    \item \textbf{إدارة الموارد البشرية:}
    \begin{itemize}
        \item تنفيذ حزم تعويضات ومزايا تنافسية
        \item تطوير فرص التقدم الوظيفي للموظفين الفنيين
        \item إنشاء أنظمة إدارة المعرفة لالتقاط الخبرة
        \item تنفيذ برامج تدريب متبادل للوظائف الحرجة
        \item تطوير شراكات مع المؤسسات التعليمية لخط إمداد المواهب
    \end{itemize}
\end{itemize}

\subsubsection{تخفيف مخاطر السوق والمخاطر الاستراتيجية}

\begin{itemize}
    \item \textbf{إدارة الطلب:}
    \begin{itemize}
        \item تنفيذ أنظمة تخطيط إنتاج مرنة
        \item تطوير محفظة منتجات متنوعة تتجاوز الأصناف الأساسية
        \item إنشاء قنوات اتصال منتظمة مع وحدات الزراعة
        \item إجراء أبحاث سوق لفرص المبيعات الخارجية
        \item تطوير القدرة على تعديل أحجام الإنتاج بناءً على الطلب
    \end{itemize}
    
    \item \textbf{ضمان الجودة:}
    \begin{itemize}
        \item تنفيذ نظام شامل لإدارة الجودة
        \item تطوير معايير جودة واضحة وإجراءات اعتماد
        \item إقامة اجتماعات مراجعة جودة منتظمة مع وحدات الزراعة
        \item تنفيذ أنظمة تتبع لجميع المواد النباتية
        \item تطوير القدرة على التحسين المستمر للجودة
    \end{itemize}
    
    \item \textbf{إدارة الموارد الوراثية:}
    \begin{itemize}
        \item إنشاء برنامج للحفاظ الوراثي للأصناف الرئيسية
        \item تطوير علاقات مع موردين متعددين للمواد الوراثية
        \item تنفيذ توثيق مناسب وإدارة الملكية الفكرية
        \item المشاركة في شبكات تبادل الموارد الوراثية
        \item تطوير القدرة على تحسين الأصناف داخليًا
    \end{itemize}
\end{itemize}

\subsection{التخطيط للطوارئ}

\subsubsection{إجراءات الاستجابة للطوارئ}
\begin{itemize}
    \item \textbf{حالات الطوارئ البيئية:}
    \begin{itemize}
        \item بروتوكول الاستجابة للطقس القاسي
        \item خطة الاستجابة لتلوث المياه
        \item إجراءات احتواء تفشي الآفات والأمراض
        \item إدارة التلوث البيئي
    \end{itemize}
    
    \item \textbf{حالات الطوارئ التقنية:}
    \begin{itemize}
        \item بروتوكول الاستجابة لانقطاع التيار الكهربائي
        \item إجراءات فشل نظام التحكم في المناخ
        \item إدارة فشل نظام الري
        \item الاستجابة لتلوث المختبر
    \end{itemize}
    
    \item \textbf{حالات الطوارئ التشغيلية:}
    \begin{itemize}
        \item خطة إدارة نقص الموظفين
        \item إجراءات الاستجابة لنقص المستلزمات
        \item بروتوكولات فشل الاتصال
        \item خطة الاستجابة لأضرار المرافق
    \end{itemize}
\end{itemize}

\subsubsection{تخطيط استمرارية الأعمال}
\begin{itemize}
    \item \textbf{تحديد الوظائف الحرجة:}
    \begin{itemize}
        \item تحديد أولويات المواد النباتية بناءً على القيمة والضعف
        \item تحديد الحد الأدنى من العمليات القابلة للتطبيق
        \item متطلبات الموظفين والمهارات الحرجة
        \item متطلبات الموارد الأساسية
    \end{itemize}
    
    \item \textbf{استراتيجيات التعافي:}
    \begin{itemize}
        \item ترتيبات المرافق المؤقتة
        \item طرق إكثار بديلة
        \item احتمالات التوريد الخارجي
        \item تخطيط التعافي المرحلي
    \end{itemize}
    
    \item \textbf{خطة الاتصال:}
    \begin{itemize}
        \item إجراءات الاتصال في حالات الطوارئ
        \item بروتوكولات إخطار أصحاب المصلحة
        \item إرشادات الاتصال الإعلامي
        \item إجراءات تبادل المعلومات الداخلية
    \end{itemize}
\end{itemize}

\subsection{مراقبة المخاطر والتحكم فيها}

\subsubsection{إجراءات مراقبة المخاطر}
\begin{itemize}
    \item مراجعات تقييم المخاطر المنتظمة (ربع سنوية)
    \item مراقبة وإعداد تقارير عن مؤشرات المخاطر الرئيسية
    \item دمج مراقبة المخاطر مع نظام إدارة الجودة
    \item آليات إبلاغ الموظفين لتحديد المخاطر
    \item مسح البيئة الخارجية للمخاطر الناشئة
\end{itemize}

\subsubsection{تقييم الاستجابة للمخاطر}
\begin{itemize}
    \item إجراءات تحليل ما بعد الحادث
    \item تقييم فعالية استجابات المخاطر
    \item توثيق الدروس المستفادة ومشاركتها
    \item تحديثات استراتيجية الاستجابة للمخاطر بناءً على النتائج
    \item التحسين المستمر في ممارسات إدارة المخاطر
\end{itemize}

\subsubsection{مسؤوليات إدارة المخاطر}
\begin{itemize}
    \item مدير المشتل: المسؤولية الشاملة عن إدارة المخاطر
    \item المشرف الفني: مراقبة المخاطر التقنية والتشغيلية
    \item منسق البحوث: إدارة مخاطر البحث والتطوير
    \item جميع الموظفين: تحديد المخاطر والإبلاغ عنها
    \item مكتب إدارة المشروع: الإشراف على المخاطر والتكامل
\end{itemize}

\subsection{إدارة الفرص}

\subsubsection{تحديد الفرص}
\begin{itemize}
    \item \textbf{الابتكار التقني:}
    \begin{itemize}
        \item تطوير تقنيات إكثار متقدمة
        \item بحث تحسين التحكم في المناخ
        \item تحسينات تركيبة وسائط النمو
        \item تعزيزات الأتمتة والكفاءة
    \end{itemize}
    
    \item \textbf{تطوير السوق:}
    \begin{itemize}
        \item تطوير أصناف متخصصة
        \item توسيع السوق الخارجي
        \item عروض خدمات ذات قيمة مضافة
        \item برامج نقل المعرفة والتدريب
    \end{itemize}
    
    \item \textbf{التميز التشغيلي:}
    \begin{itemize}
        \item تحسينات كفاءة الموارد
        \item إنجازات شهادات الجودة
        \item تطوير الموظفين والتخصص
        \item تحسين العمليات وتوحيدها
    \end{itemize}
\end{itemize}

\subsubsection{استراتيجيات استغلال الفرص}
\begin{itemize}
    \item تخطيط استثمار البحث والتطوير
    \item تطوير الشراكات الاستراتيجية
    \item برامج حوافز ابتكار الموظفين
    \item منصات مشاركة المعرفة والتعاون
    \item تطوير ثقافة التحسين المستمر
\end{itemize}

توفر خطة إدارة المخاطر هذه إطارًا شاملاً لتحديد وتقييم والاستجابة للمخاطر التي قد تؤثر على نجاح وحدة المشتل. من خلال تنفيذ هذه الاستراتيجيات، ستكون وحدة المشتل في وضع أفضل للتعامل مع التحديات والاستفادة من الفرص، مما يضمن دورها الحيوي في دعم وحدات زراعة الزيتون ونخيل التمر ضمن مشروع الاقتصاد الدائري في الطور. 