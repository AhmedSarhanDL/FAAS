\section{متطلبات الموارد لوحدة المشتل}

\subsection{متطلبات الأرض والبنية التحتية}

\subsubsection{متطلبات الأرض}
\begin{itemize}
    \item \textbf{إجمالي مساحة الأرض:} 1.5 هكتار (15,000 متر مربع)
    \begin{itemize}
        \item منطقة البيوت المحمية: 2,000 متر مربع
        \item منطقة بيوت الظل: 1,000 متر مربع
        \item قسم النباتات الأم: 500 متر مربع
        \item المختبر والمرافق: 800 متر مربع
        \item التخزين والمعالجة: 700 متر مربع
        \item طرق الوصول والمرافق: 5,000 متر مربع
        \item منطقة عازلة وتوسعة: 5,000 متر مربع
    \end{itemize}
    
    \item \textbf{خصائص الأرض:}
    \begin{itemize}
        \item تربة جيدة الصرف
        \item حماية من الرياح القوية
        \item سهولة الوصول للنقل
        \item قرب من وحدات الزراعة
        \item مناسبة لتوصيلات المرافق
    \end{itemize}
\end{itemize}

\subsubsection{البنية التحتية للمباني}
\begin{itemize}
    \item \textbf{مرافق البيوت المحمية:}
    \begin{itemize}
        \item بيوت محمية متحكمة في المناخ: 2,000 متر مربع
        \item بناء من البولي كربونات أو الزجاج
        \item أنظمة تهوية آلية
        \item ستائر حرارية لتنظيم درجة الحرارة
        \item مناضد مرتفعة لإنتاج الحاويات
    \end{itemize}
    
    \item \textbf{مرافق بيوت الظل:}
    \begin{itemize}
        \item هياكل الظل: 1,000 متر مربع
        \item تغطية قماش الظل 30-50\%
        \item بناء إطار معزز
        \item حواجز حماية من الرياح
        \item أرضيات خرسانية مع تصريف
    \end{itemize}
    
    \item \textbf{مرافق المختبر والمكاتب:}
    \begin{itemize}
        \item مختبر زراعة الأنسجة: 200 متر مربع
        \item منطقة اختبار الجودة: 100 متر مربع
        \item مساحة البحث والتطوير: 150 متر مربع
        \item مكاتب الموظفين: 150 متر مربع
        \item غرفة اجتماعات وتدريب: 100 متر مربع
        \item مناطق التعقيم وتغيير الملابس: 100 متر مربع
    \end{itemize}
    
    \item \textbf{التخزين والمعالجة:}
    \begin{itemize}
        \item منطقة تحضير وسائط النمو: 200 متر مربع
        \item تخزين الحاويات والمستلزمات: 200 متر مربع
        \item تخزين المعدات: 150 متر مربع
        \item منطقة معالجة النباتات والتجهيز: 150 متر مربع
    \end{itemize}
\end{itemize}

\subsection{متطلبات المعدات والتكنولوجيا}

\subsubsection{معدات الإكثار}
\begin{itemize}
    \item \textbf{معدات زراعة الأنسجة:}
    \begin{itemize}
        \item كبائن التدفق الصفحي (3 وحدات)
        \item أجهزة التعقيم (2 وحدة)
        \item غرف الحضانة (4 وحدات)
        \item مجاهر (2 وحدة)
        \item معدات تحضير وسط النمو
        \item معدات التعقيم
    \end{itemize}
    
    \item \textbf{الإكثار التقليدي:}
    \begin{itemize}
        \item أنظمة الإكثار بالرذاذ
        \item مناضد إكثار مدفأة
        \item أدوات التطعيم والقطع
        \item أجهزة تطبيق هرمون التجذير
        \item غرف إنبات البذور
    \end{itemize}
\end{itemize}

\subsubsection{أنظمة التحكم في المناخ}
\begin{itemize}
    \item \textbf{إدارة درجة الحرارة:}
    \begin{itemize}
        \item أنظمة التدفئة (بمساعدة الطاقة الشمسية)
        \item أنظمة التبريد (التبريد بالتبخير)
        \item ستائر حرارية
        \item أجهزة استشعار ومتحكمات درجة الحرارة
        \item أنظمة احتياطية للطوارئ
    \end{itemize}
    
    \item \textbf{التحكم في الرطوبة:}
    \begin{itemize}
        \item أنظمة الرذاذ
        \item أجهزة إزالة الرطوبة
        \item أجهزة استشعار الرطوبة
        \item مراوح التهوية
        \item أدوات تحكم آلية للتهوية
    \end{itemize}
    
    \item \textbf{إدارة الإضاءة:}
    \begin{itemize}
        \item إضاءة LED تكميلية
        \item أنظمة التحكم في الظل
        \item أجهزة استشعار الضوء
        \item مؤقتات التحكم في الفترة الضوئية
        \item محللات طيف الضوء
    \end{itemize}
\end{itemize}

\subsubsection{أنظمة الري والتسميد}
\begin{itemize}
    \item \textbf{إدارة المياه:}
    \begin{itemize}
        \item خزانات تخزين المياه (50,000 لتر)
        \item أنظمة الترشيح
        \item معدات مراقبة جودة المياه
        \item أنظمة إعادة التدوير
        \item نظام حصاد مياه الأمطار
    \end{itemize}
    
    \item \textbf{معدات الري:}
    \begin{itemize}
        \item أنظمة الري بالتنقيط الآلية
        \item أنظمة الرذاذ للإكثار
        \item وحدات تحكم في الري
        \item أجهزة استشعار الرطوبة
        \item عدادات التدفق ومنظمات الضغط
    \end{itemize}
    
    \item \textbf{معدات التسميد:}
    \begin{itemize}
        \item حاقنات الأسمدة
        \item خزانات محلول المغذيات
        \item متحكمات الـ EC والـ pH
        \item مضخات الجرعات
        \item محللات المغذيات
    \end{itemize}
\end{itemize}

\subsubsection{معدات المختبر والاختبار}
\begin{itemize}
    \item \textbf{اختبار الجودة:}
    \begin{itemize}
        \item معدات اختبار صحة النبات
        \item أدوات التحقق الوراثي
        \item معدات اختبار التربة والوسائط
        \item محللات جودة المياه
        \item مجموعات الكشف عن مسببات الأمراض
    \end{itemize}
    
    \item \textbf{معدات البحث:}
    \begin{itemize}
        \item أجهزة جمع البيانات
        \item أنظمة مراقبة بيئية
        \item غرف نمو تجريبية
        \item معدات التوثيق
        \item أدوات معالجة العينات
    \end{itemize}
\end{itemize}

\subsection{متطلبات الموارد البشرية}

\subsubsection{الكادر الفني}
\begin{itemize}
    \item \textbf{الإدارة:}
    \begin{itemize}
        \item مدير المشتل (1): العمليات العامة والتنسيق
        \item المشرف الفني (1): الإشراف على الإنتاج ومراقبة الجودة
        \item منسق البحوث (1): أنشطة البحث والتطوير وإدارة المعرفة
    \end{itemize}
    
    \item \textbf{الفنيون المتخصصون:}
    \begin{itemize}
        \item متخصصو الإكثار (2): القطع والتطعيم وزراعة الأنسجة
        \item فني التحكم في المناخ (1): إدارة الأنظمة البيئية
        \item فني المختبر (2): زراعة الأنسجة والاختبار
        \item متخصص الري (1): إدارة المياه والمغذيات
    \end{itemize}
    
    \item \textbf{الكادر العام:}
    \begin{itemize}
        \item عمال المشتل (6-8): رعاية النباتات والزراعة والصيانة
        \item فني الصيانة (1): صيانة المعدات والمرافق
        \item مساعد إداري (1): حفظ السجلات والخدمات اللوجستية
    \end{itemize}
\end{itemize}

\subsubsection{متطلبات المهارات}
\begin{itemize}
    \item \textbf{المعرفة الفنية:}
    \begin{itemize}
        \item تقنيات إكثار النباتات
        \item زراعة الزيتون ونخيل التمر
        \item إدارة البيوت المحمية
        \item إدارة الآفات والأمراض
        \item أنظمة الري والتسميد
        \item إجراءات المختبر
    \end{itemize}
    
    \item \textbf{المهارات التشغيلية:}
    \begin{itemize}
        \item تشغيل نظام التحكم في المناخ
        \item صيانة المعدات
        \item إجراءات مراقبة الجودة
        \item جمع وتحليل البيانات
        \item إدارة الموارد
        \item الجدولة والتخطيط
    \end{itemize}
\end{itemize}

\subsubsection{متطلبات التدريب}
\begin{itemize}
    \item \textbf{التدريب الأولي:}
    \begin{itemize}
        \item تقنيات إكثار الزيتون ونخيل التمر
        \item إدارة البيوت المحمية وبيوت الظل
        \item إجراءات وبروتوكولات المختبر
        \item تشغيل وصيانة المعدات
        \item معايير مراقبة الجودة
    \end{itemize}
    
    \item \textbf{التطوير المستمر:}
    \begin{itemize}
        \item طرق الإكثار المتقدمة
        \item إدارة الأصناف الجديدة
        \item منهجيات البحث
        \item ممارسات المشتل المستدامة
        \item تحديثات وتطبيقات التكنولوجيا
    \end{itemize}
\end{itemize}

\subsection{متطلبات المواد والمستلزمات}

\subsubsection{وسائط النمو والمحسنات}
\begin{itemize}
    \item \textbf{المكونات الأساسية:}
    \begin{itemize}
        \item الخث أو ألياف جوز الهند: 50 متر مكعب/سنة
        \item البيرلايت: 30 متر مكعب/سنة
        \item الفيرميكيولايت: 20 متر مكعب/سنة
        \item الرمل (المغسول): 40 متر مكعب/سنة
        \item السماد العضوي: 60 متر مكعب/سنة (بشكل أساسي من وحدات المشروع)
    \end{itemize}
    
    \item \textbf{المحسنات:}
    \begin{itemize}
        \item الفحم الحيوي: 20 متر مكعب/سنة (من وحدة الانحلال الحراري في المشروع)
        \item سماد الديدان: 15 متر مكعب/سنة (من وحدات المشروع)
        \item الجير: 2 طن/سنة
        \item لقاحات فطريات الميكورايزا: 500 كجم/سنة
        \item منتجات البكتيريا المفيدة: 200 كجم/سنة
    \end{itemize}
\end{itemize}

\subsubsection{الحاويات ومستلزمات الإكثار}
\begin{itemize}
    \item \textbf{الحاويات:}
    \begin{itemize}
        \item صواني الإكثار: 5,000 وحدة
        \item أصص صغيرة (1-2 لتر): 10,000 وحدة
        \item أصص متوسطة (5-10 لتر): 5,000 وحدة
        \item أصص كبيرة (15-25 لتر): 3,000 وحدة
        \item حاويات متخصصة لتدريب الجذور: 2,000 وحدة
    \end{itemize}
    
    \item \textbf{مواد الإكثار:}
    \begin{itemize}
        \item هرمونات التجذير: 50 كجم/سنة
        \item مستلزمات التطعيم: 5,000 وحدة/سنة
        \item شريط وشمع التطعيم: 100 كجم/سنة
        \item ملصقات الإكثار: 20,000 وحدة/سنة
        \item مكونات وسط زراعة الأنسجة: حسب الحاجة
    \end{itemize}
\end{itemize}

\subsubsection{المغذيات وحماية النبات}
\begin{itemize}
    \item \textbf{الأسمدة:}
    \begin{itemize}
        \item أسمدة بطيئة الإطلاق: 2 طن/سنة
        \item أسمدة قابلة للذوبان في الماء: 1 طن/سنة
        \item مكملات المغذيات الدقيقة: 500 كجم/سنة
        \item أسمدة عضوية: 5 طن/سنة
        \item مغذيات متخصصة للإكثار: 200 كجم/سنة
    \end{itemize}
    
    \item \textbf{حماية النبات:}
    \begin{itemize}
        \item عوامل المكافحة البيولوجية: حسب الحاجة
        \item مبيدات فطرية عضوية: 200 كجم/سنة
        \item صابون مبيد للحشرات: 300 لتر/سنة
        \item مصائد لاصقة: 5,000 وحدة/سنة
        \item حشرات نافعة: حسب الحاجة
    \end{itemize}
\end{itemize}

\subsection{متطلبات المرافق}

\subsubsection{متطلبات المياه}
\begin{itemize}
    \item \textbf{الكمية:}
    \begin{itemize}
        \item إجمالي الاحتياج السنوي: 15,000-20,000 متر مكعب
        \item ذروة الطلب اليومي: 80-100 متر مكعب
        \item قدرة إعادة التدوير: 40-50\% من الإجمالي
    \end{itemize}
    
    \item \textbf{معايير الجودة:}
    \begin{itemize}
        \item التوصيل الكهربائي: < 1.0 مللي سيمنز/سم
        \item الرقم الهيدروجيني: 6.0-7.0
        \item الصوديوم: < 50 جزء في المليون
        \item الكلوريد: < 100 جزء في المليون
        \item خالية من مسببات الأمراض
    \end{itemize}
\end{itemize}

\subsubsection{متطلبات الطاقة}
\begin{itemize}
    \item \textbf{الكهرباء:}
    \begin{itemize}
        \item الحمل المتصل: 100-120 كيلوواط
        \item الاستهلاك السنوي: 180,000-220,000 كيلوواط ساعة
        \item قدرة التوليد الشمسي: 150 كيلوواط (هدف)
        \item تخزين البطارية: 300 كيلوواط ساعة
    \end{itemize}
    
    \item \textbf{التدفئة (إذا لزم الأمر):}
    \begin{itemize}
        \item قدرة التدفئة: 500 كيلوواط
        \item الاستهلاك السنوي: يعتمد على المناخ
        \item مساهمة الطاقة الشمسية الحرارية: 60\% (هدف)
    \end{itemize}
\end{itemize}

\subsection{الحصول على الموارد المرحلي}

\subsubsection{المرحلة 1 (2026-2027)}
\begin{itemize}
    \item تجهيز الأرض والبنية التحتية الأساسية
    \item البيت المحمي الأولي (800 متر مربع) وبيت الظل (400 متر مربع)
    \item التجهيز الأساسي للمختبر
    \item معدات الإكثار الأساسية
    \item أنظمة الري الأساسية
    \item توظيف الكادر الفني الرئيسي (6-8 أشخاص)
    \item مخزون أولي من المواد والمستلزمات
\end{itemize}

\subsubsection{المرحلة 2 (2027-2028)}
\begin{itemize}
    \item بيت محمي إضافي (600 متر مربع) وبيت ظل (300 متر مربع)
    \item توسعة المختبر
    \item أنظمة متقدمة للتحكم في المناخ
    \item تعزيز الري والتسميد
    \item كادر فني إضافي (4-5 أشخاص)
    \item توسيع سلسلة توريد المواد
\end{itemize}

\subsubsection{المرحلة 3 (2028-2029)}
\begin{itemize}
    \item البيت المحمي النهائي (600 متر مربع) وبيت الظل (300 متر مربع)
    \item معدات بحثية متخصصة
    \item أنظمة أتمتة متقدمة
    \item اكتمال الكادر الوظيفي (16-18 شخص إجمالاً)
    \item مخزون كامل من المواد والمستلزمات
\end{itemize}

توضح خطة متطلبات الموارد هذه الأرض والبنية التحتية والمعدات والموارد البشرية والمواد والمرافق اللازمة للإنشاء الناجح وتشغيل وحدة المشتل ضمن مشروع الاقتصاد الدائري في الطور. يتوافق نهج الحصول على الموارد المرحلي مع الجدول الزمني العام لتنفيذ المشروع والخطة المالية. 