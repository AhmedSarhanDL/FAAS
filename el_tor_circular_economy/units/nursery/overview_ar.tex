\section{نظرة عامة على وحدة المشتل}

\subsection{الغرض والنطاق}
تعمل وحدة المشتل كمنشأة مركزية لإكثار وتنمية وتوريد الشتلات والنباتات الصغيرة عالية الجودة بشكل رئيسي لوحدتي زراعة الزيتون ونخيل التمر ضمن مشروع الاقتصاد الدائري في الطور. تضمن الوحدة الجودة الوراثية، ومقاومة الأمراض، وظروف النمو المثلى للنباتات الصغيرة قبل نقلها إلى مواقع الزراعة الدائمة.

\subsection{الوظائف الرئيسية}
\begin{itemize}
    \item \textbf{إكثار النباتات:}
    \begin{itemize}
        \item إكثار أشجار الزيتون من خلال العقل والتطعيم
        \item إكثار نخيل التمر من خلال الفسائل وزراعة الأنسجة
        \item إدارة النباتات الأم للحفاظ على المادة الوراثية
        \item تنفيذ تقنيات الإكثار المتقدمة
    \end{itemize}
    
    \item \textbf{رعاية الشتلات:}
    \begin{itemize}
        \item بيئات نمو متحكم في مناخها
        \item أنظمة ري وتسميد دقيقة
        \item مراقبة ومنع الأمراض
        \item إدارة مراحل النمو
    \end{itemize}
    
    \item \textbf{البحث والتطوير:}
    \begin{itemize}
        \item تجارب واختيار الأصناف
        \item تحسين تقنيات الإكثار
        \item أبحاث التكيف مع المناخ
        \item دراسات مقاومة الأمراض
    \end{itemize}
\end{itemize}

\subsection{القدرة والبنية التحتية}
\begin{itemize}
    \item \textbf{القدرة الإنتاجية:}
    \begin{itemize}
        \item إنتاج سنوي 2,000 شتلة زيتون
        \item إنتاج سنوي 1,000 فسيلة نخيل
        \item قدرة توسع تصل إلى 5,000 نبات سنوياً
    \end{itemize}
    
    \item \textbf{المرافق:}
    \begin{itemize}
        \item 2,000 متر مربع مساحة بيوت محمية
        \item 1,000 متر مربع منطقة مظللة
        \item 500 متر مربع قسم النباتات الأم
        \item مختبر ومرفق زراعة الأنسجة
        \item مناطق التخزين والمعالجة
    \end{itemize}
\end{itemize}

\subsection{التكامل مع الوحدات الأخرى}
\begin{itemize}
    \item \textbf{تكامل المدخلات:}
    \begin{itemize}
        \item السماد الدودي والفحم الحيوي لوسائط النمو
        \item المياه المعالجة من وحدة إدارة المياه
        \item مواد المكافحة العضوية للآفات
        \item الطاقة المتجددة للتحكم في المناخ
    \end{itemize}
    
    \item \textbf{تكامل المخرجات:}
    \begin{itemize}
        \item شتلات عالية الجودة لوحدات الزراعة
        \item بيانات البحوث وأفضل الممارسات
        \item حفظ المادة الوراثية
        \item التدريب وبناء القدرات
    \end{itemize}
\end{itemize}

\subsection{مقاييس النجاح الرئيسية}
\begin{itemize}
    \item معدل بقاء الشتلات (الهدف: >90\%)
    \item الحفاظ على النقاء الوراثي (100\%)
    \item مستويات مقاومة الأمراض
    \item معدل النمو والتجانس
    \item كفاءة استخدام الموارد
    \item مخرجات البحث والابتكار
\end{itemize}

تلعب وحدة المشتل دوراً حاسماً في ضمان نجاح وحدتي زراعة الزيتون ونخيل التمر من خلال توفير مواد زراعية عالية الجودة ومتكيفة جيداً، مع المساهمة في أهداف البحث والتطوير للمشروع. 