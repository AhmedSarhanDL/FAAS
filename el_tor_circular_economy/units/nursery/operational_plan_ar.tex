\section{الخطة التشغيلية لوحدة المشتل}

\subsection{عمليات الإنتاج}
\begin{itemize}
    \item \textbf{إكثار أشجار الزيتون:}
    \begin{itemize}
        \item \textbf{الطرق:}
        \begin{itemize}
            \item إكثار العقل شبه الخشبية
            \item التطعيم على الأصول
            \item زراعة الأنسجة للأصناف المختارة
        \end{itemize}
        \item \textbf{الجدول الزمني:}
        \begin{itemize}
            \item جمع العقل: فبراير-مارس
            \item فترة التجذير: 60-90 يوم
            \item التقسية: 30-45 يوم
            \item دورة الإنتاج الكاملة: 6-8 أشهر
        \end{itemize}
    \end{itemize}
    
    \item \textbf{إكثار نخيل التمر:}
    \begin{itemize}
        \item \textbf{الطرق:}
        \begin{itemize}
            \item فصل وزراعة الفسائل
            \item التكاثر بزراعة الأنسجة
            \item البذر المباشر للتربية
        \end{itemize}
        \item \textbf{الجدول الزمني:}
        \begin{itemize}
            \item فصل الفسائل: مارس-أبريل
            \item التأسيس الأولي: 3-4 أشهر
            \item دورة زراعة الأنسجة: 18-24 شهر
        \end{itemize}
    \end{itemize}
\end{itemize}

\subsection{إدارة المرافق}
\begin{itemize}
    \item \textbf{عمليات البيوت المحمية:}
    \begin{itemize}
        \item التحكم في درجة الحرارة (18-28 درجة مئوية)
        \item إدارة الرطوبة (60-80\%)
        \item تنظيم شدة الإضاءة
        \item تشغيل نظام التهوية
        \item المراقبة والتعديل اليومي
    \end{itemize}
    
    \item \textbf{أنظمة الري:}
    \begin{itemize}
        \item الرش الآلي للعقل
        \item الري بالتنقيط للنباتات المؤسسة
        \item جدولة التسميد بالري
        \item مراقبة جودة المياه
        \item صيانة النظام
    \end{itemize}
    
    \item \textbf{إدارة وسائط النمو:}
    \begin{itemize}
        \item تحضير وتعقيم الركائز
        \item دمج السماد الدودي
        \item إضافة الفحم الحيوي
        \item مراقبة الحموضة والتوصيل الكهربائي
        \item التخزين والمناولة
    \end{itemize}
\end{itemize}

\subsection{إدارة صحة النبات}
\begin{itemize}
    \item \textbf{الوقاية من الأمراض:}
    \begin{itemize}
        \item الفحص المنتظم للنباتات
        \item بروتوكولات النظافة
        \item المعالجات الوقائية
        \item إجراءات الحجر الصحي
        \item نظام مراقبة الأمراض
    \end{itemize}
    
    \item \textbf{إدارة الآفات:}
    \begin{itemize}
        \item الإدارة المتكاملة للآفات
        \item عوامل المكافحة البيولوجية
        \item الحواجز المادية
        \item المراقبة والاستكشاف
        \item بروتوكولات المعالجة
    \end{itemize}
    
    \item \textbf{مراقبة الجودة:}
    \begin{itemize}
        \item تقييم مراحل النمو
        \item التحقق من النقاء الوراثي
        \item شهادة الصحة
        \item اختبار الأداء
        \item نظام التوثيق
    \end{itemize}
\end{itemize}

\subsection{إدارة الموارد}
\begin{itemize}
    \item \textbf{متطلبات المواد:}
    \begin{itemize}
        \item مكونات وسائط النمو
        \item مستلزمات الإكثار
        \item مواد وقاية النبات
        \item الأسمدة والمحسنات
        \item مستلزمات المختبر
    \end{itemize}
    
    \item \textbf{تنظيم العمالة:}
    \begin{itemize}
        \item فنيون مهرة (4-6)
        \item عمال عامون (8-10)
        \item طاقم البحث (2-3)
        \item برامج التدريب
        \item جدولة العمل
    \end{itemize}
    
    \item \textbf{صيانة المعدات:}
    \begin{itemize}
        \item جدول الصيانة الوقائية
        \item معايرة المعدات
        \item بروتوكولات الإصلاح
        \item مخزون قطع الغيار
        \item توثيق الخدمة
    \end{itemize}
\end{itemize}

\subsection{أنشطة البحث والتطوير}
\begin{itemize}
    \item \textbf{تجارب الأصناف:}
    \begin{itemize}
        \item تقييم الأداء
        \item تقييم التكيف
        \item معايير الاختيار
        \item جمع البيانات
        \item تحليل النتائج
    \end{itemize}
    
    \item \textbf{بحوث الإكثار:}
    \begin{itemize}
        \item تحسين الطرق
        \item تطوير البروتوكولات
        \item تحسين معدل النجاح
        \item دراسات خفض التكلفة
        \item نقل التكنولوجيا
    \end{itemize}
\end{itemize}

\subsection{ضمان الجودة}
\begin{itemize}
    \item \textbf{المعايير والبروتوكولات:}
    \begin{itemize}
        \item إجراءات الإنتاج
        \item شهادة الصحة
        \item التوثيق الوراثي
        \item أنظمة التوثيق
        \item تدابير التتبع
    \end{itemize}
    
    \item \textbf{المراقبة والتقييم:}
    \begin{itemize}
        \item معايير النمو
        \item معدلات النجاح
        \item مقاييس الجودة
        \item كفاءة الموارد
        \item تغذية راجعة من العملاء
    \end{itemize}
\end{itemize}

توفر هذه الخطة التشغيلية إطاراً شاملاً لإدارة الأنشطة اليومية لوحدة المشتل، مما يضمن الإنتاج الفعال لمواد الزراعة عالية الجودة لوحدتي زراعة الزيتون ونخيل التمر. 