\section{متطلبات الموارد لزراعة الأكاسيا}

\subsection{متطلبات الأرض}

\begin{itemize}
    \item \textbf{المساحة الإجمالية:} 45 فدان (18.9 هكتار) مخصصة لزراعة الأكاسيا
    \item \textbf{خصائص الأرض:}
    \begin{itemize}
        \item نوع التربة: يفضل التربة الرملية إلى الرملية الطينية
        \item الطبوغرافيا: أرض منحدرة بلطف (1-3\% انحدار) للصرف الطبيعي
        \item الاتجاه: حماية من الرياح السائدة القوية خلال مرحلة التأسيس
    \end{itemize}
    \item \textbf{التكوين المكاني:}
    \begin{itemize}
        \item شرائط مصدات الرياح: بعرض 10-15 متر
        \item حدود الحقول: زراعات بعرض 5-8 متر
        \item كتل المزارع: كتل بمساحة 1-2 فدان لإنتاج العلف
        \item المناطق العازلة: 10 متر كحد أدنى بين المزارع والمحاصيل الحساسة
    \end{itemize}
    \item \textbf{تحضير الأرض:}
    \begin{itemize}
        \item تمزيق عميق: عمق 60-80 سم على طول خطوط الزراعة
        \item سدود كنتورية: ارتفاع 30 سم لحصاد المياه
        \item طرق الوصول: بعرض 3-4 متر للصيانة والحصاد
    \end{itemize}
\end{itemize}

\subsection{موارد المياه}

\begin{itemize}
    \item \textbf{متطلبات الري:}
    \begin{itemize}
        \item مرحلة التأسيس (السنة الأولى): 2,500-3,000 متر مكعب/فدان/سنة
        \item الأشجار الصغيرة (السنوات 2-3): 1,500-2,000 متر مكعب/فدان/سنة
        \item الأشجار الناضجة (السنة 4+): 800-1,200 متر مكعب/فدان/سنة
    \end{itemize}
    \item \textbf{مصادر المياه:}
    \begin{itemize}
        \item الأساسية: المياه الرمادية المعالجة من الوحدات السكنية ووحدات المعالجة
        \item الثانوية: مياه الجريان الزراعية ومياه الصرف
        \item التكميلية: المياه الجوفية لمرحلة التأسيس فقط
    \end{itemize}
    \item \textbf{معايير جودة المياه:}
    \begin{itemize}
        \item تحمل الملوحة: حتى 4-6 ديسيسيمنز/متر للأشجار المستقرة
        \item نطاق الرقم الهيدروجيني: 6.0-8.5
        \item الحد الأقصى للأكسجين الحيوي المستهلك: 100 ملغ/لتر للمياه العادمة المعالجة
    \end{itemize}
    \item \textbf{أنظمة الحفاظ على المياه:}
    \begin{itemize}
        \item أحواض تجميع صغيرة: قطر 2 متر حول كل شجرة
        \item التغطية: طبقة 10 سم من المهاد العضوي
        \item مراقبة رطوبة التربة: أجهزة استشعار على أعماق 30 سم، 60 سم، و90 سم
    \end{itemize}
\end{itemize}

\subsection{مواد الزراعة}

\begin{itemize}
    \item \textbf{متطلبات البذور:}
    \begin{itemize}
        \item أكاسيا ساليجنا: 3-4 كغم/فدان (بذر مباشر)
        \item أكاسيا تورتيليس: 2-3 كغم/فدان (بذر مباشر)
        \item معدل الإنبات: 70\% كحد أدنى بعد التخديش
    \end{itemize}
    \item \textbf{متطلبات الشتلات:}
    \begin{itemize}
        \item إجمالي الشتلات: 22,500 على مدى 5 سنوات (500 شجرة/فدان × 45 فدان)
        \item المتطلبات السنوية:
        \begin{itemize}
            \item السنة 1: 1,000 شتلة
            \item السنة 2: 3,000 شتلة
            \item السنة 3: 5,000 شتلة
            \item السنة 4: 4,500 شتلة
            \item السنة 5: 9,000 شتلة
        \end{itemize}
        \item مواصفات الشتلات: ارتفاع 30-40 سم، عمر 4-6 أشهر، مقساة
    \end{itemize}
    \item \textbf{المادة الوراثية:}
    \begin{itemize}
        \item اختيار المصدر: سلالات متكيفة محليًا مع تحمل الجفاف
        \item التنوع الجيني: 5 مصادر بذور متميزة كحد أدنى
        \item معايير الجودة: خالية من الأمراض معتمدة، عادة نمو قوية
    \end{itemize}
\end{itemize}

\subsection{المعدات والبنية التحتية}

\begin{itemize}
    \item \textbf{بنية تحتية للري:}
    \begin{itemize}
        \item نظام الري بالتنقيط: تغطية 45 فدان
        \item تخزين المياه: خزانات بسعة 500 متر مكعب
        \item نظام الترشيح: مرشحات رملية، قرصية، وشبكية
        \item قدرة الضخ: 30 متر مكعب/ساعة
        \item نظام التحكم: آلي مع أجهزة استشعار رطوبة التربة
    \end{itemize}
    \item \textbf{معدات الزراعة:}
    \begin{itemize}
        \item جرار مع ملحق تمزيق
        \item حفار لثقوب الزراعة
        \item آلات بذر للبذر المباشر
        \item أدوات ومعدات زراعة الأشجار
    \end{itemize}
    \item \textbf{معدات الصيانة:}
    \begin{itemize}
        \item أدوات التقليم: مقصات تقليم، مناشير تقليم، مقلمات على عصا
        \item معدات التغطية: آلة تقطيع/تمزيق
        \item معدات إدارة الآفات: رشاشات ظهرية
        \item معدات المراقبة: مجموعات اختبار التربة، مقاييس الرطوبة
    \end{itemize}
    \item \textbf{معدات الحصاد:}
    \begin{itemize}
        \item أدوات حصاد العلف: معدات تقليم متخصصة
        \item معدات المعالجة: قطاعة/مطحنة للعلف
        \item رفوف التجفيف: سعة 200 متر مربع
        \item مرافق التخزين: 100 متر مربع تخزين مقاوم للعوامل الجوية
    \end{itemize}
\end{itemize}

\subsection{الموارد البشرية}

\begin{itemize}
    \item \textbf{الطاقم الفني:}
    \begin{itemize}
        \item متخصص الزراعة الحرجية: 1 بدوام كامل
        \item فنيون ميدانيون: 3-4 بدوام كامل
        \item منسق التكامل: 1 بدوام جزئي (مشترك مع وحدات أخرى)
    \end{itemize}
    \item \textbf{متطلبات العمالة:}
    \begin{itemize}
        \item مرحلة التأسيس: 5-6 عمال لكل فدان للزراعة
        \item عمليات الصيانة: 2-3 عمال لكل فدان سنويًا
        \item عمليات الحصاد: 4-5 عمال لكل فدان خلال فترات الحصاد
    \end{itemize}
    \item \textbf{المهارات المتخصصة:}
    \begin{itemize}
        \item تقليم وإدارة الأشجار
        \item تشغيل وصيانة نظام الري
        \item تحديد وإدارة الآفات والأمراض
        \item مراقبة جودة التربة والمياه
    \end{itemize}
    \item \textbf{متطلبات التدريب:}
    \begin{itemize}
        \item التدريب الأولي: برنامج شامل لمدة أسبوعين
        \item التدريب المستمر: دورات تنشيطية ربع سنوية
        \item تدريب السلامة: شهادة نصف سنوية
    \end{itemize}
\end{itemize}

\subsection{المدخلات والمستلزمات}

\begin{itemize}
    \item \textbf{تعديلات التربة:}
    \begin{itemize}
        \item السماد العضوي: 5-10 أطنان/فدان في البداية، 2-3 أطنان/فدان سنويًا
        \item لقاحات الفطريات الجذرية: 5 كغم/فدان عند الزراعة
        \item لقاحات الريزوبيوم: 2 كغم/فدان عند الزراعة
    \end{itemize}
    \item \textbf{مستلزمات إدارة الآفات:}
    \begin{itemize}
        \item المكافحة البيولوجية: حشرات مفترسة، عوامل ميكروبية
        \item مبيدات عضوية: زيت النيم، البيرثرم حسب الحاجة
        \item مستلزمات المراقبة: مصائد، طعوم، مجموعات تشخيص
    \end{itemize}
    \item \textbf{مستلزمات الري:}
    \begin{itemize}
        \item خطوط التنقيط: 45,000 متر (استبدال كل 5 سنوات)
        \item النقاطات: 22,500 وحدة (استبدال حسب الحاجة)
        \item المرشحات: عناصر استبدال سنويًا
        \item مواد كيميائية لمعالجة المياه: حسب المطلوب لإدارة جودة المياه
    \end{itemize}
    \item \textbf{مستلزمات التشغيل:}
    \begin{itemize}
        \item الوقود: 1,500-2,000 لتر سنويًا للمعدات
        \item مواد التشحيم والصيانة
        \item قطع غيار للمعدات
        \item مواد التعبئة للعلف
    \end{itemize}
\end{itemize}

\subsection{الموارد المالية}

\begin{itemize}
    \item \textbf{الاستثمار الرأسمالي:}
    \begin{itemize}
        \item تحضير الأرض: 10,000-15,000 جنيه مصري لكل فدان
        \item نظام الري: 20,000-25,000 جنيه مصري لكل فدان
        \item المعدات: 500,000-750,000 جنيه مصري إجمالي
        \item الزراعة الأولية: 5,000-7,000 جنيه مصري لكل فدان
    \end{itemize}
    \item \textbf{ميزانية التشغيل:}
    \begin{itemize}
        \item العمالة السنوية: 5,000-7,000 جنيه مصري لكل فدان
        \item المدخلات والمستلزمات: 3,000-4,000 جنيه مصري لكل فدان سنويًا
        \item تكاليف المياه: 2,000-3,000 جنيه مصري لكل فدان سنويًا
        \item الصيانة والإصلاحات: 2,000-2,500 جنيه مصري لكل فدان سنويًا
    \end{itemize}
    \item \textbf{صندوق الطوارئ:}
    \begin{itemize}
        \item 15\% من ميزانية التشغيل السنوية للنفقات غير المتوقعة
        \item احتياطي طوارئ للجفاف أو تفشي الآفات
    \end{itemize}
\end{itemize}

\subsection{موارد التكامل}

\begin{itemize}
    \item \textbf{تكامل إدارة المياه:}
    \begin{itemize}
        \item قدرة معالجة المياه: 100-150 متر مكعب/يوم من وحدة إدارة المياه
        \item مراقبة جودة المياه: مرافق اختبار مشتركة
        \item إدارة الصرف: منسقة مع الوحدات الزراعية الأخرى
    \end{itemize}
    \item \textbf{تكامل وحدة الثروة الحيوانية:}
    \begin{itemize}
        \item مرافق تخزين ومناولة الأعلاف: بنية تحتية مشتركة
        \item معدات اختبار التغذية: الوصول إلى مختبر وحدة الثروة الحيوانية
        \item السماد لتعديلات التربة: 20-30 طن سنويًا من وحدة الثروة الحيوانية
    \end{itemize}
    \item \textbf{تكامل إنتاج المحاصيل:}
    \begin{itemize}
        \item معدات مشتركة لتحضير وصيانة التربة
        \item أنظمة إدارة آفات منسقة
        \item شبكة مراقبة المناخ المحلي
    \end{itemize}
\end{itemize}
