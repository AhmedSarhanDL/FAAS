\section{نظرة عامة على زراعة الأكاسيا}

\subsection{مقدمة}
تعد وحدة زراعة الأكاسيا عنصراً أساسياً في مشروع الاقتصاد الدائري في الطور، حيث تخدم أغراضاً متعددة تشمل مصدات الرياح، وإنتاج الأعلاف، وتثبيت النيتروجين. تم تصميم هذه الوحدة للتكامل مع الوحدات الزراعية الأخرى مع توفير خدمات بيئية أساسية.

\subsection{المكونات الأساسية}
\begin{itemize}
    \item \textbf{تخصيص الأراضي:} 45 فداناً مخصصة لزراعة الأكاسيا
    \item \textbf{الجدول الزمني للتنفيذ:}
    \begin{itemize}
        \item السنة الأولى: زراعة 2 فدان مبدئياً
        \item السنة الثانية: إضافة 4 أفدنة
        \item السنة الثالثة: التوسع إلى 10 أفدنة
        \item السنة الرابعة: النمو إلى 25 فداناً
        \item السنة الخامسة: التوسع النهائي إلى 45 فداناً
    \end{itemize}
\end{itemize}

\subsection{الوظائف الأساسية}
\begin{itemize}
    \item \textbf{الخدمات البيئية:}
    \begin{itemize}
        \item حماية المحاصيل الأخرى من الرياح
        \item تثبيت النيتروجين في التربة
        \item مكافحة التعرية
        \item تعزيز التنوع البيولوجي
    \end{itemize}
    
    \item \textbf{التكامل الزراعي:}
    \begin{itemize}
        \item إنتاج الأعلاف المستدامة
        \item دعم وحدة الثروة الحيوانية
        \item تحسين التربة
        \item تنظيم المناخ المحلي
    \end{itemize}
\end{itemize}

\subsection{الممارسات المستدامة}
\begin{itemize}
    \item \textbf{إدارة الموارد:}
    \begin{itemize}
        \item الاستخدام الفعال للمياه
        \item التسميد الطبيعي من خلال تثبيت النيتروجين
        \item الإدارة المتكاملة للآفات
        \item ممارسات التقليم المستدامة
    \end{itemize}
    
    \item \textbf{الفوائد البيئية:}
    \begin{itemize}
        \item احتجاز الكربون
        \item تحسين بنية التربة
        \item خلق موائل طبيعية
        \item تعزيز النظام البيئي
    \end{itemize}
\end{itemize}

\subsection{التكامل الاقتصادي}
\begin{itemize}
    \item \textbf{المنتجات والخدمات:}
    \begin{itemize}
        \item إنتاج الأعلاف الحيوانية
        \item منتجات خشبية (محدودة)
        \item خدمات بيئية
        \item تحسين التربة
    \end{itemize}
    
    \item \textbf{فوائد الاقتصاد الدائري:}
    \begin{itemize}
        \item دعم عمليات الثروة الحيوانية
        \item التسميد الطبيعي للمحاصيل الأخرى
        \item تدوير الموارد المستدام
        \item تعزيز مرونة المزرعة
    \end{itemize}
\end{itemize}
