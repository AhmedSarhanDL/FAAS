\section{\RL{نظرة عامة على وحدة زراعة الزيتون}}

\subsection{\RL{وصف الوحدة}}
\RL{وحدة زراعة الزيتون هي مكون بمساحة 45 فدان (18.9 هكتار) من مشروع الاقتصاد الدائري في الطور، مصممة لإنتاج زيت زيتون عالي الجودة مع التكامل مع وحدات الإنتاج الأخرى في نظام موارد دائري. سيتم تطوير الوحدة على خمس مراحل من 2026 إلى 2031، لتستضيف في النهاية 4,500 شجرة زيتون مقاومة للجفاف مناسبة لإنتاج الزيت. يستخدم نظام الزراعة ممارسات مستدامة تشمل الري بالتنقيط، وتطبيق الفحم الحيوي، واستخدام السماد الدودي، والتكامل مع وحدات الثروة الحيوانية والدواجن.}

\subsection{\RL{الأهمية الاستراتيجية}}
\begin{itemize}
    \item \textbf{\RL{القيمة الاقتصادية:}} \RL{إنتاج زيت زيتون ممتاز للأسواق المحلية والتصدير، مما يخلق تدفقًا للمنتجات عالية القيمة مع طلب قوي في السوق.}
    
    \item \textbf{\RL{كفاءة الموارد:}} \RL{تنفيذ طرق زراعة موفرة للمياه في بيئة قاحلة، مما يوضح الزراعة المستدامة في المناطق التي تعاني من ندرة المياه.}
    
    \item \textbf{\RL{التكامل الدائري:}} \RL{تعمل كعقدة رئيسية في الاقتصاد الدائري للمشروع، حيث تتلقى المدخلات من وتوفر المخرجات إلى وحدات الإنتاج الأخرى.}
    
    \item \textbf{\RL{احتجاز الكربون:}} \RL{تعمل أشجار الزيتون كمصارف كربون طويلة الأمد، مما يساهم في أهداف تخفيف تغير المناخ للمشروع.}
    
    \item \textbf{\RL{تعزيز التنوع البيولوجي:}} \RL{تزيد نهج الزراعة البينية والزراعة الحراجية من التنوع البيولوجي ومرونة النظام البيئي.}
\end{itemize}

\subsection{\RL{أهداف الإنتاج الرئيسية}}
\begin{itemize}
    \item \textbf{\RL{إنتاج زيت الزيتون:}}
    \begin{itemize}
        \item \RL{السنة 3: 5,000 لتر}
        \item \RL{السنة 4: 15,000 لتر}
        \item \RL{السنة 5: 30,000 لتر}
        \item \RL{النضج الكامل (السنة 10+): 67,500 لتر سنويًا}
    \end{itemize}
    
    \item \textbf{\RL{منتجات الزراعة البينية:}}
    \begin{itemize}
        \item \RL{أعشاب طبية: 2-5 أطنان سنويًا}
        \item \RL{بقوليات: 3-7 أطنان سنويًا}
        \item \RL{محاصيل علفية: 10-15 طن سنويًا}
    \end{itemize}
    
    \item \textbf{\RL{خدمات النظام البيئي:}}
    \begin{itemize}
        \item \RL{احتجاز الكربون: 450-900 طن مكافئ ثاني أكسيد الكربون سنويًا}
        \item \RL{تعزيز التنوع البيولوجي: زيادة بنسبة 30-50\% في تنوع الأنواع}
        \item \RL{تحسين صحة التربة: زيادة سنوية بنسبة 2-3\% في المادة العضوية للتربة}
    \end{itemize}
\end{itemize}

\subsection{\RL{التكامل مع الوحدات الأخرى}}
\begin{itemize}
    \item \textbf{\RL{وحدة الآزولا:}}
    \begin{itemize}
        \item \RL{تتلقى: مياه غنية بالمغذيات وسماد قائم على الآزولا}
        \item \RL{توفر: مياه الري العائدة}
    \end{itemize}
    
    \item \textbf{\RL{وحدة الثروة الحيوانية:}}
    \begin{itemize}
        \item \RL{تتلقى: حيوانات الرعي للسيطرة على الأعشاب الضارة والتسميد}
        \item \RL{توفر: تفل الزيتون كمكمل علف، محاصيل علفية}
    \end{itemize}
    
    \item \textbf{\RL{وحدة إنتاج الفحم الحيوي:}}
    \begin{itemize}
        \item \RL{تتلقى: الفحم الحيوي لتحسين التربة}
        \item \RL{توفر: مخلفات التقليم وبقايا المعالجة}
    \end{itemize}
    
    \item \textbf{\RL{وحدة التسميد الدودي:}}
    \begin{itemize}
        \item \RL{تتلقى: السماد الدودي لتحسين التربة}
        \item \RL{توفر: النفايات العضوية من المعالجة والزراعة}
    \end{itemize}
    
    \item \textbf{\RL{نظام إدارة المياه:}}
    \begin{itemize}
        \item \RL{تتلقى: مياه الري المعالجة}
        \item \RL{توفر: المياه العائدة للمعالجة وإعادة التدوير}
    \end{itemize}
\end{itemize}

\subsection{\RL{الأثر الاقتصادي}}
\begin{itemize}
    \item \textbf{\RL{مصادر الإيرادات:}}
    \begin{itemize}
        \item \RL{الأساسية: مبيعات زيت الزيتون الممتاز}
        \item \RL{الثانوية: منتجات الزراعة البينية}
        \item \RL{الثالثة: ائتمانات الكربون وخدمات النظام البيئي}
    \end{itemize}
    
    \item \textbf{\RL{توليد فرص العمل:}}
    \begin{itemize}
        \item \RL{وظائف دائمة: 8-12 وظيفة}
        \item \RL{عمالة موسمية: 15-35 وظيفة خلال موسم الحصاد والمعالجة}
        \item \RL{توظيف غير مباشر: 20-30 وظيفة في الخدمات ذات الصلة}
    \end{itemize}
    
    \item \textbf{\RL{التوقعات المالية:}}
    \begin{itemize}
        \item \RL{الاستثمار الأولي: 717,500 دولار}
        \item \RL{تكاليف التشغيل السنوية: 150,000-300,000 دولار}
        \item \RL{الإيرادات السنوية عند الإنتاج الكامل: 500,000-750,000 دولار}
        \item \RL{العائد المتوقع على الاستثمار: 15-20\% بعد النضج الكامل}
        \item \RL{فترة الاسترداد: 7-9 سنوات}
    \end{itemize}
\end{itemize}

\subsection{\RL{الاستدامة البيئية}}
\begin{itemize}
    \item \textbf{\RL{الحفاظ على المياه:}}
    \begin{itemize}
        \item \RL{كفاءة الري بنسبة 85\% من خلال أنظمة التنقيط}
        \item \RL{تخفيض استخدام المياه بنسبة 30-40\% مقارنة بالطرق التقليدية}
        \item \RL{تكامل إعادة تدوير المياه ومعالجتها}
    \end{itemize}
    
    \item \textbf{\RL{صحة التربة:}}
    \begin{itemize}
        \item \RL{تطبيق الفحم الحيوي لاحتجاز الكربون}
        \item \RL{السماد الدودي لتعزيز المادة العضوية}
        \item \RL{ممارسات الحراثة الدنيا}
        \item \RL{زراعة الغطاء والتغطية}
    \end{itemize}
    
    \item \textbf{\RL{التنوع البيولوجي:}}
    \begin{itemize}
        \item \RL{نظام زراعة بينية متنوع}
        \item \RL{إنشاء موائل للحشرات المفيدة}
        \item \RL{مدخلات كيميائية بالحد الأدنى}
        \item \RL{إدارة متكاملة للآفات}
    \end{itemize}
\end{itemize}

\RL{تمثل وحدة زراعة الزيتون هذه مكونًا رئيسيًا في مشروع الاقتصاد الدائري في الطور، مما يوضح كيف يمكن دمج المحاصيل المتوسطية التقليدية في أنظمة زراعية دائرية حديثة مع توفير فوائد اقتصادية وبيئية واجتماعية.}
