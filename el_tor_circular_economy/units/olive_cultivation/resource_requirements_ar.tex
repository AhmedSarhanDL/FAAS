\section{\RL{متطلبات الموارد لزراعة الزيتون}}

\subsection{\RL{متطلبات الأرض}}
\begin{itemize}
    \item \textbf{\RL{المساحة الإجمالية:}} \RL{45 فدان (18.9 هكتار)}
    \item \textbf{\RL{كثافة الزراعة:}} \RL{100 شجرة لكل فدان}
    \item \textbf{\RL{إجمالي الأشجار:}} \RL{4,500 شجرة زيتون بالسعة الكاملة}
    \item \textbf{\RL{التطوير المرحلي:}}
    \begin{itemize}
        \item \RL{المرحلة 1 (2026-2027): 3 فدادين (300 شجرة)}
        \item \RL{المرحلة 2 (2027-2028): إجمالي 9 فدادين (900 شجرة)}
        \item \RL{المرحلة 3 (2028-2029): إجمالي 19 فدان (1,900 شجرة)}
        \item \RL{المرحلة 4 (2029-2030): إجمالي 34 فدان (3,400 شجرة)}
        \item \RL{المرحلة 5 (2030-2031): إجمالي 45 فدان (4,500 شجرة)}
    \end{itemize}
\end{itemize}

\subsection{\RL{متطلبات المياه}}
\begin{itemize}
    \item \textbf{\RL{احتياج المياه السنوي:}} \RL{4,000-6,000 متر مكعب لكل فدان}
    \item \textbf{\RL{إجمالي المياه السنوية (بالسعة الكاملة):}} \RL{180,000-270,000 متر مكعب}
    \item \textbf{\RL{نظام الري:}} \RL{الري بالتنقيط بكفاءة 85\%}
    \item \textbf{\RL{مصادر المياه:}}
    \begin{itemize}
        \item \RL{الأساسي: المياه الجوفية من بئر المشروع}
        \item \RL{الثانوي: مياه الصرف المعالجة من مرافق المشروع}
        \item \RL{التكميلي: أنظمة حصاد مياه الأمطار}
    \end{itemize}
    \item \textbf{\RL{تدابير الحفاظ على المياه:}}
    \begin{itemize}
        \item \RL{مراقبة رطوبة التربة}
        \item \RL{الري الناقص خلال الفترات غير الحرجة}
        \item \RL{التغطية وغطاء الأرض}
        \item \RL{مصدات الرياح لتقليل التبخر}
    \end{itemize}
\end{itemize}

\subsection{\RL{المدخلات المادية}}
\begin{itemize}
    \item \textbf{\RL{مواد الزراعة:}}
    \begin{itemize}
        \item \RL{شتلات الزيتون: 4,500 شجرة (على مراحل)}
        \item \RL{أصناف مقاومة للجفاف مناسبة لإنتاج الزيت}
        \item \RL{بذور الزراعة البينية (أعشاب طبية، بقوليات)}
        \item \RL{مصدات الرياح والنباتات المرافقة}
    \end{itemize}
    \item \textbf{\RL{محسنات التربة:}}
    \begin{itemize}
        \item \RL{السماد الدودي: 5-40 طن سنوياً (تزداد مع المراحل)}
        \item \RL{الفحم الحيوي: 5-40 طن سنوياً (تزداد مع المراحل)}
        \item \RL{سماد الآزولا: 2-20 طن سنوياً}
        \item \RL{المكملات المعدنية حسب الحاجة بناءً على اختبارات التربة}
    \end{itemize}
    \item \textbf{\RL{إدارة الآفات:}}
    \begin{itemize}
        \item \RL{عوامل المكافحة البيولوجية}
        \item \RL{مثبطات الآفات العضوية}
        \item \RL{معدات المراقبة}
        \item \RL{مدخلات كيميائية بالحد الأدنى عند الضرورة}
    \end{itemize}
\end{itemize}

\subsection{\RL{المعدات والبنية التحتية}}
\begin{itemize}
    \item \textbf{\RL{بنية تحتية للري:}}
    \begin{itemize}
        \item \RL{نظام الري بالتنقيط لـ 45 فدان}
        \item \RL{مضخات المياه وأنظمة الترشيح}
        \item \RL{خزانات تخزين المياه (سعة 50,000 لتر)}
        \item \RL{أجهزة استشعار رطوبة التربة ومعدات المراقبة}
    \end{itemize}
    \item \textbf{\RL{معدات المعالجة:}}
    \begin{itemize}
        \item \RL{معصرة زيتون (سعة: 500 كجم/ساعة)}
        \item \RL{خزانات تخزين زيت الزيتون (فولاذ مقاوم للصدأ)}
        \item \RL{معدات الترشيح والتعبئة}
        \item \RL{معدات مختبر اختبار الجودة}
    \end{itemize}
    \item \textbf{\RL{معدات المزرعة:}}
    \begin{itemize}
        \item \RL{جرار صغير مع ملحقاته}
        \item \RL{أدوات التقليم والحصاد}
        \item \RL{معدات الرش}
        \item \RL{مركبات النقل}
    \end{itemize}
    \item \textbf{\RL{المباني:}}
    \begin{itemize}
        \item \RL{مرفق المعالجة (200 متر مربع)}
        \item \RL{مستودع التخزين (150 متر مربع)}
        \item \RL{سقيفة المعدات (100 متر مربع)}
        \item \RL{مرافق الموظفين (50 متر مربع)}
    \end{itemize}
\end{itemize}

\subsection{\RL{الموارد البشرية}}
\begin{itemize}
    \item \textbf{\RL{الموظفون الدائمون:}}
    \begin{itemize}
        \item \RL{متخصص في زراعة الزيتون (1)}
        \item \RL{مدير المزرعة (1)}
        \item \RL{فني معالجة (1)}
        \item \RL{عمال ميدانيون (4-8، يزدادون مع المراحل)}
        \item \RL{فني صيانة (1)}
    \end{itemize}
    \item \textbf{\RL{العمال الموسميون:}}
    \begin{itemize}
        \item \RL{فريق الحصاد (10-20 خلال موسم الحصاد)}
        \item \RL{فريق التقليم (5-10 خلال موسم التقليم)}
        \item \RL{مساعدو المعالجة (3-5 خلال موسم المعالجة)}
    \end{itemize}
    \item \textbf{\RL{الدعم الخارجي:}}
    \begin{itemize}
        \item \RL{مستشار جودة زيت الزيتون}
        \item \RL{متخصص في إدارة الآفات}
        \item \RL{متخصص في التسويق والمبيعات}
        \item \RL{فنيو صيانة المعدات}
    \end{itemize}
\end{itemize}

\subsection{\RL{الموارد المالية}}
\begin{itemize}
    \item \textbf{\RL{الاستثمار الرأسمالي:}}
    \begin{itemize}
        \item \RL{تحضير الأرض: 90,000 دولار}
        \item \RL{نظام الري: 135,000 دولار}
        \item \RL{الأشجار والزراعة: 67,500 دولار}
        \item \RL{معدات المعالجة: 150,000 دولار}
        \item \RL{المباني والبنية التحتية: 200,000 دولار}
        \item \RL{معدات المزرعة: 75,000 دولار}
        \item \RL{إجمالي الاستثمار الرأسمالي: 717,500 دولار}
    \end{itemize}
    \item \textbf{\RL{تكاليف التشغيل السنوية:}}
    \begin{itemize}
        \item \RL{العمالة: 60,000-120,000 دولار (تزداد مع المراحل)}
        \item \RL{المدخلات والمواد: 30,000-60,000 دولار}
        \item \RL{المياه والطاقة: 15,000-30,000 دولار}
        \item \RL{الصيانة: 20,000-40,000 دولار}
        \item \RL{التسويق والتوزيع: 25,000-50,000 دولار}
        \item \RL{إجمالي تكاليف التشغيل السنوية: 150,000-300,000 دولار}
    \end{itemize}
\end{itemize}

\subsection{\RL{موارد التكامل}}
\begin{itemize}
    \item \textbf{\RL{المدخلات من الوحدات الأخرى:}}
    \begin{itemize}
        \item \RL{السماد الدودي من وحدة التسميد الدودي}
        \item \RL{الفحم الحيوي من وحدة الانحلال الحراري}
        \item \RL{سماد الآزولا من برك الآزولا}
        \item \RL{المياه المعالجة من نظام إدارة المياه}
        \item \RL{الماشية للرعي والسماد}
    \end{itemize}
    \item \textbf{\RL{المخرجات إلى الوحدات الأخرى:}}
    \begin{itemize}
        \item \RL{مخلفات التقليم لإنتاج الفحم الحيوي}
        \item \RL{مخلفات المعالجة للتسميد الدودي}
        \item \RL{تفل الزيتون كمكمل لعلف الماشية}
        \item \RL{منتجات الزراعة البينية للسوق والماشية}
        \item \RL{خدمات النظام البيئي (احتجاز الكربون، التنوع البيولوجي)}
    \end{itemize}
\end{itemize}

\RL{تضمن خطة متطلبات الموارد هذه أن وحدة زراعة الزيتون لديها المدخلات اللازمة للتنفيذ الناجح مع تعظيم التكامل مع الوحدات الأخرى في مشروع الاقتصاد الدائري في الطور.}
