\section{\RL{الخطة التشغيلية لزراعة الزيتون}}

\subsection{\RL{جدول التنفيذ السنوي (2026-2031)}}

\subsubsection{\RL{السنة الأولى (2026-2027)}}
\begin{itemize}
    \item \textbf{\RL{تحضير الأرض:}}
    \begin{itemize}
        \item \RL{تحليل وتحسين التربة}
        \item \RL{تركيب نظام الري}
        \item \RL{إنشاء مصدات الرياح}
        \item \RL{إنشاء المدرجات حيثما يلزم}
    \end{itemize}
    \item \textbf{\RL{الزراعة:}}
    \begin{itemize}
        \item \RL{3 فدان (300 شجرة)}
        \item \RL{التباعد: 10م × 10م}
        \item \RL{اختيار أصناف مقاومة للجفاف}
        \item \RL{التطبيق الأولي للفحم الحيوي (5 أطنان)}
    \end{itemize}
    \item \textbf{\RL{الإدارة:}}
    \begin{itemize}
        \item \RL{جدولة الري}
        \item \RL{مكافحة الأعشاب (يدويًا وبالتغطية)}
        \item \RL{إعداد نظام مراقبة الآفات}
        \item \RL{الزراعة البينية مع الأعشاب الطبية}
    \end{itemize}
    \item \textbf{\RL{البنية التحتية:}}
    \begin{itemize}
        \item \RL{إنشاء المشتل}
        \item \RL{مرافق تخزين أساسية}
        \item \RL{طرق ومسارات الوصول}
        \item \RL{خزانات تخزين المياه}
    \end{itemize}
    \item \textbf{\RL{التكامل مع المشتل:}} \label{sec:olive_nursery_integration_ar}
    \begin{itemize}
        \item \textbf{\RL{الحصول على الشتلات الأولية:}}
        \begin{itemize}
            \item \RL{استلام 325 شتلة زيتون معتمدة من وحدة المشتل المركزية (ديسمبر 2026)}
            \item \RL{تنفيذ بروتوكول التحقق الوراثي لتأكيد الصنف (الترميز الشريطي للحمض النووي)}
            \item \RL{إجراء فترة تقسية لمدة 14 يومًا في بيئة انتقالية قبل الزراعة في الحقل}
            \item \RL{توثيق مصدر المادة الأصلية وتاريخ الأداء من سجلات المشتل}
        \end{itemize}
        \item \textbf{\RL{الإدارة الوراثية:}}
        \begin{itemize}
            \item \RL{إنشاء نظام تتبع الأصناف مع وضع علامات رمز الاستجابة السريعة (QR) لكل شجرة}
            \item \RL{تنفيذ قاعدة بيانات مشتركة مع وحدة المشتل (المرجع: \ref{sec:nursery_olive_integration})}
            \item \RL{إجراء تقييمات ظاهرية فصلية باستخدام بروتوكولات موحدة}
            \item \RL{إنشاء مخزون رقمي مع معلمات النمو المرتبطة بالملفات الجينية}
        \end{itemize}
    \end{itemize}
\end{itemize}

\subsubsection{\RL{السنة الثانية (2027-2028)}}
\begin{itemize}
    \item \textbf{\RL{التوسع:}}
    \begin{itemize}
        \item \RL{6 فدان إضافية (600 شجرة)}
        \item \RL{توسيع نظام الري}
        \item \RL{توسيع مناطق الزراعة البينية}
        \item \RL{تعزيز زراعة مصدات الرياح}
    \end{itemize}
    \item \textbf{\RL{الإدارة:}}
    \begin{itemize}
        \item \RL{تقليم أشجار السنة الأولى}
        \item \RL{تنفيذ برنامج التسميد}
        \item \RL{الإدارة المتكاملة للآفات}
        \item \RL{مراقبة رطوبة التربة}
    \end{itemize}
    \item \textbf{\RL{المعالجة:}}
    \begin{itemize}
        \item \RL{تركيب معصرة زيتون صغيرة}
        \item \RL{بروتوكولات المعالجة الأولية}
        \item \RL{أنظمة مراقبة الجودة}
        \item \RL{منشأة تعبئة على نطاق صغير}
    \end{itemize}
    \item \textbf{\RL{التكامل:}}
    \begin{itemize}
        \item \RL{التكامل الأولي مع الماشية (5 أبقار)}
        \item \RL{الاتصال ببرك الأزولا (3 فدان)}
        \item \RL{تطبيق محسن للفحم الحيوي (15 طن)}
        \item \RL{تكامل الدواجن (200 دجاجة، 100 بطة)}
    \end{itemize}
    \item \textbf{\RL{التكامل المتقدم مع المشتل:}}
    \begin{itemize}
        \item \textbf{\RL{تسليم الشتلات المجدول:}}
        \begin{itemize}
            \item \RL{استلام 650 شتلة زيتون معتمدة في ثلاث دفعات (أكتوبر، ديسمبر، فبراير)}
            \item \RL{تنفيذ الفحص قبل التسليم في منشأة المشتل مع توقيع مهندس زراعي}
            \item \RL{تنسيق الخدمات اللوجستية مع وقت نقل أقصاه 6 ساعات لتقليل الإجهاد}
            \item \RL{الحفاظ على منطقة تأقلم مخصصة مع بروتوكول انتقال لمدة 72 ساعة}
        \end{itemize}
        \item \textbf{\RL{نظام التغذية الراجعة للأداء:}}
        \begin{itemize}
            \item \RL{تقديم بيانات التأسيس لمدة 6 أشهر إلى المشتل لتحسين الإكثار}
            \item \RL{تنفيذ بروتوكول مشترك لمراقبة أداء الأصناف}
            \item \RL{المشاركة في اجتماعات التنسيق نصف الشهرية بين الوحدات}
            \item \RL{المساهمة ببيانات الأداء الميداني في برنامج التربية المركزي}
        \end{itemize}
    \end{itemize}
\end{itemize}

\subsubsection{\RL{السنة الثالثة (2028-2029)}}
\begin{itemize}
    \item \textbf{\RL{التوسع:}}
    \begin{itemize}
        \item \RL{إضافة 10 فدان (1000 شجرة)}
        \item \RL{تكنولوجيا ري متقدمة}
        \item \RL{توسيع نظام الزراعة البينية}
        \item \RL{تحسين إدارة التربة}
    \end{itemize}
    \item \textbf{\RL{الإدارة:}}
    \begin{itemize}
        \item \RL{برنامج تقليم مكثف}
        \item \RL{نظام تسميد متقدم}
        \item \RL{إدارة شاملة للآفات}
        \item \RL{أول حصاد كبير}
    \end{itemize}
    \item \textbf{\RL{المعالجة:}}
    \begin{itemize}
        \item \RL{تحسين مرفق المعالجة}
        \item \RL{التحضير لشهادة الجودة}
        \item \RL{تطوير منتجات ذات قيمة مضافة}
        \item \RL{توسيع سعة التخزين}
    \end{itemize}
    \item \textbf{\RL{التكامل:}}
    \begin{itemize}
        \item \RL{توسيع تكامل الثروة الحيوانية (15 بقرة)}
        \item \RL{الاتصال ببرك الآزولا (5 فدان)}
        \item \RL{الاستخدام الأمثل للفحم الحيوي (30 طن)}
        \item \RL{توسيع الدواجن (500 دجاجة، 200 بطة)}
    \end{itemize}
\end{itemize}

\subsubsection{\RL{السنة الرابعة (2029-2030)}}
\begin{itemize}
    \item \textbf{\RL{التوسع:}}
    \begin{itemize}
        \item \RL{إضافة 15 فدان (1500 شجرة)}
        \item \RL{أنظمة ري آلية}
        \item \RL{تنفيذ كامل للزراعة البينية}
        \item \RL{تقنيات متقدمة لإدارة التربة}
    \end{itemize}
    \item \textbf{\RL{الإدارة:}}
    \begin{itemize}
        \item \RL{حصاد على نطاق تجاري}
        \item \RL{تنفيذ الزراعة الدقيقة}
        \item \RL{أنظمة متقدمة لإدارة الآفات}
        \item \RL{إدارة مثلى للمياه}
    \end{itemize}
    \item \textbf{\RL{المعالجة:}}
    \begin{itemize}
        \item \RL{تكنولوجيا متقدمة لمعالجة زيت الزيتون}
        \item \RL{شهادة جودة كاملة}
        \item \RL{توسيع نطاق المنتجات}
        \item \RL{تطوير السوق والعلامة التجارية}
    \end{itemize}
    \item \textbf{\RL{التكامل:}}
    \begin{itemize}
        \item \RL{تكامل كامل مع الثروة الحيوانية (25 بقرة)}
        \item \RL{الاتصال ببرك الآزولا (30 فدان)}
        \item \RL{تطبيق أقصى للفحم الحيوي (40 طن)}
        \item \RL{تكامل كامل للدواجن (800 دجاجة، 300 بطة)}
    \end{itemize}
\end{itemize}

\subsubsection{\RL{السنة الخامسة (2030-2031)}}
\begin{itemize}
    \item \textbf{\RL{التوسع:}}
    \begin{itemize}
        \item \RL{11 فدان نهائي (1100 شجرة)}
        \item \RL{تحسين النظام}
        \item \RL{تنفيذ كامل للزراعة الحراجية}
        \item \RL{برنامج نهائي لتحسين التربة}
    \end{itemize}
    \item \textbf{\RL{الإدارة:}}
    \begin{itemize}
        \item \RL{كفاءة إنتاج قصوى}
        \item \RL{تنفيذ تكنولوجيا الزراعة الذكية}
        \item \RL{أنظمة مراقبة شاملة}
        \item \RL{بروتوكولات حصاد محسنة}
    \end{itemize}
    \item \textbf{\RL{المعالجة:}}
    \begin{itemize}
        \item \RL{مرفق معالجة كامل}
        \item \RL{تطوير منتجات ممتازة}
        \item \RL{تطوير سوق التصدير}
        \item \RL{تكامل كامل لسلسلة القيمة}
    \end{itemize}
    \item \textbf{\RL{التكامل:}}
    \begin{itemize}
        \item \RL{تكامل كامل مع الاقتصاد الدائري}
        \item \RL{الاتصال ببرك الآزولا القصوى (50 فدان)}
        \item \RL{دورة موارد محسنة}
        \item \RL{كفاءة قصوى للنظام}
    \end{itemize}
\end{itemize}

\subsection{\RL{البروتوكولات التشغيلية}}

\subsubsection{\RL{إدارة الري}}
\begin{itemize}
    \item \RL{نظام الري بالتنقيط بكفاءة 85\%}
    \item \RL{تكنولوجيا مراقبة رطوبة التربة}
    \item \RL{ري ناقص خلال الفترات غير الحرجة}
    \item \RL{أنظمة إعادة تدوير ومعالجة المياه}
    \item \RL{جدولة ري ذكية بناءً على بيانات المناخ}
\end{itemize}

\subsubsection{\RL{برنامج التسميد}}
\begin{itemize}
    \item \RL{مدخلات عضوية بشكل أساسي (سماد دودي، آزولا)}
    \item \RL{تطبيق الفحم الحيوي لاحتجاز الكربون}
    \item \RL{تطبيقات ورقية خلال مراحل النمو الحرجة}
    \item \RL{اختبار التربة وإدارة دقيقة للمغذيات}
    \item \RL{مدخلات اصطناعية بالحد الأدنى عند الضرورة}
\end{itemize}

\subsubsection{\RL{إدارة الآفات والأمراض}}
\begin{itemize}
    \item \RL{نهج الإدارة المتكاملة للآفات}
    \item \RL{عوامل المكافحة البيولوجية}
    \item \RL{أنظمة المراقبة والكشف المبكر}
    \item \RL{زراعة بينية استراتيجية لقمع الآفات}
    \item \RL{تدخلات كيميائية بالحد الأدنى عند الضرورة}
\end{itemize}

\subsubsection{\RL{الحصاد والمعالجة}}
\begin{itemize}
    \item \RL{توقيت أمثل لأقصى جودة للزيت}
    \item \RL{حصاد آلي للكفاءة}
    \item \RL{عصر بارد خلال 24 ساعة من الحصاد}
    \item \RL{مراقبة الجودة في جميع مراحل المعالجة}
    \item \RL{تخزين مناسب للحفاظ على الجودة}
\end{itemize}

\subsection{\RL{متطلبات المعدات}}

\subsection{\RL{تكامل سلسلة التوريد مع المشتل}}

\subsubsection{\RL{جدول زمني تفصيلي للتوريد}}
\begin{itemize}
    \item \textbf{\RL{دورة التخطيط السنوية:}}
    \begin{itemize}
        \item \RL{تقديم متطلبات الزراعة المستقبلية لمدة 24 شهرًا إلى وحدة المشتل بحلول أغسطس}
        \item \RL{استلام تأكيد جدول الإكثار بحلول أكتوبر}
        \item \RL{إجراء مراجعات التخطيط نصف السنوية (أبريل وأكتوبر)}
        \item \RL{المشاركة في لجنة اختيار الأصناف (يناير)}
    \end{itemize}
    \item \textbf{\RL{جدول الاستلام الموسمي:}}
    \begin{itemize}
        \item \RL{فترة التسليم الرئيسية: أكتوبر-ديسمبر (الزراعة المثلى لزيتون البحر المتوسط)}
        \item \RL{فترة التسليم الثانوية: فبراير-مارس (زراعة الربيع لأصناف محددة)}
        \item \RL{تخصيص الطوارئ: الاحتفاظ بمخزون احتياطي بنسبة 8٪ في المشتل}
        \item \RL{الأصناف المتخصصة: جدول إكثار مخصص مع مهلة 36 شهرًا}
    \end{itemize}
\end{itemize}

\subsubsection{\RL{بروتوكولات التحقق الوراثي}}
\begin{itemize}
    \item \textbf{\RL{طرق التحقق:}}
    \begin{itemize}
        \item \RL{البصمة الوراثية لجميع النباتات الأم باستخدام لوحة 12 واسم وراثي مجهري}
        \item \RL{أخذ عينات للتحقق من 5٪ من كل دفعة صنف مستلمة}
        \item \RL{المصادقة المورفولوجية عند 12 و24 شهرًا باستخدام قائمة واصفة موحدة}
        \item \RL{تحليل ملف الزيت عند الإنتاج الأول (4-5 سنوات) للتحقق النهائي}
    \end{itemize}
    \item \textbf{\RL{أنظمة التوثيق:}}
    \begin{itemize}
        \item \RL{جواز سفر وراثي رقمي لكل دفعة صنف باستخدام تقنية البلوكشين}
        \item \RL{قاعدة بيانات آمنة مع تتبع كامل للنسب من النبات الأم إلى موقع الحقل}
        \item \RL{علامات الأشجار المرمزة بالـ QR مرتبطة بقاعدة البيانات الجينية المركزية}
        \item \RL{وثائق الامتثال لمتطلبات تسمية المنشأ المحمية}
    \end{itemize}
\end{itemize}

\subsubsection{\RL{تكامل ضمان الجودة}}
\begin{itemize}
    \item \textbf{\RL{معايير القبول:}}
    \begin{itemize}
        \item \RL{الحد الأدنى لقطر الجذع: 1.5-2.0 سم على ارتفاع 10 سم}
        \item \RL{متطلبات نظام الجذر: 8 جذور رئيسية كحد أدنى، توزيع متوازن}
        \item \RL{فحص مسببات الأمراض: التحقق البصري والمختبري لـ 7 مسببات أمراض رئيسية للزيتون}
        \item \RL{اختبار الإجهاد قبل التسليم: محاكاة الجفاف لمدة 7 أيام وتقييم التعافي}
    \end{itemize}
    \item \textbf{\RL{مراقبة الأداء:}}
    \begin{itemize}
        \item \RL{جدول مراقبة ثلاثي المستويات: تقييمات 90 يومًا و6 أشهر و12 شهرًا}
        \item \RL{الإبلاغ عن معلمات النمو الموحدة باستخدام قياس رقمي معاير}
        \item \RL{مشاركة البيانات ثنائية الاتجاه من خلال نظام قاعدة بيانات متكامل}
        \item \RL{مراجعة سنوية لأداء الأصناف مع فريق إدارة المشتل}
    \end{itemize}
\end{itemize}

\subsubsection{\RL{الإشارة المرجعية إلى خطة تكامل المشتل}}
\begin{itemize}
    \item \RL{المواءمة المباشرة مع بروتوكولات إكثار الزيتون في خطة تكامل المشتل (القسم 4.2)}
    \item \RL{جداول الإنتاج المتزامنة وفقًا لتخطيط قدرة المشتل (القسم 3.6)}
    \item \RL{إجراءات التحقق الوراثي المتناغمة مع نظام ضمان الجودة بالمشتل (القسم 5.3)}
    \item \RL{أنظمة قواعد البيانات المتكاملة كما هو محدد في بروتوكولات إدارة البيانات بالمشتل (القسم 7.1)}
\end{itemize}

\RL{توفر هذه الخطة التشغيلية نهجًا منظمًا لتنفيذ وإدارة وحدة زراعة الزيتون، مما يضمن الاستخدام الفعال للموارد وممارسات الإنتاج المستدامة والتكامل السلس مع سلسلة التوريد الجينية لوحدة المشتل.}
