\section{\RL{الخطة التشغيلية لزراعة الزيتون}}

\subsection{\RL{جدول التنفيذ السنوي (2026-2031)}}

\subsubsection{\RL{السنة الأولى (2026-2027)}}
\begin{itemize}
    \item \textbf{\RL{تحضير الأرض:}}
    \begin{itemize}
        \item \RL{تحليل وتحسين التربة}
        \item \RL{تركيب نظام الري}
        \item \RL{إنشاء مصدات الرياح}
        \item \RL{إنشاء المدرجات حيثما يلزم}
    \end{itemize}
    \item \textbf{\RL{الزراعة:}}
    \begin{itemize}
        \item \RL{3 فدادين (300 شجرة)}
        \item \RL{المسافة بين الأشجار: 10م × 10م}
        \item \RL{اختيار أصناف مقاومة للجفاف}
        \item \RL{التطبيق الأولي للفحم الحيوي (5 أطنان)}
    \end{itemize}
    \item \textbf{\RL{الإدارة:}}
    \begin{itemize}
        \item \RL{جدولة الري}
        \item \RL{مكافحة الأعشاب الضارة (يدوياً وبالتغطية)}
        \item \RL{إعداد نظام مراقبة الآفات}
        \item \RL{الزراعة البينية مع النباتات الطبية}
    \end{itemize}
    \item \textbf{\RL{البنية التحتية:}}
    \begin{itemize}
        \item \RL{إنشاء المشتل}
        \item \RL{مرافق تخزين أساسية}
        \item \RL{طرق ومسارات الوصول}
        \item \RL{خزانات تخزين المياه}
    \end{itemize}
\end{itemize}

\subsubsection{\RL{السنة الثانية (2027-2028)}}
\begin{itemize}
    \item \textbf{\RL{التوسع:}}
    \begin{itemize}
        \item \RL{إضافة 6 فدادين (600 شجرة)}
        \item \RL{توسيع نظام الري}
        \item \RL{توسيع مناطق الزراعة البينية}
        \item \RL{تعزيز زراعة مصدات الرياح}
    \end{itemize}
    \item \textbf{\RL{الإدارة:}}
    \begin{itemize}
        \item \RL{تقليم أشجار السنة الأولى}
        \item \RL{تنفيذ برنامج التسميد}
        \item \RL{الإدارة المتكاملة للآفات}
        \item \RL{مراقبة رطوبة التربة}
    \end{itemize}
    \item \textbf{\RL{المعالجة:}}
    \begin{itemize}
        \item \RL{تركيب معصرة زيتون صغيرة}
        \item \RL{بروتوكولات المعالجة الأولية}
        \item \RL{أنظمة مراقبة الجودة}
        \item \RL{مرفق تعبئة صغير الحجم}
    \end{itemize}
    \item \textbf{\RL{التكامل:}}
    \begin{itemize}
        \item \RL{التكامل الأولي مع الثروة الحيوانية (5 أبقار)}
        \item \RL{الاتصال ببرك الآزولا (3 فدادين)}
        \item \RL{تعزيز تطبيق الفحم الحيوي (15 طن)}
        \item \RL{تكامل الدواجن (200 دجاجة، 100 بطة)}
    \end{itemize}
\end{itemize}

\subsubsection{\RL{السنة الثالثة (2028-2029)}}
\begin{itemize}
    \item \textbf{\RL{التوسع:}}
    \begin{itemize}
        \item \RL{إضافة 10 فدادين (1000 شجرة)}
        \item \RL{تكنولوجيا ري متقدمة}
        \item \RL{توسيع نظام الزراعة البينية}
        \item \RL{تحسين إدارة التربة}
    \end{itemize}
    \item \textbf{\RL{الإدارة:}}
    \begin{itemize}
        \item \RL{برنامج تقليم مكثف}
        \item \RL{نظام تسميد متقدم}
        \item \RL{إدارة شاملة للآفات}
        \item \RL{أول حصاد كبير}
    \end{itemize}
    \item \textbf{\RL{المعالجة:}}
    \begin{itemize}
        \item \RL{تحسين مرفق المعالجة}
        \item \RL{التحضير لشهادة الجودة}
        \item \RL{تطوير منتجات ذات قيمة مضافة}
        \item \RL{توسيع سعة التخزين}
    \end{itemize}
    \item \textbf{\RL{التكامل:}}
    \begin{itemize}
        \item \RL{توسيع تكامل الثروة الحيوانية (15 بقرة)}
        \item \RL{الاتصال ببرك الآزولا (5 فدادين)}
        \item \RL{الاستخدام الأمثل للفحم الحيوي (30 طن)}
        \item \RL{توسيع الدواجن (500 دجاجة، 200 بطة)}
    \end{itemize}
\end{itemize}

\subsubsection{\RL{السنة الرابعة (2029-2030)}}
\begin{itemize}
    \item \textbf{\RL{التوسع:}}
    \begin{itemize}
        \item \RL{إضافة 15 فدان (1500 شجرة)}
        \item \RL{أنظمة ري آلية}
        \item \RL{تنفيذ كامل للزراعة البينية}
        \item \RL{تقنيات متقدمة لإدارة التربة}
    \end{itemize}
    \item \textbf{\RL{الإدارة:}}
    \begin{itemize}
        \item \RL{حصاد على نطاق تجاري}
        \item \RL{تنفيذ الزراعة الدقيقة}
        \item \RL{أنظمة متقدمة لإدارة الآفات}
        \item \RL{إدارة مثلى للمياه}
    \end{itemize}
    \item \textbf{\RL{المعالجة:}}
    \begin{itemize}
        \item \RL{تكنولوجيا متقدمة لمعالجة زيت الزيتون}
        \item \RL{شهادة جودة كاملة}
        \item \RL{توسيع نطاق المنتجات}
        \item \RL{تطوير السوق والعلامة التجارية}
    \end{itemize}
    \item \textbf{\RL{التكامل:}}
    \begin{itemize}
        \item \RL{تكامل كامل مع الثروة الحيوانية (25 بقرة)}
        \item \RL{الاتصال ببرك الآزولا (30 فدان)}
        \item \RL{تطبيق أقصى للفحم الحيوي (40 طن)}
        \item \RL{تكامل كامل للدواجن (800 دجاجة، 300 بطة)}
    \end{itemize}
\end{itemize}

\subsubsection{\RL{السنة الخامسة (2030-2031)}}
\begin{itemize}
    \item \textbf{\RL{التوسع:}}
    \begin{itemize}
        \item \RL{11 فدان نهائي (1100 شجرة)}
        \item \RL{تحسين النظام}
        \item \RL{تنفيذ كامل للزراعة الحراجية}
        \item \RL{برنامج نهائي لتحسين التربة}
    \end{itemize}
    \item \textbf{\RL{الإدارة:}}
    \begin{itemize}
        \item \RL{كفاءة إنتاج قصوى}
        \item \RL{تنفيذ تكنولوجيا الزراعة الذكية}
        \item \RL{أنظمة مراقبة شاملة}
        \item \RL{بروتوكولات حصاد محسنة}
    \end{itemize}
    \item \textbf{\RL{المعالجة:}}
    \begin{itemize}
        \item \RL{مرفق معالجة كامل}
        \item \RL{تطوير منتجات ممتازة}
        \item \RL{تطوير سوق التصدير}
        \item \RL{تكامل كامل لسلسلة القيمة}
    \end{itemize}
    \item \textbf{\RL{التكامل:}}
    \begin{itemize}
        \item \RL{تكامل كامل مع الاقتصاد الدائري}
        \item \RL{الاتصال ببرك الآزولا القصوى (50 فدان)}
        \item \RL{دورة موارد محسنة}
        \item \RL{كفاءة قصوى للنظام}
    \end{itemize}
\end{itemize}

\subsection{\RL{البروتوكولات التشغيلية}}

\subsubsection{\RL{إدارة الري}}
\begin{itemize}
    \item \RL{نظام الري بالتنقيط بكفاءة 85\%}
    \item \RL{تكنولوجيا مراقبة رطوبة التربة}
    \item \RL{ري ناقص خلال الفترات غير الحرجة}
    \item \RL{أنظمة إعادة تدوير ومعالجة المياه}
    \item \RL{جدولة ري ذكية بناءً على بيانات المناخ}
\end{itemize}

\subsubsection{\RL{برنامج التسميد}}
\begin{itemize}
    \item \RL{مدخلات عضوية بشكل أساسي (سماد دودي، آزولا)}
    \item \RL{تطبيق الفحم الحيوي لاحتجاز الكربون}
    \item \RL{تطبيقات ورقية خلال مراحل النمو الحرجة}
    \item \RL{اختبار التربة وإدارة دقيقة للمغذيات}
    \item \RL{مدخلات اصطناعية بالحد الأدنى عند الضرورة}
\end{itemize}

\subsubsection{\RL{إدارة الآفات والأمراض}}
\begin{itemize}
    \item \RL{نهج الإدارة المتكاملة للآفات}
    \item \RL{عوامل المكافحة البيولوجية}
    \item \RL{أنظمة المراقبة والكشف المبكر}
    \item \RL{زراعة بينية استراتيجية لقمع الآفات}
    \item \RL{تدخلات كيميائية بالحد الأدنى عند الضرورة}
\end{itemize}

\subsubsection{\RL{الحصاد والمعالجة}}
\begin{itemize}
    \item \RL{توقيت أمثل لأقصى جودة للزيت}
    \item \RL{حصاد آلي للكفاءة}
    \item \RL{عصر بارد خلال 24 ساعة من الحصاد}
    \item \RL{مراقبة الجودة في جميع مراحل المعالجة}
    \item \RL{تخزين مناسب للحفاظ على الجودة}
\end{itemize}

\RL{تضمن هذه الخطة التشغيلية التطوير المنهجي لوحدة زراعة الزيتون، مع أهداف سنوية واضحة وبروتوكولات إدارية تتماشى مع أهداف الاقتصاد الدائري الأوسع لمشروع الطور.}
