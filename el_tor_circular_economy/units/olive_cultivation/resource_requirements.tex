\section{Resource Requirements for Olive Cultivation}

\subsection{Land Requirements}
\begin{itemize}
    \item \textbf{Total Area:} 45 Feddans (18.9 hectares)
    \item \textbf{Planting Density:} 100 trees per Feddan
    \item \textbf{Total Trees:} 4,500 olive trees at full capacity
    \item \textbf{Phased Development:}
    \begin{itemize}
        \item Phase 1 (2026-2027): 3 Feddans (300 trees)
        \item Phase 2 (2027-2028): 9 Feddans total (900 trees)
        \item Phase 3 (2028-2029): 19 Feddans total (1,900 trees)
        \item Phase 4 (2029-2030): 34 Feddans total (3,400 trees)
        \item Phase 5 (2030-2031): 45 Feddans total (4,500 trees)
    \end{itemize}
\end{itemize}

\subsection{Water Requirements}
\begin{itemize}
    \item \textbf{Annual Water Need:} 4,000-6,000 m³ per Feddan
    \item \textbf{Total Annual Water (at full capacity):} 180,000-270,000 m³
    \item \textbf{Irrigation System:} Drip irrigation with 85\% efficiency
    \item \textbf{Water Sources:}
    \begin{itemize}
        \item Primary: Groundwater from project well
        \item Secondary: Treated wastewater from project facilities
        \item Supplementary: Rainwater harvesting systems
    \end{itemize}
    \item \textbf{Water Conservation Measures:}
    \begin{itemize}
        \item Soil moisture monitoring
        \item Deficit irrigation during non-critical periods
        \item Mulching and ground cover
        \item Windbreaks to reduce evaporation
    \end{itemize}
\end{itemize}

\subsection{Material Inputs}
\begin{itemize}
    \item \textbf{Planting Materials:}
    \begin{itemize}
        \item Olive saplings: 4,500 trees (phased)
        \item Drought-resistant varieties suitable for oil production
        \item Intercropping seeds (medicinal herbs, legumes)
        \item Windbreak and companion plants
    \end{itemize}
    \item \textbf{Soil Amendments:}
    \begin{itemize}
        \item Vermicompost: 5-40 tons annually (increasing with phases)
        \item Biochar: 5-40 tons annually (increasing with phases)
        \item Azolla-based fertilizer: 2-20 tons annually
        \item Mineral supplements as needed based on soil tests
    \end{itemize}
    \item \textbf{Pest Management:}
    \begin{itemize}
        \item Biological control agents
        \item Organic pest deterrents
        \item Monitoring equipment
        \item Minimal chemical inputs when necessary
    \end{itemize}
\end{itemize}

\subsection{Equipment and Infrastructure}
\begin{itemize}
    \item \textbf{Irrigation Infrastructure:}
    \begin{itemize}
        \item Drip irrigation system for 45 Feddans
        \item Water pumps and filtration systems
        \item Water storage tanks (50,000 liters capacity)
        \item Soil moisture sensors and monitoring equipment
    \end{itemize}
    \item \textbf{Processing Equipment:}
    \begin{itemize}
        \item Olive press (capacity: 500 kg/hour)
        \item Olive oil storage tanks (stainless steel)
        \item Filtration and bottling equipment
        \item Quality testing laboratory equipment
    \end{itemize}
    \item \textbf{Farm Equipment:}
    \begin{itemize}
        \item Small tractor with implements
        \item Pruning and harvesting tools
        \item Spraying equipment
        \item Transportation vehicles
    \end{itemize}
    \item \textbf{Buildings:}
    \begin{itemize}
        \item Processing facility (200 m²)
        \item Storage warehouse (150 m²)
        \item Equipment shed (100 m²)
        \item Staff facilities (50 m²)
    \end{itemize}
\end{itemize}

\subsection{Human Resources}
\begin{itemize}
    \item \textbf{Permanent Staff:}
    \begin{itemize}
        \item Olive cultivation specialist (1)
        \item Farm manager (1)
        \item Processing technician (1)
        \item Field workers (4-8, increasing with phases)
        \item Maintenance technician (1)
    \end{itemize}
    \item \textbf{Seasonal Workers:}
    \begin{itemize}
        \item Harvesting crew (10-20 during harvest season)
        \item Pruning crew (5-10 during pruning season)
        \item Processing assistants (3-5 during processing season)
    \end{itemize}
    \item \textbf{External Support:}
    \begin{itemize}
        \item Olive oil quality consultant
        \item Pest management specialist
        \item Marketing and sales specialist
        \item Equipment maintenance technicians
    \end{itemize}
\end{itemize}

\subsection{Financial Resources}
\begin{itemize}
    \item \textbf{Capital Investment:}
    \begin{itemize}
        \item Land preparation: \$90,000
        \item Irrigation system: \$135,000
        \item Trees and planting: \$67,500
        \item Processing equipment: \$150,000
        \item Buildings and infrastructure: \$200,000
        \item Farm equipment: \$75,000
        \item Total capital investment: \$717,500
    \end{itemize}
    \item \textbf{Annual Operating Costs:}
    \begin{itemize}
        \item Labor: \$60,000-120,000 (increasing with phases)
        \item Inputs and materials: \$30,000-60,000
        \item Water and energy: \$15,000-30,000
        \item Maintenance: \$20,000-40,000
        \item Marketing and distribution: \$25,000-50,000
        \item Total annual operating costs: \$150,000-300,000
    \end{itemize}
\end{itemize}

\subsection{Integration Resources}
\begin{itemize}
    \item \textbf{Inputs from Other Units:}
    \begin{itemize}
        \item Vermicompost from vermicomposting unit
        \item Biochar from pyrolysis unit
        \item Azolla-based fertilizer from Azolla ponds
        \item Treated water from water management system
        \item Livestock for grazing and manure
    \end{itemize}
    \item \textbf{Outputs to Other Units:}
    \begin{itemize}
        \item Pruning waste to biochar production
        \item Processing waste to vermicomposting
        \item Olive pomace for livestock feed supplement
        \item Intercropping products for market and livestock
        \item Ecosystem services (carbon sequestration, biodiversity)
    \end{itemize}
\end{itemize}

This resource requirements plan ensures the olive cultivation unit has the necessary inputs for successful implementation while maximizing integration with other units in the El Tor Circular Economy project.
