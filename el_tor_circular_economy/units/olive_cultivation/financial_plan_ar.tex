\section{الخطة المالية لزراعة الزيتون}

\subsection{متطلبات الاستثمار الرأسمالي}

\begin{table}[h]
\centering
\begin{tabular}{|r|r|}
\hline
\textbf{المبلغ (دولار أمريكي)} & \textbf{فئة الاستثمار} \\
\hline
90,000 دولار & إعداد الأرض \\
135,000 دولار & نظام الري \\
67,500 دولار & الأشجار والزراعة \\
150,000 دولار & معدات المعالجة \\
200,000 دولار & المباني والبنية التحتية \\
75,000 دولار & معدات المزرعة \\
\hline
717,500 دولار & \textbf{إجمالي الاستثمار الرأسمالي} \\
\hline
\end{tabular}
\caption{تفصيل الاستثمار الرأسمالي}
\end{table}

\subsubsection{جدول الاستثمار المرحلي}
\begin{itemize}
    \item \textbf{المرحلة الأولى (2026-2027):} 215,000 دولار
    \begin{itemize}
        \item إعداد الأرض (3 فدان): 18,000 دولار
        \item نظام الري الأولي: 27,000 دولار
        \item الأشجار والزراعة الأولية: 13,500 دولار
        \item معدات المزرعة الأساسية: 30,000 دولار
        \item البنية التحتية الأولية: 40,000 دولار
        \item إنشاء المشتل: 15,000 دولار
        \item تخزين المياه: 25,000 دولار
        \item تحضير التربة: 20,000 دولار
        \item التسييج والأمن: 15,000 دولار
        \item التخطيط الفني: 11,500 دولار
    \end{itemize}
    
    \item \textbf{المرحلة الثانية (2027-2028):} 172,500 دولار
    \begin{itemize}
        \item إعداد الأرض (6 فدان): 36,000 دولار
        \item توسيع نظام الري: 27,000 دولار
        \item أشجار وزراعة إضافية: 27,000 دولار
        \item معصرة زيتون صغيرة: 60,000 دولار
        \item مرافق التخزين: 22,500 دولار
    \end{itemize}
    
    \item \textbf{المرحلة الثالثة (2028-2029):} 150,000 دولار
    \begin{itemize}
        \item إعداد الأرض (10 فدان): 60,000 دولار
        \item توسيع نظام الري: 30,000 دولار
        \item أشجار وزراعة إضافية: 45,000 دولار
        \item تحسين مرافق المعالجة: 15,000 دولار
    \end{itemize}
    
    \item \textbf{المرحلة الرابعة (2029-2030):} 120,000 دولار
    \begin{itemize}
        \item إعداد الأرض (15 فدان): 90,000 دولار
        \item توسيع نظام الري: 45,000 دولار
        \item أشجار وزراعة إضافية: 67,500 دولار
        \item معدات معالجة متقدمة: 75,000 دولار
        \item توسيع البنية التحتية: 42,500 دولار
    \end{itemize}
    
    \item \textbf{المرحلة الخامسة (2030-2031):} 60,000 دولار
    \begin{itemize}
        \item إعداد الأرض (11 فدان): 66,000 دولار
        \item نظام الري النهائي: 33,000 دولار
        \item الأشجار والزراعة النهائية: 49,500 دولار
        \item تحسين النظام: 15,000 دولار
        \item البنية التحتية النهائية: 15,000 دولار
    \end{itemize}
\end{itemize}

\subsection{تكاليف التشغيل}

\begin{table}[h]
\centering
\begin{tabular}{|r|r|r|r|r|r|}
\hline
\textbf{السنة 5} & \textbf{السنة 4} & \textbf{السنة 3} & \textbf{السنة 2} & \textbf{السنة 1} & \textbf{فئة التكلفة} \\
\hline
120,000 دولار & 105,000 دولار & 90,000 دولار & 75,000 دولار & 60,000 دولار & العمالة \\
60,000 دولار & 52,500 دولار & 45,000 دولار & 37,500 دولار & 30,000 دولار & المدخلات والمواد \\
30,000 دولار & 26,250 دولار & 22,500 دولار & 18,750 دولار & 15,000 دولار & المياه والطاقة \\
40,000 دولار & 35,000 دولار & 30,000 دولار & 25,000 دولار & 20,000 دولار & الصيانة \\
50,000 دولار & 43,750 دولار & 37,500 دولار & 31,250 دولار & 25,000 دولار & التسويق والتوزيع \\
\hline
300,000 دولار & 262,500 دولار & 225,000 دولار & 187,500 دولار & 150,000 دولار & \textbf{إجمالي تكاليف التشغيل السنوية} \\
\hline
\end{tabular}
\caption{توقعات تكاليف التشغيل السنوية}
\end{table}

\subsubsection{تفاصيل تكاليف التشغيل}
\begin{itemize}
    \item \textbf{العمالة:}
    \begin{itemize}
        \item الموظفون الدائمون: 40,000-80,000 دولار سنويًا
        \item العمال الموسميون: 20,000-40,000 دولار سنويًا
        \item التدريب والتطوير: 5,000-10,000 دولار سنويًا
    \end{itemize}
    
    \item \textbf{المدخلات والمواد:}
    \begin{itemize}
        \item الأسمدة العضوية: 10,000-20,000 دولار سنويًا
        \item إدارة الآفات: 5,000-10,000 دولار سنويًا
        \item مواد التعبئة: 10,000-20,000 دولار سنويًا
        \item مستلزمات أخرى: 5,000-10,000 دولار سنويًا
    \end{itemize}
    
    \item \textbf{المياه والطاقة:}
    \begin{itemize}
        \item مياه الري: 8,000-16,000 دولار سنويًا
        \item الكهرباء للمعالجة: 5,000-10,000 دولار سنويًا
        \item الوقود للمعدات: 2,000-4,000 دولار سنويًا
    \end{itemize}
    
    \item \textbf{الصيانة:}
    \begin{itemize}
        \item نظام الري: 5,000-10,000 دولار سنويًا
        \item معدات المعالجة: 8,000-16,000 دولار سنويًا
        \item المباني والبنية التحتية: 5,000-10,000 دولار سنويًا
        \item معدات المزرعة: 2,000-4,000 دولار سنويًا
    \end{itemize}
    
    \item \textbf{التسويق والتوزيع:}
    \begin{itemize}
        \item التعبئة والتغليف: 10,000-20,000 دولار سنويًا
        \item النقل: 5,000-10,000 دولار سنويًا
        \item التسويق والترويج: 8,000-16,000 دولار سنويًا
        \item شهادة الجودة: 2,000-4,000 دولار سنويًا
    \end{itemize}
\end{itemize}

\subsection{توقعات الإيرادات}

\begin{table}[h]
\centering
\begin{tabular}{|r|r|r|r|r|r|}
\hline
\textbf{السنة 5} & \textbf{السنة 4} & \textbf{السنة 3} & \textbf{السنة 2} & \textbf{السنة 1} & \textbf{مصدر الإيرادات} \\
\hline
450,000 دولار & 300,000 دولار & 150,000 دولار & 50,000 دولار & 0 دولار & مبيعات زيت الزيتون \\
100,000 دولار & 80,000 دولار & 60,000 دولار & 40,000 دولار & 20,000 دولار & منتجات الزراعة البينية \\
60,000 دولار & 45,000 دولار & 30,000 دولار & 15,000 دولار & 5,000 دولار & المنتجات الثانوية \\
40,000 دولار & 30,000 دولار & 20,000 دولار & 10,000 دولار & 0 دولار & خدمات النظام البيئي \\
\hline
650,000 دولار & 455,000 دولار & 260,000 دولار & 115,000 دولار & 25,000 دولار & \textbf{إجمالي الإيرادات السنوية} \\
\hline
\end{tabular}
\caption{توقعات الإيرادات السنوية}
\end{table}

\subsubsection{تفاصيل مصادر الإيرادات}
\begin{itemize}
    \item \textbf{مبيعات زيت الزيتون:}
    \begin{itemize}
        \item زيت الزيتون الممتاز: 15-20 دولار للتر
        \item زيت الزيتون القياسي: 10-15 دولار للتر
        \item الزيوت المنكهة/المتخصصة: 20-30 دولار للتر
    \end{itemize}
    
    \item \textbf{منتجات الزراعة البينية:}
    \begin{itemize}
        \item الأعشاب الطبية: 5,000-20,000 دولار سنويًا
        \item البقوليات: 10,000-30,000 دولار سنويًا
        \item محاصيل العلف: 5,000-50,000 دولار سنويًا
    \end{itemize}
    
    \item \textbf{المنتجات الثانوية:}
    \begin{itemize}
        \item تفل الزيتون لعلف الحيوانات: 10,000-20,000 دولار سنويًا
        \item أوراق الزيتون للشاي العشبي: 5,000-15,000 دولار سنويًا
        \item مكونات مستحضرات التجميل: 10,000-25,000 دولار سنويًا
    \end{itemize}
    
    \item \textbf{خدمات النظام البيئي:}
    \begin{itemize}
        \item ائتمانات الكربون: 10,000-25,000 دولار سنويًا
        \item تعزيز التنوع البيولوجي: 5,000-10,000 دولار سنويًا
        \item السياحة الزراعية/التعليمية: 5,000-15,000 دولار سنويًا
    \end{itemize}
\end{itemize}

\subsection{التحليل المالي}

\begin{table}[h]
\centering
\begin{tabular}{|r|r|r|r|r|r|}
\hline
\textbf{السنة 5} & \textbf{السنة 4} & \textbf{السنة 3} & \textbf{السنة 2} & \textbf{السنة 1} & \textbf{المؤشر المالي} \\
\hline
650,000 دولار & 455,000 دولار & 260,000 دولار & 115,000 دولار & 25,000 دولار & إجمالي الإيرادات \\
300,000 دولار & 262,500 دولار & 225,000 دولار & 187,500 دولار & 150,000 دولار & إجمالي تكاليف التشغيل \\
60,000 دولار & 120,000 دولار & 150,000 دولار & 172,500 دولار & 215,000 دولار & الاستثمار الرأسمالي \\
\hline
290,000 دولار & 72,500 دولار & -115,000 دولار & -245,000 دولار & -340,000 دولار & صافي التدفق النقدي \\
-337,500 دولار & -627,500 دولار & -700,000 دولار & -585,000 دولار & -340,000 دولار & التدفق النقدي التراكمي \\
\hline
\end{tabular}
\caption{توقعات التدفق النقدي (السنوات الخمس الأولى)}
\end{table}

\subsubsection{التوقعات المالية طويلة الأجل}
\begin{itemize}
    \item \textbf{نقطة التعادل:} السنة الثامنة (2033)
    \item \textbf{العائد على الاستثمار:} 15-20\% بعد النضج الكامل
    \item \textbf{معدل العائد الداخلي (IRR):} 12-15\% (أفق 10 سنوات)
    \item \textbf{صافي القيمة الحالية (NPV):} 1.2-1.5 مليون دولار (أفق 10 سنوات، معدل خصم 8\%)
    \item \textbf{مؤشر الربحية:} 1.7-2.1
\end{itemize}

\subsection{استراتيجية التمويل}
\begin{itemize}
    \item \textbf{استثمار رأس المال:} 40\% (287,000 دولار)
    \item \textbf{التمويل بالديون:} 35\% (251,125 دولار)
    \item \textbf{المنح والإعانات:} 15\% (107,625 دولار)
    \item \textbf{إعادة استثمار الإيرادات:} 10\% (71,750 دولار)
\end{itemize}

\subsubsection{مصادر التمويل المحتملة}
\begin{itemize}
    \item بنوك التنمية الزراعية
    \item مبادرات تمويل المناخ
    \item صناديق الاستثمار في الزراعة المستدامة
    \item الإعانات الحكومية للزراعة الموفرة للمياه
    \item التمويل المسبق لائتمانات الكربون
    \item المستثمرون المهتمون بالزراعة المستدامة
\end{itemize}

\subsection{إدارة المخاطر}
\begin{itemize}
    \item \textbf{مخاطر السوق:}
    \begin{itemize}
        \item تخفيف تقلبات الأسعار من خلال تنويع المنتجات
        \item عقود مستقبلية مع المشترين المميزين
        \item تطوير قنوات البيع المباشر للمستهلك
    \end{itemize}
    
    \item \textbf{مخاطر الإنتاج:}
    \begin{itemize}
        \item التأمين على المحاصيل ضد الظواهر المناخية القاسية
        \item تنويع الأصناف لتوزيع مخاطر الأمراض
        \item أمن المياه من خلال مصادر متعددة
    \end{itemize}
    
    \item \textbf{المخاطر المالية:}
    \begin{itemize}
        \item الاستثمار المرحلي للحد من التعرض للمخاطر
        \item مصادر إيرادات متعددة لضمان التدفق النقدي
        \item التحوط من تقلبات العملة للمبيعات التصديرية
    \end{itemize}
\end{itemize}

توضح هذه الخطة المالية الجدوى الاقتصادية لوحدة زراعة الزيتون ضمن مشروع الاقتصاد الدائري في الطور، مع عوائد قوية على المدى الطويل رغم متطلبات الاستثمار الأولي الكبيرة. يخلق التكامل مع وحدات المشروع الأخرى تآزرًا تشغيليًا يعزز الأداء المالي العام.
