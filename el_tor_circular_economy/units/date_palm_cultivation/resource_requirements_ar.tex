\section{متطلبات الموارد لزراعة النخيل}

\subsection{متطلبات التنفيذ المرحلي (2026-2031)}

\subsubsection{المرحلة الأولى (2026-2027) - التأسيس الأولي}
\begin{itemize}
    \item \textbf{الموارد الأرضية:}
    \begin{itemize}
        \item 5 فدان للزراعة الأولية
        \item 0.5 فدان للمشتل
        \item منطقة تجهيز التربة الأساسية
    \end{itemize}
    \item \textbf{الموارد المائية:}
    \begin{itemize}
        \item 15 متر مكعب/يوم من المياه المعالجة
        \item البنية التحتية الأساسية للري
        \item معدات مراقبة جودة المياه
    \end{itemize}
    \item \textbf{الموارد البشرية:}
    \begin{itemize}
        \item مهندس زراعي واحد
        \item 3 عمال مهرة
        \item 5 عمال عاديين
    \end{itemize}
    \item \textbf{المعدات:}
    \begin{itemize}
        \item أدوات زراعية أساسية
        \item جرار صغير
        \item مكونات الري الأولية
    \end{itemize}
\end{itemize}

\subsubsection{المرحلة الثانية (2027-2028) - التطوير المبكر}
\begin{itemize}
    \item \textbf{الموارد الأرضية:}
    \begin{itemize}
        \item التوسع إلى 15 فدان
        \item فدان واحد لعمليات المشتل
        \item إنشاء منطقة المعالجة
    \end{itemize}
    \item \textbf{الموارد المائية:}
    \begin{itemize}
        \item 45 متر مكعب/يوم من المياه المعالجة
        \item نظام ري محسن
        \item إعداد إعادة تدوير المياه
    \end{itemize}
    \item \textbf{الموارد البشرية:}
    \begin{itemize}
        \item مهندسان زراعيان
        \item 5 عمال مهرة
        \item 8 عمال عاديين
    \end{itemize}
    \item \textbf{المعدات:}
    \begin{itemize}
        \item معدات زراعية إضافية
        \item أدوات معالجة أساسية
        \item نظام ري موسع
    \end{itemize}
\end{itemize}

\subsubsection{المرحلة الثالثة (2028-2029) - التوسع}
\begin{itemize}
    \item \textbf{الموارد الأرضية:}
    \begin{itemize}
        \item التوسع إلى 30 فدان
        \item 1.5 فدان للمرافق المساندة
        \item منشأة معالجة كاملة
    \end{itemize}
    \item \textbf{الموارد المائية:}
    \begin{itemize}
        \item 90 متر مكعب/يوم من المياه المعالجة
        \item نظام ري متقدم
        \item تكامل كامل لإعادة تدوير المياه
    \end{itemize}
    \item \textbf{الموارد البشرية:}
    \begin{itemize}
        \item 3 مهندسين زراعيين
        \item 8 عمال مهرة
        \item 12 عامل عادي
    \end{itemize}
    \item \textbf{المعدات:}
    \begin{itemize}
        \item أسطول زراعي كامل
        \item معدات المعالجة
        \item مرافق التخزين
    \end{itemize}
\end{itemize}

\subsubsection{المرحلة الرابعة (2029-2030) - العمليات المتقدمة}
\begin{itemize}
    \item \textbf{الموارد الأرضية:}
    \begin{itemize}
        \item التوسع إلى 45 فدان
        \item فدانان للمرافق المساندة
        \item مناطق معالجة متقدمة
    \end{itemize}
    \item \textbf{الموارد المائية:}
    \begin{itemize}
        \item 135 متر مكعب/يوم من المياه المعالجة
        \item أنظمة ري آلية
        \item إدارة مياه متقدمة
    \end{itemize}
    \item \textbf{الموارد البشرية:}
    \begin{itemize}
        \item 4 مهندسين زراعيين
        \item 10 عمال مهرة
        \item 15 عامل عادي
    \end{itemize}
    \item \textbf{المعدات:}
    \begin{itemize}
        \item أنظمة زراعة آلية
        \item خط معالجة متقدم
        \item معدات مراقبة الجودة
    \end{itemize}
\end{itemize}

\subsubsection{المرحلة الخامسة (2030-2031) - التشغيل الكامل}
\begin{itemize}
    \item \textbf{الموارد الأرضية:}
    \begin{itemize}
        \item التوسع النهائي إلى 60 فدان
        \item 2.5 فدان للمرافق المساندة
        \item تكامل كامل للمنشآت
    \end{itemize}
    \item \textbf{الموارد المائية:}
    \begin{itemize}
        \item 180 متر مكعب/يوم من المياه المعالجة
        \item أنظمة ري محسنة
        \item كفاءة مائية قصوى
    \end{itemize}
    \item \textbf{الموارد البشرية:}
    \begin{itemize}
        \item 5 مهندسين زراعيين
        \item 12 عامل مهرة
        \item 20 عامل عادي
    \end{itemize}
    \item \textbf{المعدات:}
    \begin{itemize}
        \item أنظمة أتمتة كاملة
        \item منشآت معالجة متكاملة
        \item أنظمة مراقبة متكاملة
    \end{itemize}
\end{itemize}

\subsection{مؤشرات كفاءة الموارد}
\begin{itemize}
    \item \textbf{كفاءة استخدام المياه:}
    \begin{itemize}
        \item المرحلة الأولى: 3 متر مكعب/فدان/يوم
        \item المرحلة الثانية: 3 متر مكعب/فدان/يوم
        \item المرحلة الثالثة: 3 متر مكعب/فدان/يوم
        \item المرحلة الرابعة: 3 متر مكعب/فدان/يوم
        \item المرحلة الخامسة: 3 متر مكعب/فدان/يوم
    \end{itemize}
    \item \textbf{كفاءة العمالة:}
    \begin{itemize}
        \item المرحلة الأولى: 1.8 عامل/فدان
        \item المرحلة الثانية: 1.0 عامل/فدان
        \item المرحلة الثالثة: 0.8 عامل/فدان
        \item المرحلة الرابعة: 0.6 عامل/فدان
        \item المرحلة الخامسة: 0.5 عامل/فدان
    \end{itemize}
    \item \textbf{استخدام المعدات:}
    \begin{itemize}
        \item المرحلة الأولى: 60\% معدل الاستخدام
        \item المرحلة الثانية: 70\% معدل الاستخدام
        \item المرحلة الثالثة: 80\% معدل الاستخدام
        \item المرحلة الرابعة: 90\% معدل الاستخدام
        \item المرحلة الخامسة: 95\% معدل الاستخدام
    \end{itemize}
\end{itemize}

تحدد خطة متطلبات الموارد هذه التدرج في توسيع الموارد اللازمة لوحدة زراعة النخيل، مع ضمان الاستخدام الفعال للموارد خلال مراحل التنفيذ.
