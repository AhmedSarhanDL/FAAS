\section{\RL{الخطة التشغيلية لزراعة النخيل}}

\subsection{\RL{التنفيذ المرحلي (2026-2031)}}

\subsubsection{\RL{عمليات المرحلة الأولى (2026-2027)}}
\begin{itemize}
    \item \textbf{\RL{تحضير الأرض:}}
    \begin{itemize}
        \item \RL{تحليل التربة الأولي وتعديلها}
        \item \RL{تركيب نظام الري بالتنقيط الأساسي}
        \item \RL{تصميم تخطيط الحقل والمسافات}
        \item \RL{إنشاء مصدات الرياح}
    \end{itemize}
    \item \textbf{\RL{عمليات الزراعة:}}
    \begin{itemize}
        \item \RL{إنشاء المشتل (سعة 2500)}
        \item \RL{الزراعة الأولية لـ 200 نخلة}
        \item \RL{تجارب اختيار الأصناف}
        \item \RL{جدولة الري الأساسية}
    \end{itemize}
    \item \textbf{\RL{أنظمة الإدارة:}}
    \begin{itemize}
        \item \RL{حفظ السجلات الأساسية}
        \item \RL{التدريب الأولي للموظفين}
        \item \RL{جداول صيانة المعدات}
        \item \RL{بروتوكولات المراقبة البسيطة}
    \end{itemize}
    \item \textbf{\RL{التكامل مع المشتل:}} \label{sec:date_palm_nursery_integration_ar}
    \begin{itemize}
        \item \textbf{\RL{الحصول على الشتلات الأولية:}}
        \begin{itemize}
            \item \RL{استلام 250 شتلة منتجة بزراعة الأنسجة من وحدة المشتل المركزية في يناير 2026}
            \item \RL{إكمال عملية التحقق الوراثي (اختبار PCR) لتأكيد الصنف}
            \item \RL{تنفيذ بروتوكول التأقلم لمدة 21 يومًا في بيئة متحكم بها}
            \item \RL{توثيق التاريخ الوراثي للمادة المصدرية وخصائص الأداء}
        \end{itemize}
        \item \textbf{\RL{الإدارة الوراثية:}}
        \begin{itemize}
            \item \RL{إنشاء نظام تتبع الأصناف بمعرفات فريدة}
            \item \RL{تنفيذ قاعدة بيانات مرجعية متبادلة مع وحدة المشتل (المرجع: \ref{sec:nursery_palm_integration})}
            \item \RL{إجراء تقييمات النمو الشهرية مع مشاركة البيانات مع المشتل}
            \item \RL{الاحتفاظ بسجلات فينولوجية رقمية لبرنامج التربية}
        \end{itemize}
    \end{itemize}
\end{itemize}

\subsubsection{\RL{عمليات المرحلة الثانية (2027-2028)}}
\begin{itemize}
    \item \textbf{\RL{أنشطة التوسع:}}
    \begin{itemize}
        \item \RL{زراعة 400 نخلة إضافية}
        \item \RL{نظام ري محسن}
        \item \RL{تنفيذ الزراعة البينية}
        \item \RL{إعداد المعالجة الأولية}
    \end{itemize}
    \item \textbf{\RL{ممارسات الزراعة:}}
    \begin{itemize}
        \item \RL{برنامج التسميد}
        \item \RL{نظام مراقبة الآفات}
        \item \RL{جداول التقليم}
        \item \RL{إدارة التلقيح}
    \end{itemize}
    \item \textbf{\RL{إدارة الموارد:}}
    \begin{itemize}
        \item \RL{مراقبة استخدام المياه}
        \item \RL{تتبع المغذيات}
        \item \RL{أنظمة جمع النفايات}
        \item \RL{سجلات الإنتاج الأولية}
    \end{itemize}
    \item \textbf{\RL{التكامل المتقدم مع المشتل:}}
    \begin{itemize}
        \item \textbf{\RL{تسليم الشتلات المجدول:}}
        \begin{itemize}
            \item \RL{استلام 450 شتلة منتجة بزراعة الأنسجة (دفعات فصلية من 150) من وحدة المشتل}
            \item \RL{تنفيذ بروتوكول الفحص قبل التسليم في منشأة المشتل}
            \item \RL{تنسيق الخدمات اللوجستية للسلسلة الباردة مع نافذة تسليم مدتها 4 ساعات}
            \item \RL{الحفاظ على الحجر الصحي والفحص لمدة 48 ساعة قبل وضعها في الحقل}
        \end{itemize}
        \item \textbf{\RL{حلقة التغذية الراجعة للأداء:}}
        \begin{itemize}
            \item \RL{تقديم بيانات الأداء لمدة 90 يومًا إلى المشتل لتحسين الإكثار}
            \item \RL{توثيق استجابة الأصناف المحددة لظروف الحقل}
            \item \RL{المشاركة في اجتماعات التحسين الوراثي الشهرية بين الوحدات}
            \item \RL{المساهمة في صيانة قاعدة بيانات الأصناف المركزية}
        \end{itemize}
    \end{itemize}
\end{itemize}

\subsubsection{\RL{عمليات المرحلة الثالثة (2028-2029)}}
\begin{itemize}
    \item \textbf{\RL{الأنظمة المتقدمة:}}
    \begin{itemize}
        \item \RL{التحكم الآلي في الري}
        \item \RL{إدارة شاملة للآفات}
        \item \RL{عمليات معالجة كاملة}
        \item \RL{حفظ سجلات متقدم}
    \end{itemize}
    \item \textbf{\RL{إدارة الإنتاج:}}
    \begin{itemize}
        \item \RL{تحسين الإنتاجية}
        \item \RL{أنظمة مراقبة الجودة}
        \item \RL{جدولة الحصاد}
        \item \RL{معالجة ما بعد الحصاد}
    \end{itemize}
    \item \textbf{\RL{أنشطة التكامل:}}
    \begin{itemize}
        \item \RL{أنظمة رعي الماشية}
        \item \RL{عمليات التسميد}
        \item \RL{تطبيق الفحم الحيوي}
        \item \RL{إعادة تدوير المياه}
    \end{itemize}
\end{itemize}

\subsubsection{\RL{عمليات المرحلة الرابعة (2029-2030)}}
\begin{itemize}
    \item \textbf{\RL{الإنتاج المتقدم:}}
    \begin{itemize}
        \item \RL{تقنيات الزراعة الدقيقة}
        \item \RL{طرق تلقيح متقدمة}
        \item \RL{توقيت حصاد محسن}
        \item \RL{أنظمة تصنيف الجودة}
    \end{itemize}
    \item \textbf{\RL{عمليات المعالجة:}}
    \begin{itemize}
        \item \RL{معالجة القيمة المضافة}
        \item \RL{تنويع المنتجات}
        \item \RL{تحسين التخزين}
        \item \RL{تكامل السوق}
    \end{itemize}
    \item \textbf{\RL{تدابير الاستدامة:}}
    \begin{itemize}
        \item \RL{تتبع البصمة الكربونية}
        \item \RL{مراقبة التنوع البيولوجي}
        \item \RL{تقييم صحة التربة}
        \item \RL{مقاييس كفاءة المياه}
    \end{itemize}
\end{itemize}

\subsubsection{\RL{عمليات المرحلة الخامسة (2030-2031)}}
\begin{itemize}
    \item \textbf{\RL{تحسين النظام:}}
    \begin{itemize}
        \item \RL{تكامل الأتمتة الكاملة}
        \item \RL{كفاءة قصوى للموارد}
        \item \RL{مراقبة جودة كاملة}
        \item \RL{تحسين السوق}
    \end{itemize}
    \item \textbf{\RL{التكامل المتقدم:}}
    \begin{itemize}
        \item \RL{أنظمة دائرية كاملة}
        \item \RL{تكامل كامل مع الثروة الحيوانية}
        \item \RL{معالجة محسنة}
        \item \RL{تحقيق أقصى قيمة}
    \end{itemize}
    \item \textbf{\RL{مقاييس الأداء:}}
    \begin{itemize}
        \item \RL{تحسين الإنتاجية}
        \item \RL{كفاءة استخدام الموارد}
        \item \RL{معايير الجودة}
        \item \RL{مؤشرات الاستدامة}
    \end{itemize}
\end{itemize}

\subsection{\RL{المقاييس التشغيلية}}
\begin{itemize}
    \item \textbf{\RL{أهداف الإنتاج:}}
    \begin{itemize}
        \item \RL{المرحلة 1: التأسيس}
        \item \RL{المرحلة 2: الإنتاج الأولي}
        \item \RL{المرحلة 3: 30\% من الطاقة}
        \item \RL{المرحلة 4: 60\% من الطاقة}
        \item \RL{المرحلة 5: 90\% من الطاقة}
    \end{itemize}
    \item \textbf{\RL{معايير الجودة:}}
    \begin{itemize}
        \item \RL{مواصفات حجم الثمار}
        \item \RL{مستويات محتوى السكر}
        \item \RL{معايير الرطوبة}
        \item \RL{متانة التخزين}
    \end{itemize}
    \item \textbf{\RL{كفاءة الموارد:}}
    \begin{itemize}
        \item \RL{استخدام المياه لكل كجم}
        \item \RL{كفاءة الطاقة}
        \item \RL{إنتاجية العمل}
        \item \RL{تقليل المخلفات}
    \end{itemize}
\end{itemize}

\subsection{\RL{تكامل سلسلة التوريد مع المشتل}}

\subsubsection{\RL{جدول زمني تفصيلي للتوريد}}
\begin{itemize}
    \item \textbf{\RL{دورة التخطيط السنوية:}}
    \begin{itemize}
        \item \RL{تقديم متطلبات الزراعة المستقبلية لمدة 18 شهرًا إلى وحدة المشتل بحلول أكتوبر}
        \item \RL{استلام جدول تأكيد إنتاج المشتل بحلول ديسمبر}
        \item \RL{إجراء تحديثات وتعديلات التخطيط الفصلية}
        \item \RL{المشاركة في لجنة اختيار الجينات السنوية (مارس)}
    \end{itemize}
    \item \textbf{\RL{جدول الاستلام الموسمي:}}
    \begin{itemize}
        \item \RL{فترة التسليم الرئيسية: فبراير-مارس (فترة الزراعة المثلى)}
        \item \RL{فترة التسليم الثانوية: سبتمبر-أكتوبر (فترة زراعة الخريف)}
        \item \RL{تخصيص استبدال طارئ: الاحتفاظ بمخزون احتياطي بنسبة 10٪ في المشتل}
        \item \RL{الأصناف الخاصة: جدول إكثار مخصص مع مهلة 24 شهرًا}
    \end{itemize}
\end{itemize}

\subsubsection{\RL{التحقق وإدارة الجينات}}
\begin{itemize}
    \item \textbf{\RL{بروتوكولات التحقق:}}
    \begin{itemize}
        \item \RL{البصمة الوراثية لجميع النباتات الأم (بالتنسيق مع المشتل)}
        \item \RL{اختبار PCR عشوائي لـ 2٪ من نباتات زراعة الأنسجة المستلمة}
        \item \RL{التحقق المورفولوجي عند 6 و12 و24 شهرًا بعد الزراعة}
        \item \RL{التحقق السنوي من تطابق النبات الأم والفسائل}
    \end{itemize}
    \item \textbf{\RL{متطلبات التوثيق:}}
    \begin{itemize}
        \item \RL{جواز سفر وراثي كامل لكل دفعة نباتات مستلمة}
        \item \RL{سجل تتبع بلوكشين رقمي من المشتل إلى موقع الحقل}
        \item \RL{قاعدة بيانات تتبع الأداء مرتبطة بسجلات إكثار المشتل}
        \item \RL{وثائق الامتثال للوائح إدارة الموارد الوراثية}
    \end{itemize}
\end{itemize}

\subsubsection{\RL{تكامل مراقبة الجودة}}
\begin{itemize}
    \item \textbf{\RL{معايير القبول:}}
    \begin{itemize}
        \item \RL{معايير الحد الأدنى لتطور الجذور: 5 جذور رئيسية، الحد الأدنى للطول 15 سم}
        \item \RL{متطلبات اختبار مسببات الأمراض: تخليص PCR لـ 5 مسببات أمراض رئيسية}
        \item \RL{مقاييس النمو: الحد الأدنى للارتفاع 30 سم، 5 أوراق وظيفية للأصناف القياسية}
        \item \RL{اختبار الإجهاد: تقييم تحمل الجفاف قبل التسليم (بروتوكول 5 أيام)}
    \end{itemize}
    \item \textbf{\RL{مراقبة الأداء:}}
    \begin{itemize}
        \item \RL{الإبلاغ عن مقاييس البقاء والتأسيس لمدة 30/60/90 يومًا إلى المشتل}
        \item \RL{تقييم أداء النمو في السنة الأولى}
        \item \RL{تحليل الارتباط بين المشتل والحقل لتحسين الإكثار}
        \item \RL{دورة تحسين مدفوعة بالتغذية الراجعة لبروتوكولات الإكثار}
    \end{itemize}
\end{itemize}

\RL{توفر هذه الخطة التشغيلية نهجًا منظمًا لتنفيذ وإدارة وحدة زراعة النخيل، مما يضمن الاستخدام الفعال للموارد وممارسات الإنتاج المستدامة والتكامل السلس مع سلسلة التوريد الوراثية لوحدة المشتل.}
