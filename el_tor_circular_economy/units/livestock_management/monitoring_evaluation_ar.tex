\section{إطار المراقبة والتقييم}

\subsection{مراقبة الإنتاج}
\begin{itemize}
    \item \textbf{أداء الثروة الحيوانية:}
    \begin{itemize}
        \item مراقبة معدل النمو
        \begin{itemize}
            \item سجلات الزيادة اليومية في الوزن
            \item كفاءة تحويل الأعلاف
            \item نسب العمر إلى وزن السوق
            \item معايير خاصة بالسلالة
        \end{itemize}
        
        \item تتبع الحالة الصحية
        \begin{itemize}
            \item تقييمات صحية منتظمة
            \item معدلات حدوث الأمراض
            \item معدلات النفوق
            \item الامتثال للتحصين
        \end{itemize}
        
        \item مقاييس التكاثر
        \begin{itemize}
            \item معدلات نجاح التربية
            \item معدلات بقاء النسل
            \item تتبع التحسين الوراثي
            \item كفاءة دورة التربية
        \end{itemize}
    \end{itemize}
    
    \item \textbf{إدارة الأعلاف:}
    \begin{itemize}
        \item تقييم جودة الأعلاف
        \begin{itemize}
            \item تحليل محتوى المغذيات
            \item مراقبة الطزاجة
            \item ظروف التخزين
            \item تقييم الموردين
        \end{itemize}
        
        \item تتبع الاستهلاك
        \begin{itemize}
            \item سجلات الاستهلاك اليومي
            \item قياس هدر الأعلاف
            \item التكلفة لكل وحدة زيادة
            \item إدارة المخزون
        \end{itemize}
        
        \item إنتاج الأزولا
        \begin{itemize}
            \item قياس معدل النمو
            \item تحليل محتوى المغذيات
            \item كفاءة الإنتاج
            \item معايير الجودة
        \end{itemize}
    \end{itemize}
\end{itemize}

\subsection{المراقبة البيئية}
\begin{itemize}
    \item \textbf{استخدام الموارد:}
    \begin{itemize}
        \item استهلاك المياه
        \begin{itemize}
            \item تتبع الاستخدام اليومي
            \item مقاييس الكفاءة
            \item معايير الجودة
            \item معدلات إعادة التدوير
        \end{itemize}
        
        \item استخدام الطاقة
        \begin{itemize}
            \item أنماط الاستهلاك
            \item إجراءات الكفاءة
            \item استخدام الطاقة المتجددة
            \item تحليل التكلفة
        \end{itemize}
        
        \item كفاءة استخدام الأراضي
        \begin{itemize}
            \item كثافة التخزين
            \item أنماط الرعي
            \item صحة التربة
            \item تأثير التنوع البيولوجي
        \end{itemize}
    \end{itemize}
    
    \item \textbf{إدارة المخلفات:}
    \begin{itemize}
        \item معالجة السماد
        \begin{itemize}
            \item كفاءة الجمع
            \item معدلات المعالجة
            \item تقييم الجودة
            \item تتبع الاستخدام
        \end{itemize}
        
        \item معالجة مياه الصرف
        \begin{itemize}
            \item كفاءة المعالجة
            \item معايير الجودة
            \item معدلات إعادة الاستخدام
            \item مراقبة الامتثال
        \end{itemize}
        
        \item مراقبة الانبعاثات
        \begin{itemize}
            \item مستويات غازات الدفيئة
            \item إدارة الروائح
            \item معايير جودة الهواء
            \item فعالية التخفيف
        \end{itemize}
    \end{itemize}
\end{itemize}

\subsection{التقييم الاقتصادي}
\begin{itemize}
    \item \textbf{الأداء المالي:}
    \begin{itemize}
        \item تتبع الإيرادات
        \begin{itemize}
            \item حجم المبيعات
            \item تحقيق الأسعار
            \item مزيج المنتجات
            \item حصة السوق
        \end{itemize}
        
        \item إدارة التكاليف
        \begin{itemize}
            \item تكاليف التشغيل
            \item نفقات الأعلاف
            \item كفاءة العمالة
            \item تكاليف الصيانة
        \end{itemize}
        
        \item تحليل الربحية
        \begin{itemize}
            \item الهوامش الإجمالية
            \item العائد على الاستثمار
            \item تحليل نقطة التعادل
            \item إدارة التدفق النقدي
        \end{itemize}
    \end{itemize}
    
    \item \textbf{أداء السوق:}
    \begin{itemize}
        \item جودة المنتج
        \begin{itemize}
            \item معايير الجودة
            \item رضا العملاء
            \item معدلات الرفض
            \item الامتثال للشهادات
        \end{itemize}
        
        \item اختراق السوق
        \begin{itemize}
            \item نمو قاعدة العملاء
            \item كفاءة التوزيع
            \item التعرف على العلامة التجارية
            \item الموقف التنافسي
        \end{itemize}
        
        \item كفاءة سلسلة القيمة
        \begin{itemize}
            \item مقاييس سلسلة التوريد
            \item كفاءة التصنيع
            \item تحسين التخزين
            \item تكاليف النقل
        \end{itemize}
    \end{itemize}
\end{itemize}

\subsection{تقييم الأثر الاجتماعي}
\begin{itemize}
    \item \textbf{تأثير التوظيف:}
    \begin{itemize}
        \item خلق فرص العمل
        \begin{itemize}
            \item التوظيف المباشر
            \item التوظيف غير المباشر
            \item تطوير المهارات
            \item مستويات الدخل
        \end{itemize}
        
        \item ظروف العمل
        \begin{itemize}
            \item الامتثال للسلامة
            \item رضا الموظفين
            \item فعالية التدريب
            \item التطور الوظيفي
        \end{itemize}
        
        \item إشراك المجتمع
        \begin{itemize}
            \item المشاركة المحلية
            \item نقل المعرفة
            \item تعليقات المجتمع
            \item الاندماج الاجتماعي
        \end{itemize}
    \end{itemize}
    
    \item \textbf{الأمن الغذائي:}
    \begin{itemize}
        \item توفر المنتج
        \begin{itemize}
            \item الإمداد المحلي
            \item القدرة على تحمل التكاليف
            \item القيمة الغذائية
            \item مقاييس الوصول
        \end{itemize}
        
        \item ضمان الجودة
        \begin{itemize}
            \item معايير السلامة
            \item ممارسات النظافة
            \item إمكانية التتبع
            \item تثقيف المستهلك
        \end{itemize}
        
        \item تأثير المجتمع
        \begin{itemize}
            \item تحسين النظام الغذائي
            \item النتائج الصحية
            \item الاندماج الثقافي
            \item الفوائد الاجتماعية
        \end{itemize}
    \end{itemize}
\end{itemize}

\subsection{طرق التقييم}
\begin{itemize}
    \item \textbf{جمع البيانات:}
    \begin{itemize}
        \item المراقبة المنتظمة
        \begin{itemize}
            \item السجلات اليومية
            \item التقييمات الأسبوعية
            \item التقارير الشهرية
            \item المراجعات السنوية
        \end{itemize}
        
        \item مقاييس الأداء
        \begin{itemize}
            \item المؤشرات الكمية
            \item التقييمات النوعية
            \item مقارنات المعايير
            \item تحليل الاتجاهات
        \end{itemize}
        
        \item آليات التغذية الراجعة
        \begin{itemize}
            \item استطلاعات أصحاب المصلحة
            \item مدخلات المجتمع
            \item تعليقات الموظفين
            \item تقييمات الخبراء
        \end{itemize}
    \end{itemize}
    
    \item \textbf{التحليل وإعداد التقارير:}
    \begin{itemize}
        \item تحليل البيانات
        \begin{itemize}
            \item التحليل الإحصائي
            \item اتجاهات الأداء
            \item تقييم الأثر
            \item الدراسات المقارنة
        \end{itemize}
        
        \item إنشاء التقارير
        \begin{itemize}
            \item التحديثات المنتظمة
            \item لوحات متابعة الأداء
            \item تقارير أصحاب المصلحة
            \item وثائق الامتثال
        \end{itemize}
        
        \item التحسين المستمر
        \begin{itemize}
            \item تخطيط العمل
            \item تتبع التنفيذ
            \item التحقق من النتائج
            \item تحديثات النظام
        \end{itemize}
    \end{itemize}
\end{itemize} 