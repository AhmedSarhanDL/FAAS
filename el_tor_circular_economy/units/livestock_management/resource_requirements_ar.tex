\section{متطلبات الموارد لإدارة الثروة الحيوانية}

\subsection{متطلبات البنية التحتية}
\begin{itemize}
    \item \textbf{إسكان الحيوانات:}
    \begin{itemize}
        \item حظائر الأغنام والماعز: 2.0 متر مربع لكل حيوان
        \item بيوت الدواجن: 0.25 متر مربع لكل طائر
        \item منشآت الأبقار الحلوب: 10 متر مربع لكل بقرة
        \item أنظمة التهوية
        \item أنظمة الإضاءة
        \item أنظمة جمع النفايات
    \end{itemize}
    
    \item \textbf{تخزين الأعلاف:}
    \begin{itemize}
        \item منشأة تخزين التبن: 200 متر مربع
        \item صوامع تخزين الحبوب: سعة 100 طن متري
        \item أحواض زراعة الأزولا: 500 متر مربع
        \item منطقة خلط الأعلاف: 50 متر مربع
    \end{itemize}
    
    \item \textbf{منشآت المعالجة:}
    \begin{itemize}
        \item وحدة معالجة الألبان: 100 متر مربع
        \item جمع وتخزين البيض: 50 متر مربع
        \item منطقة معالجة اللحوم: 150 متر مربع
        \item منشآت التخزين البارد: 100 متر مربع
    \end{itemize}
\end{itemize}

\subsection{متطلبات المعدات}
\begin{itemize}
    \item \textbf{إدارة الأعلاف:}
    \begin{itemize}
        \item خلاطات الأعلاف: وحدتان
        \item معدات توزيع الأعلاف
        \item موازين
        \item حاويات تخزين
        \item معدات فحص الأعلاف
    \end{itemize}
    
    \item \textbf{إدارة الصحة:}
    \begin{itemize}
        \item أدوات ومعدات بيطرية
        \item وحدات تخزين اللقاحات
        \item أجهزة مراقبة الصحة
        \item منشآت العلاج
        \item معدات الحجر الصحي
    \end{itemize}
    
    \item \textbf{معدات الإنتاج:}
    \begin{itemize}
        \item آلات الحلب: 10 وحدات
        \item معدات جمع البيض
        \item أدوات معالجة اللحوم
        \item معدات التعبئة والتغليف
        \item أجهزة فحص الجودة
    \end{itemize}
\end{itemize}

\subsection{الموارد البشرية}
\begin{itemize}
    \item \textbf{طاقم الإدارة:}
    \begin{itemize}
        \item مدير الثروة الحيوانية: 1
        \item مشرفو الإنتاج: 2
        \item مدير مراقبة الجودة: 1
        \item الطاقم الإداري: 2
    \end{itemize}
    
    \item \textbf{الطاقم الفني:}
    \begin{itemize}
        \item طبيب بيطري: 1
        \item متخصصو رعاية الحيوانات: 4
        \item فنيو إدارة الأعلاف: 2
        \item فنيو المعالجة: 3
    \end{itemize}
    
    \item \textbf{طاقم الدعم:}
    \begin{itemize}
        \item عمال عامون: 8
        \item طاقم الصيانة: 2
        \item أفراد الأمن: 2
        \item طاقم التنظيف: 3
    \end{itemize}
\end{itemize}

\subsection{الموارد المستهلكة}
\begin{itemize}
    \item \textbf{موارد الأعلاف:}
    \begin{itemize}
        \item التبن والعلف: 500 طن/سنة
        \item علف الحبوب: 200 طن/سنة
        \item المكملات المعدنية: 10 طن/سنة
        \item إنتاج الأزولا: 100 طن/سنة
    \end{itemize}
    
    \item \textbf{المستلزمات الصحية:}
    \begin{itemize}
        \item اللقاحات والأدوية
        \item مستلزمات التنظيف
        \item المطهرات
        \item مواد الإسعافات الأولية
    \end{itemize}
    
    \item \textbf{مستلزمات الإنتاج:}
    \begin{itemize}
        \item مواد التعبئة والتغليف
        \item مستلزمات المعالجة
        \item حاويات التخزين
        \item مواد مراقبة الجودة
    \end{itemize}
\end{itemize}

\subsection{متطلبات المرافق}
\begin{itemize}
    \item \textbf{موارد المياه:}
    \begin{itemize}
        \item مياه الشرب: 50,000 لتر/يوم
        \item مياه التنظيف: 20,000 لتر/يوم
        \item مياه المعالجة: 10,000 لتر/يوم
        \item ري الأعلاف: 30,000 لتر/يوم
    \end{itemize}
    
    \item \textbf{موارد الطاقة:}
    \begin{itemize}
        \item الكهرباء: 100 كيلوواط ساعة/يوم
        \item وقود التدفئة: 5,000 لتر/شهر
        \item أنظمة الطاقة الشمسية
        \item مولدات احتياطية
    \end{itemize}
    
    \item \textbf{إدارة النفايات:}
    \begin{itemize}
        \item معالجة السماد: 10 طن/يوم
        \item معالجة المياه العادمة
        \item التخلص من النفايات الصلبة
        \item أنظمة إعادة التدوير
    \end{itemize}
\end{itemize}

\subsection{متطلبات التكنولوجيا}
\begin{itemize}
    \item \textbf{أنظمة الإدارة:}
    \begin{itemize}
        \item برنامج إدارة الثروة الحيوانية
        \item نظام تتبع المخزون
        \item نظام الإدارة المالية
        \item برنامج مراقبة الجودة
    \end{itemize}
    
    \item \textbf{معدات المراقبة:}
    \begin{itemize}
        \item أجهزة استشعار بيئية
        \item كاميرات مراقبة
        \item أنظمة تتبع الحيوانات
        \item أجهزة مراقبة الإنتاج
    \end{itemize}
    
    \item \textbf{أنظمة الاتصال:}
    \begin{itemize}
        \item شبكة الاتصال الداخلي
        \item نظام إنذار الطوارئ
        \item الأجهزة المحمولة
        \item اتصال بالإنترنت
    \end{itemize}
\end{itemize}
