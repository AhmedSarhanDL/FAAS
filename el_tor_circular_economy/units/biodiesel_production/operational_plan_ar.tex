\section{الخطة التشغيلية لإنتاج الديزل الحيوي}

\subsection{تصميم وتخطيط المنشأة}

\subsubsection{مناطق الإنتاج}
\begin{itemize}
    \item \textbf{استقبال وتخزين المواد الخام:} مساحة مغطاة 500 متر مربع مع تخزين منفصل لأنواع المواد الخام المختلفة
    \item \textbf{منطقة استخراج الزيت:} 300 متر مربع لمعدات الضغط الميكانيكي والاستخراج بالمذيبات
    \item \textbf{منطقة تكرير الزيت:} 250 متر مربع لعمليات إزالة الصمغ والتحييد والترشيح
    \item \textbf{وحدة الأسترة:} 400 متر مربع لأوعية التفاعل، واسترداد الميثانول، وأنظمة الغسيل
    \item \textbf{تجهيز الديزل الحيوي:} 200 متر مربع للترشيح النهائي، واختبار الجودة، والتخزين
    \item \textbf{منطقة الانحلال الحراري:} 350 متر مربع لمعدات إنتاج الفحم الحيوي وأنظمة التبريد
    \item \textbf{معالجة المنتجات الثانوية:} 200 متر مربع لتنقية الجلسرين ومعالجة الفحم الحيوي
    \item \textbf{مختبر مراقبة الجودة:} 100 متر مربع لمعدات الاختبار وتخزين العينات
    \item \textbf{ورشة الصيانة:} 150 متر مربع لإصلاح المعدات وتخزين قطع الغيار
    \item \textbf{المنطقة الإدارية:} 200 متر مربع للمكاتب، وقاعات الاجتماعات، ومرافق الموظفين
\end{itemize}

\subsubsection{تصميم تدفق المواد}
\begin{itemize}
    \item تدفق عملية خطي مع الحد الأدنى من تراجع المواد
    \item نقل بمساعدة الجاذبية حيثما أمكن لتقليل متطلبات الضخ
    \item أنظمة أنابيب علوية لنقل السوائل بين مناطق العملية
    \item نقل هوائي للكتلة الحيوية الجافة ومواد الفحم الحيوي
    \item أنظمة تنظيف في المكان (CIP) مخصصة لمعدات العملية
    \item أنظمة احتواء الانسكابات في جميع مناطق الإنتاج
\end{itemize}

\subsection{عمليات الإنتاج}

\subsubsection{تحضير المواد الخام}
\begin{itemize}
    \item \textbf{معالجة زيت الطهي المستعمل:}
    \begin{itemize}
        \item الترشيح من خلال مصافي 100 ميكرون لإزالة جزيئات الطعام
        \item التسخين إلى 60 درجة مئوية لفصل الماء
        \item الترسيب لمدة 24 ساعة في خزانات مخروطية
        \item اختبار الأحماض الدهنية الحرة (FFA) والفصل على أساس الجودة
    \end{itemize}
    
    \item \textbf{معالجة الكتلة الحيوية للأزولا:}
    \begin{itemize}
        \item التجفيف إلى محتوى رطوبة 10\% باستخدام المجففات الشمسية مع استعادة الحرارة الاحتياطية
        \item الطحن إلى حجم جزيئات <2 مم باستخدام مطاحن المطرقة
        \item التحبيب للاستخراج الفعال باستخدام المكابس اللولبية
        \item التخزين في صوامع متحكم في مناخها لمنع التدهور
    \end{itemize}
    
    \item \textbf{معالجة المخلفات الزراعية:}
    \begin{itemize}
        \item الفرز لإزالة الملوثات غير العضوية
        \item تقليل الحجم باستخدام القواطع والمطاحن
        \item تعديل محتوى الرطوبة بناءً على الاستخدام المقصود
        \item التخزين المؤقت في مخازن مغطاة مع التهوية
    \end{itemize}
\end{itemize}

\subsubsection{استخراج الزيت والتكرير}
\begin{itemize}
    \item \textbf{الاستخراج الميكانيكي:}
    \begin{itemize}
        \item الضغط البارد باستخدام مكابس لولبية عند ضغط 40-60 بار
        \item تشغيل مستمر مع كفاءة استرداد زيت 70\%
        \item جمع كعكة الضغط للاستخراج الثانوي أو الانحلال الحراري
        \item ترشيح الزيت الخام من خلال مرشحات كيس 20 ميكرون
    \end{itemize}
    
    \item \textbf{الاستخراج بالمذيبات (للأزولا والمخلفات):}
    \begin{itemize}
        \item استخراج بالتيار المعاكس باستخدام مذيبات حيوية
        \item استرداد المذيب من خلال التبخير متعدد المراحل (>98\% استرداد)
        \item إزالة المذيب من الوجبة للمناولة الآمنة
        \item إزالة الصمغ من الزيت المستخرج باستخدام معالجة حمض الفوسفوريك
    \end{itemize}
    
    \item \textbf{تكرير الزيت:}
    \begin{itemize}
        \item إزالة الصمغ باستخدام عمليات المياه والإنزيمات
        \item تحييد الأحماض الدهنية الحرة بمحلول قلوي
        \item الغسيل بالماء الدافئ لإزالة الصابون والمحفزات المتبقية
        \item التجفيف بالتفريغ إلى محتوى رطوبة <0.1\%
        \item الترشيح من خلال مرشحات 5 ميكرون للتوضيح النهائي
    \end{itemize}
\end{itemize}

\subsubsection{إنتاج الديزل الحيوي}
\begin{itemize}
    \item \textbf{عملية الأسترة:}
    \begin{itemize}
        \item نظام مفاعل التدفق المستمر مع وقت بقاء 4 ساعات
        \item ظروف التفاعل: 60 درجة مئوية، ضغط جوي، نسبة ميثانول:زيت 6:1
        \item محفز هيدروكسيد البوتاسيوم بنسبة 1\% من وزن الزيت
        \item تفاعل على مرحلتين مع فصل الجلسرين المتوسط
        \item استرداد الميثانول من خلال التقطير (>99\% استرداد)
    \end{itemize}
    
    \item \textbf{تنقية الديزل الحيوي:}
    \begin{itemize}
        \item فصل الجلسرين من خلال الترسيب بالجاذبية
        \item الغسيل بالماء الدافئ (3 دورات) لإزالة المحفز والصابون
        \item معالجة براتنج التبادل الأيوني للتنقية النهائية
        \item التجفيف بالتفريغ لإزالة الماء المتبقي
        \item إضافة مضادات الأكسدة لاستقرار التخزين
    \end{itemize}
    
    \item \textbf{مراقبة الجودة:}
    \begin{itemize}
        \item مراقبة مباشرة للمعايير الرئيسية (الرقم الهيدروجيني، درجة الحرارة، معدلات التدفق)
        \item أخذ عينات عند نقاط التحكم الحرجة للاختبار المعملي
        \item شهادة الدفعة بناءً على معايير EN 14214 وASTM D6751
        \item نظام تتبع يربط المنتج النهائي بمصادر المواد الخام
    \end{itemize}
\end{itemize}

\subsubsection{إنتاج الفحم الحيوي}
\begin{itemize}
    \item \textbf{نظام الانحلال الحراري:}
    \begin{itemize}
        \item انحلال حراري بطيء عند 450-550 درجة مئوية مع وقت بقاء 1-2 ساعة
        \item بيئة محدودة الأكسجين (<2\% أكسجين)
        \item نظام تغذية لولبي مستمر لإنتاجية متسقة
        \item استعادة حرارة العملية لتجفيف المواد الخام الواردة
        \item التقاط الغاز الحيوي لإنتاج الطاقة الحرارية
    \end{itemize}
    
    \item \textbf{معالجة الفحم الحيوي:}
    \begin{itemize}
        \item تبريد متحكم به في غرف مغلقة
        \item تدريج الحجم من خلال غرابيل اهتزازية (أجزاء 0.5-5 مم)
        \item تعديل الرطوبة إلى 30\% للتحكم في الغبار
        \item إثراء اختياري بالمغذيات للتطبيقات المتخصصة
        \item التعبئة في أكياس كبيرة مقاومة للرطوبة
    \end{itemize}
    
    \item \textbf{مراقبة الكربون:}
    \begin{itemize}
        \item تحليل محتوى الكربون باستخدام طريقة الفقد عند الاحتراق
        \item اختبار الاستقرار باستخدام تقنيات الأكسدة المسرعة
        \item توثيق كفاءة تحويل الكربون
        \item حسابات توازن الكتلة للتحقق من ائتمانات الكربون
    \end{itemize}
\end{itemize}

\subsection{مواصفات المعدات}

\subsubsection{معدات الإنتاج الرئيسية}
\begin{itemize}
    \item \textbf{استخراج الزيت:}
    \begin{itemize}
        \item 2 × مكابس لولبية (سعة 500 كجم/ساعة لكل منها)
        \item 1 × نظام استخراج بالمذيبات (سعة 1,000 كجم/يوم)
        \item 1 × وحدة تقطير لاسترداد المذيبات
        \item 2 × أنظمة ترشيح مع غسيل عكسي آلي
    \end{itemize}
    
    \item \textbf{تكرير الزيت:}
    \begin{itemize}
        \item 2 × مفاعلات إزالة الصمغ (2,000 لتر لكل منها)
        \item 1 × نظام تحييد مع خلط مباشر
        \item 2 × أعمدة غسيل مع تدفق معاكس
        \item 1 × نظام تجفيف بالتفريغ (سعة 500 لتر/ساعة)
    \end{itemize}
    
    \item \textbf{إنتاج الديزل الحيوي:}
    \begin{itemize}
        \item 2 × مفاعلات تدفق مستمر (250 لتر/ساعة لكل منها)
        \item 1 × عمود استرداد الميثانول
        \item 3 × أعمدة غسيل مع إعادة تدوير المياه
        \item 1 × نظام تنقية بالتبادل الأيوني
        \item 1 × وحدة تجفيف بالتفريغ للمنتج النهائي
    \end{itemize}
    
    \item \textbf{إنتاج الفحم الحيوي:}
    \begin{itemize}
        \item 2 × وحدات انحلال حراري (500 كجم/يوم لكل منها)
        \item 1 × نظام تنظيف وتخزين الغاز الحيوي
        \item 1 × شبكة مبادلات حرارية لاستعادة الطاقة
        \item 1 × نظام تبريد ومناولة الفحم الحيوي
        \item 1 × خط معالجة وتعبئة الفحم الحيوي
    \end{itemize}
\end{itemize}

\subsubsection{الأنظمة المساعدة}
\begin{itemize}
    \item \textbf{أنظمة الطاقة:}
    \begin{itemize}
        \item 1 × حارق غاز حيوي لحرارة العملية (500 كيلوواط حراري)
        \item 1 × مولد ديزل حيوي احتياطي (100 كيلوواط)
        \item نظام طاقة شمسية 200 كيلوواط مع تخزين البطاريات
        \item مبادلات حرارية لاستعادة الحرارة في جميع أنحاء العملية
    \end{itemize}
    
    \item \textbf{إدارة المياه:}
    \begin{itemize}
        \item نظام إعادة تدوير المياه مغلق الدورة (95\% استرداد)
        \item معالجة مياه الصرف باستخدام مفاعل حيوي غشائي
        \item نظام تجميع مياه الأمطار لمياه العملية
        \item نظام مراقبة وتحكم في جودة المياه
    \end{itemize}
    
    \item \textbf{التحكم في جودة الهواء:}
    \begin{itemize}
        \item مؤكسد حراري لتدمير المركبات العضوية المتطايرة
        \item أنظمة جمع الغبار لمناطق مناولة المواد الصلبة
        \item مرشحات كربون للتحكم في الروائح
        \item نظام مراقبة مستمرة للانبعاثات
    \end{itemize}
\end{itemize}

\subsection{إجراءات التشغيل}

\subsubsection{العمليات اليومية}
\begin{itemize}
    \item \textbf{إجراءات بدء التشغيل:}
    \begin{itemize}
        \item فحوصات سلامة النظام والتحقق من الأمان
        \item بدء تشغيل متسلسل لوحدات العملية وفقًا للبروتوكولات القياسية
        \item فترات التسخين للمفاعلات والمبادلات الحرارية
        \item فحوصات المعايرة للأجهزة الحرجة
    \end{itemize}
    
    \item \textbf{العمليات الروتينية:}
    \begin{itemize}
        \item مراقبة مستمرة لمعايير العملية
        \item أخذ عينات منتظمة واختبار الجودة
        \item تعديل ظروف العملية بناءً على تغيرات المواد الخام
        \item تنسيق حركة المواد بين مناطق العملية
        \item توثيق بيانات الإنتاج ونتائج الجودة
    \end{itemize}
    
    \item \textbf{إجراءات الإيقاف:}
    \begin{itemize}
        \item إيقاف متحكم به ومتسلسل لوحدات العملية
        \item تنظيف وغسل المعدات الحرجة
        \item تخزين آمن للمواد قيد المعالجة
        \item قفل النظام لأنشطة الصيانة
        \item توثيق حالة التشغيل
    \end{itemize}
\end{itemize}

\subsubsection{جدول الصيانة}
\begin{itemize}
    \item \textbf{الصيانة اليومية:}
    \begin{itemize}
        \item فحوصات بصرية لجميع المعدات
        \item تنظيف المرشحات والمصافي
        \item فحوصات التشحيم للمعدات الدوارة
        \item التحقق من معايرة الأجهزة الحرجة
    \end{itemize}
    
    \item \textbf{الصيانة الأسبوعية:}
    \begin{itemize}
        \item اختبار أداء المضخات والمحركات
        \item تنظيف المبادلات الحرارية
        \item فحص السدادات والحشيات
        \item اختبار أنظمة السلامة والإنذارات
    \end{itemize}
    
    \item \textbf{الصيانة الشهرية:}
    \begin{itemize}
        \item فحص شامل للمعدات
        \item استبدال قطع التآكل حسب الحاجة
        \item معايرة جميع أجهزة القياس
        \item فحص العناصر الهيكلية
    \end{itemize}
    
    \item \textbf{الصيانة السنوية:}
    \begin{itemize}
        \item إيقاف كامل للمصنع للفحص الشامل
        \item إصلاح شامل للمعدات الرئيسية
        \item اختبار الضغط للأوعية والأنابيب
        \item فحص وإصلاح المواد الحرارية في وحدات الانحلال الحراري
        \item تجديد شهادات معدات الضغط
    \end{itemize}
\end{itemize}

\subsection{نظام إدارة الجودة}

\subsubsection{معايير مراقبة الجودة}
\begin{itemize}
    \item \textbf{جودة المواد الخام:}
    \begin{itemize}
        \item محتوى الأحماض الدهنية الحرة (<5\% للمعالجة الفعالة)
        \item محتوى الرطوبة (<0.5\% للزيوت المكررة)
        \item مستويات الشوائب (<0.1\% للزيوت المكررة)
        \item محتوى الفوسفور (<10 جزء في المليون للزيوت المكررة)
    \end{itemize}
    
    \item \textbf{جودة الديزل الحيوي (EN 14214 / ASTM D6751):}
    \begin{itemize}
        \item محتوى الإستر (>96.5\%)
        \item الكثافة (860-900 كجم/متر³)
        \item اللزوجة (3.5-5.0 مم²/ثانية)
        \item نقطة الوميض (>101 درجة مئوية)
        \item محتوى الكبريت (<10 مجم/كجم)
        \item متبقي الكربون (<0.3\%)
        \item رقم السيتان (>51)
        \item ثبات الأكسدة (>8 ساعات)
        \item قيمة الحموضة (<0.5 مجم هيدروكسيد البوتاسيوم/جم)
        \item محتوى الميثانول (<0.2\%)
        \item محتوى الماء (<500 مجم/كجم)
    \end{itemize}
    
    \item \textbf{جودة الفحم الحيوي:}
    \begin{itemize}
        \item محتوى الكربون (>70\%)
        \item نسبة الهيدروجين:الكربون (<0.7 للاستقرار)
        \item المساحة السطحية (>300 متر²/جم)
        \item الرقم الهيدروجيني (6.5-9.5 حسب التطبيق)
        \item محتوى الرماد (<10\%)
        \item محتوى المعادن الثقيلة (أقل من الحدود التنظيمية)
        \item محتوى الهيدروكربونات العطرية متعددة الحلقات (<4 مجم/كجم)
    \end{itemize}
\end{itemize}

\subsubsection{إجراءات الاختبار}
\begin{itemize}
    \item \textbf{اختبارات أثناء العملية:}
    \begin{itemize}
        \item اختبار سريع للأحماض الدهنية الحرة باستخدام طرق المعايرة
        \item تحليل الرطوبة باستخدام معايرة كارل فيشر
        \item مراقبة التحويل باستخدام كروماتوغرافيا الطبقة الرقيقة
        \item قياس محتوى الميثانول باستخدام كروماتوغرافيا الغاز للحيز العلوي
        \item مراقبة الرقم الهيدروجيني عند نقاط العملية الحرجة
    \end{itemize}
    
    \item \textbf{اختبار المنتج النهائي:}
    \begin{itemize}
        \item اختبار شامل وفقًا لمعايير EN 14214 / ASTM D6751
        \item اختبار الثبات باستخدام طريقة رانسيمات
        \item اختبار خصائص التدفق البارد (نقطة تصفية البارافين، نقطة التعكر)
        \item اختبار التلوث الميكروبي
        \item مراقبة ثبات التخزين
    \end{itemize}
    
    \item \textbf{اختبار الفحم الحيوي:}
    \begin{itemize}
        \item قياس محتوى الكربون باستخدام التحليل العنصري
        \item قياس المساحة السطحية باستخدام طريقة BET
        \item اختبار الرقم الهيدروجيني والتوصيل الكهربائي
        \item تحليل المعادن الثقيلة باستخدام ICP-MS
        \item اختبار الهيدروكربونات العطرية متعددة الحلقات باستخدام GC-MS
    \end{itemize}
\end{itemize}

\subsection{مراقبة ائتمانات الكربون}

\subsubsection{أنظمة القياس}
\begin{itemize}
    \item \textbf{حساب كربون الفحم الحيوي:}
    \begin{itemize}
        \item قياس محتوى الكربون في المواد الخام
        \item مراقبة كفاءة تحويل الكربون
        \item تحديد كمية الكربون المستقر في الفحم الحيوي
        \item توثيق تطبيق وتخزين الفحم الحيوي
        \item التحقق من الاستقرار طويل المدى
    \end{itemize}
    
    \item \textbf{حساب خفض الانبعاثات:}
    \begin{itemize}
        \item حسابات خط الأساس لاستبدال الوقود الأحفوري
        \item مراقبة إنتاج واستخدام الديزل الحيوي
        \item تقييم دورة الحياة لعمليات الإنتاج
        \item تحديد كمية صافي خفض الانبعاثات
        \item تحقق طرف ثالث من الحسابات
    \end{itemize}
    
    \item \textbf{مراقبة كفاءة العملية:}
    \begin{itemize}
        \item تتبع استهلاك الطاقة لكل وحدة إنتاج
        \item توليد واستخدام الطاقة المتجددة
        \item تحسين العمليات لخفض الانبعاثات
        \item توثيق تحسينات الكفاءة
        \item مقارنة مع معايير الصناعة
    \end{itemize}
\end{itemize}

\subsubsection{التقارير والتحقق}
\begin{itemize}
    \item \textbf{إدارة البيانات:}
    \begin{itemize}
        \item جمع آلي للبيانات من أنظمة التحكم في العملية
        \item قاعدة بيانات آمنة لجميع قياسات الكربون
        \item تدقيق داخلي منتظم لجودة البيانات
        \item توثيق سلسلة الحيازة لجميع المنتجات
        \item منهجيات حساب شفافة
    \end{itemize}
    
    \item \textbf{إجراءات التحقق:}
    \begin{itemize}
        \item الامتثال لمنهجيات ائتمان الكربون الدولية
        \item عمليات تدقيق تحقق منتظمة من طرف ثالث
        \item تحليل عدم اليقين لجميع القياسات
        \item مبادئ التقدير المحافظ
        \item تحسين مستمر لأنظمة المراقبة
    \end{itemize}
    
    \item \textbf{جدول التقارير:}
    \begin{itemize}
        \item تقارير أداء الكربون الداخلية الشهرية
        \item تحقق ربع سنوي من توليد ائتمانات الكربون
        \item تدقيق كربون شامل سنوي
        \item تقارير لسجلات الكربون ذات الصلة
        \item إفصاح عام عن أداء الكربون
    \end{itemize}
\end{itemize}

\subsection{التوظيف والتدريب}

\subsubsection{الهيكل التنظيمي}
\begin{itemize}
    \item \textbf{فريق الإدارة:}
    \begin{itemize}
        \item مدير المصنع (1)
        \item مشرف الإنتاج (1)
        \item مدير مراقبة الجودة (1)
        \item مشرف الصيانة (1)
        \item مسؤول الإدارة والمالية (1)
    \end{itemize}
    
    \item \textbf{الطاقم التقني:}
    \begin{itemize}
        \item مهندسو العمليات (2)
        \item فنيو المختبر (2)
        \item فنيو الصيانة (3)
        \item متخصص الأجهزة (1)
        \item مسؤول الامتثال البيئي (1)
    \end{itemize}
    
    \item \textbf{طاقم التشغيل:}
    \begin{itemize}
        \item مشغلو إنتاج الديزل الحيوي (4)
        \item مشغلو إنتاج الفحم الحيوي (2)
        \item مشغلو تحضير المواد الخام (3)
        \item مشغلو مناولة المواد (2)
        \item مشغلو أنظمة المرافق (2)
    \end{itemize}
\end{itemize}

\subsubsection{برنامج التدريب}
\begin{itemize}
    \item \textbf{التدريب الأولي:}
    \begin{itemize}
        \item أساسيات العملية والكيمياء (40 ساعة)
        \item تشغيل المعدات وحل المشكلات (80 ساعة)
        \item إجراءات السلامة والطوارئ (24 ساعة)
        \item مراقبة الجودة وطرق الاختبار (40 ساعة)
        \item أنظمة الإدارة البيئية (16 ساعة)
    \end{itemize}
    
    \item \textbf{التدريب المستمر:}
    \begin{itemize}
        \item تدريب تنشيطي شهري للسلامة (4 ساعات)
        \item تطوير المهارات التقنية ربع السنوي (8 ساعات)
        \item تجديد الشهادات السنوي للأدوار المتخصصة
        \item برنامج التدريب المتبادل للمرونة التشغيلية
        \item فرص التدريب الخارجي للمهارات المتقدمة
    \end{itemize}
    
    \item \textbf{إدارة المعرفة:}
    \begin{itemize}
        \item توثيق شامل لإجراءات التشغيل
        \item نظام إدارة التعلم الإلكتروني
        \item تتبع مصفوفة المهارات لجميع العاملين
        \item برنامج التوجيه لنقل المعرفة
        \item جلسات منتظمة لتبادل المعرفة
    \end{itemize}
\end{itemize}

\subsection{إدارة السلامة والبيئة}

\subsubsection{أنظمة السلامة}
\begin{itemize}
    \item \textbf{سلامة العمليات:}
    \begin{itemize}
        \item تحليل المخاطر وقابلية التشغيل (HAZOP) لجميع العمليات
        \item أقفال السلامة الآلية على المعدات الحرجة
        \item أنظمة جمع الغبار لمناطق مناولة المواد الصلبة
        \item مراقبة مستمرة للانبعاثات للملوثات المنظمة
        \item مرشحات حيوية للتحكم في الروائح
        \item اختبار وتقارير منتظمة للمداخن
    \end{itemize}
    
    \item \textbf{سلامة العاملين:}
    \begin{itemize}
        \item متطلبات معدات الحماية الشخصية لجميع المناطق
        \item محطات دش السلامة وغسل العين في جميع أنحاء المنشأة
        \item إجراءات ومعدات الدخول إلى الأماكن المحصورة
        \item نظام قفل/وضع العلامات لأنشطة الصيانة
        \item تدريبات سلامة منتظمة وتدريب على الاستجابة للطوارئ
    \end{itemize}
    
    \item \textbf{الحماية من الحرائق:}
    \begin{itemize}
        \item أنظمة كشف وإخماد الحرائق الآلية
        \item أنظمة رغوة لمناطق السوائل القابلة للاشتعال
        \item حلقة مياه إطفاء مع قدرة ضخ احتياطية
        \item معدات الاستجابة للطوارئ وفريق مدرب
        \item فحص واختبار منتظم لجميع أنظمة الحريق
    \end{itemize}
\end{itemize}

\subsubsection{الضوابط البيئية}
\begin{itemize}
    \item \textbf{انبعاثات الهواء:}
    \begin{itemize}
        \item مؤكسد حراري لتدمير المركبات العضوية المتطايرة (كفاءة >99\%)
        \item أنظمة جمع الغبار مع ترشيح HEPA
        \item مراقبة مستمرة للانبعاثات للملوثات المنظمة
        \item مرشحات حيوية للتحكم في الروائح
        \item اختبار وتقارير منتظمة للمداخن
    \end{itemize}
    
    \item \textbf{إدارة المياه:}
    \begin{itemize}
        \item نظام تصريف سائل صفري
        \item مفاعل حيوي غشائي لمعالجة مياه العملية
        \item نظام إدارة مياه الأمطار مع احتواء الدفق الأول
        \item احتواء الانسكابات في جميع مناطق مناولة المواد الكيميائية
        \item مراقبة منتظمة لجودة المياه
    \end{itemize}
    
    \item \textbf{إدارة النفايات:}
    \begin{itemize}
        \item برنامج شامل لفصل النفايات
        \item إعادة تدوير جميع المواد المتوافقة
        \item تحويل النفايات العضوية إلى فحم حيوي
        \item إجراءات التخلص السليم من النفايات الخطرة
        \item أهداف وتتبع تقليل النفايات
    \end{itemize}
\end{itemize}
