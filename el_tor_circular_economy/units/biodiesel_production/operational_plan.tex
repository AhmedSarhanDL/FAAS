\section{Biodiesel Production Operational Plan}

\subsection{Facility Design and Layout}

\subsubsection{Production Areas}
\begin{itemize}
    \item \textbf{Feedstock Reception and Storage:} 500 m² covered area with segregated storage for different feedstock types
    \item \textbf{Oil Extraction Zone:} 300 m² for mechanical pressing and solvent extraction equipment
    \item \textbf{Oil Refining Area:} 250 m² for degumming, neutralization, and filtration processes
    \item \textbf{Transesterification Unit:} 400 m² for reaction vessels, methanol recovery, and washing systems
    \item \textbf{Biodiesel Finishing:} 200 m² for final filtration, quality testing, and storage
    \item \textbf{Pyrolysis Zone:} 350 m² for biochar production equipment and cooling systems
    \item \textbf{By-product Processing:} 200 m² for glycerin purification and biochar post-processing
    \item \textbf{Quality Control Laboratory:} 100 m² for testing equipment and sample storage
    \item \textbf{Maintenance Workshop:} 150 m² for equipment repair and spare parts storage
    \item \textbf{Administrative Area:} 200 m² for offices, meeting rooms, and staff facilities
\end{itemize}

\subsubsection{Material Flow Design}
\begin{itemize}
    \item Linear process flow with minimal backtracking of materials
    \item Gravity-assisted transfer where possible to reduce pumping requirements
    \item Overhead piping systems for liquid transfers between process areas
    \item Pneumatic conveying for dry biomass and biochar materials
    \item Dedicated clean-in-place (CIP) systems for process equipment
    \item Spill containment systems throughout production areas
\end{itemize}

\subsection{Production Processes}

\subsubsection{Feedstock Preparation}
\begin{itemize}
    \item \textbf{Waste Cooking Oil Processing:}
    \begin{itemize}
        \item Filtration through 100-micron screens to remove food particles
        \item Heating to 60°C for water separation
        \item Settling for 24 hours in conical tanks
        \item Free fatty acid (FFA) testing and segregation based on quality
    \end{itemize}
    
    \item \textbf{Azolla Biomass Processing:}
    \begin{itemize}
        \item Drying to 10\% moisture content using solar dryers with backup heat recovery
        \item Grinding to <2mm particle size using hammer mills
        \item Pelletizing for efficient extraction using screw presses
        \item Storage in climate-controlled silos to prevent degradation
    \end{itemize}
    
    \item \textbf{Agricultural Residue Processing:}
    \begin{itemize}
        \item Sorting to remove non-organic contaminants
        \item Size reduction using chippers and grinders
        \item Moisture content adjustment based on intended use
        \item Temporary storage in covered bunkers with aeration
    \end{itemize}
\end{itemize}

\subsubsection{Oil Extraction and Refining}
\begin{itemize}
    \item \textbf{Mechanical Extraction:}
    \begin{itemize}
        \item Cold pressing using screw presses at 40-60 bar pressure
        \item Continuous operation with 70\% oil recovery efficiency
        \item Press cake collection for secondary extraction or pyrolysis
        \item Crude oil filtration through 20-micron bag filters
    \end{itemize}
    
    \item \textbf{Solvent Extraction (for Azolla and residues):}
    \begin{itemize}
        \item Countercurrent extraction using bio-based solvents
        \item Solvent recovery through multi-stage evaporation (>98\% recovery)
        \item Desolventizing of meal for safe handling
        \item Extracted oil degumming using phosphoric acid treatment
    \end{itemize}
    
    \item \textbf{Oil Refining:}
    \begin{itemize}
        \item Degumming using water and enzymatic processes
        \item Neutralization of free fatty acids with alkali solution
        \item Washing with warm water to remove soaps and residual catalysts
        \item Vacuum drying to <0.1\% moisture content
        \item Filtration through 5-micron filters for final clarification
    \end{itemize}
\end{itemize}

\subsubsection{Biodiesel Production}
\begin{itemize}
    \item \textbf{Transesterification Process:}
    \begin{itemize}
        \item Continuous-flow reactor system with 4-hour residence time
        \item Reaction conditions: 60°C, atmospheric pressure, 6:1 methanol:oil ratio
        \item Potassium hydroxide catalyst at 1\% of oil weight
        \item Two-stage reaction with intermediate glycerin separation
        \item Methanol recovery through distillation (>99\% recovery)
    \end{itemize}
    
    \item \textbf{Biodiesel Purification:}
    \begin{itemize}
        \item Glycerin separation through gravity settling
        \item Warm water washing (3 cycles) to remove catalyst and soaps
        \item Ion exchange resin treatment for final purification
        \item Vacuum drying to remove residual water
        \item Antioxidant addition for storage stability
    \end{itemize}
    
    \item \textbf{Quality Control:}
    \begin{itemize}
        \item Inline monitoring of key parameters (pH, temperature, flow rates)
        \item Sampling at critical control points for laboratory testing
        \item Batch certification based on EN 14214 and ASTM D6751 standards
        \item Traceability system linking finished product to feedstock sources
    \end{itemize}
\end{itemize}

\subsubsection{Biochar Production}
\begin{itemize}
    \item \textbf{Pyrolysis System:}
    \begin{itemize}
        \item Slow pyrolysis at 450-550°C with 1-2 hour residence time
        \item Oxygen-limited environment (<2\% O\textsubscript{2})
        \item Continuous feed auger system for consistent throughput
        \item Process heat recovery for drying incoming feedstock
        \item Syngas capture for thermal energy production
    \end{itemize}
    
    \item \textbf{Biochar Processing:}
    \begin{itemize}
        \item Controlled cooling in sealed chambers
        \item Size grading through vibrating screens (0.5-5mm fractions)
        \item Moisture adjustment to 30\% for dust control
        \item Optional nutrient enrichment for specialized applications
        \item Packaging in moisture-resistant bulk bags
    \end{itemize}
    
    \item \textbf{Carbon Monitoring:}
    \begin{itemize}
        \item Carbon content analysis using loss-on-ignition method
        \item Stability testing using accelerated oxidation techniques
        \item Documentation of carbon conversion efficiency
        \item Mass balance calculations for carbon credit verification
    \end{itemize}
\end{itemize}

\subsection{Equipment Specifications}

\subsubsection{Major Production Equipment}
\begin{itemize}
    \item \textbf{Oil Extraction:}
    \begin{itemize}
        \item 2 × Screw presses (500 kg/hr capacity each)
        \item 1 × Solvent extraction system (1,000 kg/day capacity)
        \item 1 × Solvent recovery distillation unit
        \item 2 × Filtration systems with automated backwashing
    \end{itemize}
    
    \item \textbf{Oil Refining:}
    \begin{itemize}
        \item 2 × Degumming reactors (2,000 L each)
        \item 1 × Neutralization system with inline mixing
        \item 2 × Washing columns with counter-current flow
        \item 1 × Vacuum drying system (500 L/hr capacity)
    \end{itemize}
    
    \item \textbf{Biodiesel Production:}
    \begin{itemize}
        \item 2 × Continuous flow reactors (250 L/hr each)
        \item 1 × Methanol recovery column
        \item 3 × Washing columns with water recycling
        \item 1 × Ion exchange purification system
        \item 1 × Vacuum drying unit for final product
    \end{itemize}
    
    \item \textbf{Biochar Production:}
    \begin{itemize}
        \item 2 × Pyrolysis units (500 kg/day each)
        \item 1 × Syngas cleaning and storage system
        \item 1 × Heat exchanger network for energy recovery
        \item 1 × Biochar cooling and handling system
        \item 1 × Biochar processing and packaging line
    \end{itemize}
\end{itemize}

\subsubsection{Auxiliary Systems}
\begin{itemize}
    \item \textbf{Energy Systems:}
    \begin{itemize}
        \item 1 × Syngas burner for process heat (500 kW thermal)
        \item 1 × Backup biodiesel generator (100 kW)
        \item 200 kW solar PV system with battery storage
        \item Heat recovery exchangers throughout process
    \end{itemize}
    
    \item \textbf{Water Management:}
    \begin{itemize}
        \item Closed-loop water recycling system (95\% recovery)
        \item Wastewater treatment using membrane bioreactor
        \item Rainwater harvesting system for process water
        \item Water quality monitoring and control system
    \end{itemize}
    
    \item \textbf{Air Quality Control:}
    \begin{itemize}
        \item Thermal oxidizer for VOC destruction
        \item Dust collection systems for solid handling areas
        \item Carbon filters for odor control
        \item Continuous emissions monitoring system
    \end{itemize}
\end{itemize}

\subsection{Operational Procedures}

\subsubsection{Daily Operations}
\begin{itemize}
    \item \textbf{Start-up Procedures:}
    \begin{itemize}
        \item System integrity checks and safety verification
        \item Sequential start-up of process units following standard protocols
        \item Warm-up periods for reactors and heat exchangers
        \item Calibration checks for critical instrumentation
    \end{itemize}
    
    \item \textbf{Routine Operations:}
    \begin{itemize}
        \item Continuous monitoring of process parameters
        \item Regular sampling and quality testing
        \item Adjustment of process conditions based on feedstock variations
        \item Coordination of material movements between process areas
        \item Documentation of production data and quality results
    \end{itemize}
    
    \item \textbf{Shutdown Procedures:}
    \begin{itemize}
        \item Controlled sequential shutdown of process units
        \item Flushing and cleaning of critical equipment
        \item Secure storage of in-process materials
        \item System lockout for maintenance activities
        \item Documentation of operational status
    \end{itemize}
\end{itemize}

\subsubsection{Maintenance Schedule}
\begin{itemize}
    \item \textbf{Daily Maintenance:}
    \begin{itemize}
        \item Visual inspections of all equipment
        \item Cleaning of filters and strainers
        \item Lubrication checks on rotating equipment
        \item Calibration verification for critical instruments
    \end{itemize}
    
    \item \textbf{Weekly Maintenance:}
    \begin{itemize}
        \item Pump and motor performance testing
        \item Cleaning of heat exchangers
        \item Inspection of seals and gaskets
        \item Testing of safety systems and alarms
    \end{itemize}
    
    \item \textbf{Monthly Maintenance:}
    \begin{itemize}
        \item Comprehensive equipment inspection
        \item Replacement of wear parts as needed
        \item Calibration of all instrumentation
        \item Inspection of structural elements
    \end{itemize}
    
    \item \textbf{Annual Maintenance:}
    \begin{itemize}
        \item Complete plant shutdown for thorough inspection
        \item Overhaul of major equipment
        \item Pressure testing of vessels and piping
        \item Refractory inspection and repair in pyrolysis units
        \item Certification renewal for pressure equipment
    \end{itemize}
\end{itemize}

\subsection{Quality Management System}

\subsubsection{Quality Control Parameters}
\begin{itemize}
    \item \textbf{Feedstock Quality:}
    \begin{itemize}
        \item Free fatty acid content (<5\% for efficient processing)
        \item Moisture content (<0.5\% for refined oils)
        \item Impurity levels (<0.1\% for refined oils)
        \item Phosphorus content (<10 ppm for refined oils)
    \end{itemize}
    
    \item \textbf{Biodiesel Quality (EN 14214 / ASTM D6751):}
    \begin{itemize}
        \item Ester content (>96.5\%)
        \item Density (860-900 kg/m³)
        \item Viscosity (3.5-5.0 mm²/s)
        \item Flash point (>101°C)
        \item Sulfur content (<10 mg/kg)
        \item Carbon residue (<0.3\%)
        \item Cetane number (>51)
        \item Oxidation stability (>8 hours)
        \item Acid value (<0.5 mg KOH/g)
        \item Methanol content (<0.2\%)
        \item Water content (<500 mg/kg)
    \end{itemize}
    
    \item \textbf{Biochar Quality:}
    \begin{itemize}
        \item Carbon content (>70\%)
        \item H:C ratio (<0.7 for stability)
        \item Surface area (>300 m²/g)
        \item pH (6.5-9.5 depending on application)
        \item Ash content (<10\%)
        \item Heavy metal content (below regulatory limits)
        \item PAH content (<4 mg/kg)
    \end{itemize}
\end{itemize}

\subsubsection{Testing Procedures}
\begin{itemize}
    \item \textbf{In-process Testing:}
    \begin{itemize}
        \item Rapid FFA testing using titration methods
        \item Moisture analysis using Karl Fischer titration
        \item Conversion monitoring using thin-layer chromatography
        \item Methanol content using headspace gas chromatography
        \item pH monitoring at critical process points
    \end{itemize}
    
    \item \textbf{Final Product Testing:}
    \begin{itemize}
        \item Comprehensive testing according to EN 14214 / ASTM D6751
        \item Stability testing using Rancimat method
        \item Cold flow properties testing (CFPP, cloud point)
        \item Microbial contamination testing
        \item Storage stability monitoring
    \end{itemize}
    
    \item \textbf{Biochar Testing:}
    \begin{itemize}
        \item Carbon content using elemental analysis
        \item Surface area measurement using BET method
        \item pH and electrical conductivity testing
        \item Heavy metal analysis using ICP-MS
        \item PAH testing using GC-MS
    \end{itemize}
\end{itemize}

\subsection{Staffing and Training}

\subsubsection{Organizational Structure}
\begin{itemize}
    \item \textbf{Management Team:}
    \begin{itemize}
        \item Plant Manager (1)
        \item Production Supervisor (1)
        \item Quality Control Manager (1)
        \item Maintenance Supervisor (1)
        \item Administration and Finance Officer (1)
    \end{itemize}
    
    \item \textbf{Technical Staff:}
    \begin{itemize}
        \item Process Engineers (2)
        \item Laboratory Technicians (2)
        \item Maintenance Technicians (3)
        \item Instrumentation Specialist (1)
        \item Environmental Compliance Officer (1)
    \end{itemize}
    
    \item \textbf{Operations Staff:}
    \begin{itemize}
        \item Biodiesel Production Operators (4)
        \item Biochar Production Operators (2)
        \item Feedstock Preparation Operators (3)
        \item Material Handling Operators (2)
        \item Utility Systems Operators (2)
    \end{itemize}
\end{itemize}

\subsubsection{Training Program}
\begin{itemize}
    \item \textbf{Initial Training:}
    \begin{itemize}
        \item Process fundamentals and chemistry (40 hours)
        \item Equipment operation and troubleshooting (80 hours)
        \item Safety and emergency procedures (24 hours)
        \item Quality control and testing methods (40 hours)
        \item Environmental management systems (16 hours)
    \end{itemize}
    
    \item \textbf{Ongoing Training:}
    \begin{itemize}
        \item Monthly safety refresher training (4 hours)
        \item Quarterly technical skills development (8 hours)
        \item Annual certification renewal for specialized roles
        \item Cross-training program for operational flexibility
        \item External training opportunities for advanced skills
    \end{itemize}
    
    \item \textbf{Knowledge Management:}
    \begin{itemize}
        \item Comprehensive operating procedures documentation
        \item Electronic learning management system
        \item Skills matrix tracking for all personnel
        \item Mentoring program for knowledge transfer
        \item Regular knowledge-sharing sessions
    \end{itemize}
\end{itemize}

\subsection{Safety and Environmental Management}

\subsubsection{Safety Systems}
\begin{itemize}
    \item \textbf{Process Safety:}
    \begin{itemize}
        \item Hazard and operability (HAZOP) analysis for all processes
        \item Automated safety interlocks on critical equipment
        \item Explosion-proof electrical systems in hazardous areas
        \item Pressure relief systems on all pressure vessels
        \item Emergency shutdown systems with multiple activation points
    \end{itemize}
    
    \item \textbf{Personnel Safety:}
    \begin{itemize}
        \item Personal protective equipment requirements for all areas
        \item Safety shower and eyewash stations throughout facility
        \item Confined space entry procedures and equipment
        \item Lock-out/tag-out system for maintenance activities
        \item Regular safety drills and emergency response training
    \end{itemize}
    
    \item \textbf{Fire Protection:}
    \begin{itemize}
        \item Automatic fire detection and suppression systems
        \item Foam systems for flammable liquid areas
        \item Fire water loop with redundant pumping capacity
        \item Emergency response equipment and trained team
        \item Regular inspection and testing of all fire systems
    \end{itemize}
\end{itemize}

\subsubsection{Environmental Controls}
\begin{itemize}
    \item \textbf{Air Emissions:}
    \begin{itemize}
        \item Thermal oxidizer for VOC destruction (>99\% efficiency)
        \item Dust collection systems with HEPA filtration
        \item Continuous emissions monitoring for regulated pollutants
        \item Biofilters for odor control
        \item Regular stack testing and reporting
    \end{itemize}
    
    \item \textbf{Water Management:}
    \begin{itemize}
        \item Zero liquid discharge system
        \item Membrane bioreactor for process water treatment
        \item Stormwater management system with first-flush containment
        \item Spill containment throughout chemical handling areas
        \item Regular water quality monitoring
    \end{itemize}
    
    \item \textbf{Waste Management:}
    \begin{itemize}
        \item Comprehensive waste segregation program
        \item Recycling of all compatible materials
        \item Conversion of organic wastes to biochar
        \item Proper disposal procedures for hazardous wastes
        \item Waste reduction targets and tracking
    \end{itemize}
\end{itemize}

\subsection{Carbon Credit Monitoring}

\subsubsection{Measurement Systems}
\begin{itemize}
    \item \textbf{Biochar Carbon Accounting:}
    \begin{itemize}
        \item Measurement of feedstock carbon content
        \item Monitoring of carbon conversion efficiency
        \item Quantification of stable carbon in biochar
        \item Documentation of biochar application and storage
        \item Long-term stability verification
    \end{itemize}
    
    \item \textbf{Emissions Reduction Accounting:}
    \begin{itemize}
        \item Baseline fossil fuel displacement calculations
        \item Monitoring of biodiesel production and use
        \item Life cycle assessment of production processes
        \item Quantification of net emissions reduction
        \item Third-party verification of calculations
    \end{itemize}
    
    \item \textbf{Process Efficiency Monitoring:}
    \begin{itemize}
        \item Energy consumption tracking per unit of production
        \item Renewable energy generation and utilization
        \item Process optimization for emissions reduction
        \item Documentation of efficiency improvements
        \item Comparison against industry benchmarks
    \end{itemize}
\end{itemize}

\subsubsection{Reporting and Verification}
\begin{itemize}
    \item \textbf{Data Management:}
    \begin{itemize}
        \item Automated data collection from process control systems
        \item Secure database for all carbon-related measurements
        \item Regular internal audits of data quality
        \item Chain of custody documentation for all products
        \item Transparent calculation methodologies
    \end{itemize}
    
    \item \textbf{Verification Procedures:}
    \begin{itemize}
        \item Compliance with international carbon credit methodologies
        \item Regular third-party verification audits
        \item Uncertainty analysis for all measurements
        \item Conservative estimation principles
        \item Continuous improvement of monitoring systems
    \end{itemize}
    
    \item \textbf{Reporting Schedule:}
    \begin{itemize}
        \item Monthly internal carbon performance reports
        \item Quarterly verification of carbon credit generation
        \item Annual comprehensive carbon audit
        \item Reporting to relevant carbon registries
        \item Public disclosure of carbon performance
    \end{itemize}
\end{itemize}
