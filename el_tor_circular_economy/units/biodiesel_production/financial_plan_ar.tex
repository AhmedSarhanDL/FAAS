\section{الخطة المالية لوحدة إنتاج الديزل الحيوي}

\subsection{متطلبات الاستثمار}

\subsubsection{النفقات الرأسمالية}
\begin{itemize}
    \item \textbf{الأرض وتطوير الموقع:}
    \begin{itemize}
        \item شراء الأرض (2,500 متر مربع): 2,500,000 جنيه مصري
        \item تجهيز وتطوير الموقع: 1,000,000 جنيه مصري
        \item توصيلات المرافق والبنية التحتية: 750,000 جنيه مصري
    \end{itemize}
    
    \item \textbf{معدات الإنتاج:}
    \begin{itemize}
        \item نظام استخراج وتكرير الزيت: 4,500,000 جنيه مصري
        \item وحدات إنتاج الديزل الحيوي: 6,000,000 جنيه مصري
        \item نظام إنتاج الفحم الحيوي: 3,500,000 جنيه مصري
        \item مختبر مراقبة الجودة: 1,200,000 جنيه مصري
        \item الأنظمة المساعدة والمرافق: 2,000,000 جنيه مصري
    \end{itemize}
    
    \item \textbf{البنية التحتية:}
    \begin{itemize}
        \item المباني والإنشاءات: 3,500,000 جنيه مصري
        \item خزانات التخزين والصوامع: 2,000,000 جنيه مصري
        \item أنظمة السلامة والبيئة: 1,500,000 جنيه مصري
        \item مرافق المكاتب والموظفين: 800,000 جنيه مصري
    \end{itemize}
    
    \item \textbf{إجمالي النفقات الرأسمالية: 29,250,000 جنيه مصري}
\end{itemize}

\subsection{تكاليف التشغيل (سنوياً)}

\subsubsection{تكاليف الإنتاج المباشرة}
\begin{itemize}
    \item \textbf{المواد الخام:}
    \begin{itemize}
        \item زيت الطهي المستعمل: 3,000,000 جنيه مصري
        \item الكتلة الحيوية للأزولا: 1,200,000 جنيه مصري
        \item المخلفات الزراعية: 800,000 جنيه مصري
        \item المواد الكيميائية للعمليات: 1,500,000 جنيه مصري
    \end{itemize}
    
    \item \textbf{العمالة:}
    \begin{itemize}
        \item طاقم الإنتاج (15 موظف): 1,800,000 جنيه مصري
        \item الطاقم التقني (6 موظفين): 1,200,000 جنيه مصري
        \item الإدارة والشؤون الإدارية (5 موظفين): 1,500,000 جنيه مصري
        \item التدريب والتطوير: 300,000 جنيه مصري
    \end{itemize}
    
    \item \textbf{المرافق:}
    \begin{itemize}
        \item الكهرباء: 900,000 جنيه مصري
        \item المياه: 200,000 جنيه مصري
        \item حرارة العمليات: 400,000 جنيه مصري
    \end{itemize}
\end{itemize}

\subsubsection{التكاليف غير المباشرة}
\begin{itemize}
    \item الصيانة والإصلاحات: 1,200,000 جنيه مصري
    \item التأمين: 600,000 جنيه مصري
    \item الامتثال البيئي: 400,000 جنيه مصري
    \item المختبر ومراقبة الجودة: 300,000 جنيه مصري
    \item التسويق والمبيعات: 500,000 جنيه مصري
    \item المصروفات الإدارية: 400,000 جنيه مصري
\end{itemize}

\subsubsection{إجمالي تكاليف التشغيل: 14,200,000 جنيه مصري}

\subsection{توقعات الإيرادات (سنوياً)}

\subsubsection{المنتجات الرئيسية}
\begin{itemize}
    \item \textbf{الديزل الحيوي:}
    \begin{itemize}
        \item الإنتاج: 500,000 لتر
        \item السعر لكل لتر: 20 جنيه مصري
        \item الإيراد السنوي: 10,000,000 جنيه مصري
    \end{itemize}
    
    \item \textbf{الفحم الحيوي:}
    \begin{itemize}
        \item الإنتاج: 300 طن
        \item السعر لكل طن: 8,000 جنيه مصري
        \item الإيراد السنوي: 2,400,000 جنيه مصري
    \end{itemize}
\end{itemize}

\subsubsection{المنتجات الثانوية وائتمانات الكربون}
\begin{itemize}
    \item \textbf{الجلسرين:}
    \begin{itemize}
        \item الإنتاج: 50 طن
        \item السعر لكل طن: 12,000 جنيه مصري
        \item الإيراد السنوي: 600,000 جنيه مصري
    \end{itemize}
    
    \item \textbf{ائتمانات الكربون:}
    \begin{itemize}
        \item خفض الكربون: 2,100 طن مكافئ ثاني أكسيد الكربون
        \item السعر لكل طن: 2,000 جنيه مصري
        \item الإيراد السنوي: 4,200,000 جنيه مصري
    \end{itemize}
    
    \item \textbf{استعادة حرارة العمليات:}
    \begin{itemize}
        \item توفير الطاقة: 1,800 ميجاواط ساعة
        \item القيمة لكل ميجاواط ساعة: 800 جنيه مصري
        \item التوفير السنوي: 1,440,000 جنيه مصري
    \end{itemize}
\end{itemize}

\subsubsection{إجمالي الإيرادات السنوية: 18,640,000 جنيه مصري}

\subsection{التحليل المالي}

\subsubsection{مؤشرات الربحية}
\begin{itemize}
    \item \textbf{الربح التشغيلي السنوي:}
    \begin{itemize}
        \item إجمالي الإيرادات: 18,640,000 جنيه مصري
        \item تكاليف التشغيل: 14,200,000 جنيه مصري
        \item الربح التشغيلي: 4,440,000 جنيه مصري
    \end{itemize}
    
    \item \textbf{العائد على الاستثمار:}
    \begin{itemize}
        \item الاستثمار الأولي: 29,250,000 جنيه مصري
        \item الربح السنوي: 4,440,000 جنيه مصري
        \item العائد البسيط على الاستثمار: 15.2\%
        \item فترة الاسترداد: 6.6 سنوات
    \end{itemize}
\end{itemize}

\subsubsection{الاستدامة المالية}
\begin{itemize}
    \item \textbf{إدارة رأس المال العامل:}
    \begin{itemize}
        \item معدل دوران المخزون: 12 مرة سنوياً
        \item فترة تحصيل الذمم المدينة: 30 يوم
        \item فترة سداد الذمم الدائنة: 45 يوم
        \item متطلبات رأس المال العامل: 3,550,000 جنيه مصري
    \end{itemize}
    
    \item \textbf{تخفيف المخاطر:}
    \begin{itemize}
        \item التحوط لأسعار المواد الخام
        \item تنويع مصادر الإيرادات
        \item اتفاقيات البيع المسبق لائتمانات الكربون
        \item صندوق الطوارئ: 10\% من الإيرادات السنوية
    \end{itemize}
\end{itemize}

\subsection{هيكل التمويل}

\subsubsection{مصادر رأس المال}
\begin{itemize}
    \item استثمار حقوق الملكية: 40\% (11,700,000 جنيه مصري)
    \item سندات خضراء: 30\% (8,775,000 جنيه مصري)
    \item قرض بنكي: 20\% (5,850,000 جنيه مصري)
    \item منح حكومية: 10\% (2,925,000 جنيه مصري)
\end{itemize}

\subsubsection{التخطيط المالي}
\begin{itemize}
    \item \textbf{خدمة الدين:}
    \begin{itemize}
        \item مدة القرض: 7 سنوات
        \item معدل الفائدة: 12\% سنوياً
        \item خدمة الدين السنوية: 1,200,000 جنيه مصري
    \end{itemize}
    
    \item \textbf{الصناديق الاحتياطية:}
    \begin{itemize}
        \item احتياطي الصيانة: 1,000,000 جنيه مصري
        \item الامتثال البيئي: 500,000 جنيه مصري
        \item تحديث التكنولوجيا: 1,500,000 جنيه مصري
    \end{itemize}
\end{itemize}
