\section{Integration Plan for Biodiesel Production}

\subsection{Phased Integration (2026-2031)}

\subsubsection{Phase 1 (2026-2027)}
\begin{itemize}
    \item \textbf{Inputs:}
    \begin{itemize}
        \item Initial Azolla feedstock (5 tons annually)
        \item Waste cooking oil collection
        \item Basic methanol and catalyst supplies
        \item Solar power integration
    \end{itemize}
    \item \textbf{Outputs:}
    \begin{itemize}
        \item Biodiesel production (50,000 liters annually)
        \item Glycerin for soap making
        \item Process heat for other units
        \item Initial carbon credits
    \end{itemize}
    \item \textbf{Integration Points:}
    \begin{itemize}
        \item Azolla unit: Feedstock supply
        \item Vermicomposting: Process residues
        \item Agricultural units: Fuel supply
    \end{itemize}
\end{itemize}

\subsubsection{Phase 2 (2027-2028)}
\begin{itemize}
    \item \textbf{Inputs:}
    \begin{itemize}
        \item Expanded Azolla feedstock (15 tons annually)
        \item Enhanced waste oil collection
        \item Optimized chemical inputs
        \item Improved energy efficiency
    \end{itemize}
    \item \textbf{Outputs:}
    \begin{itemize}
        \item Increased biodiesel (150,000 liters annually)
        \item Enhanced glycerin products
        \item Biochar from processing residues
        \item Growing carbon credits
    \end{itemize}
    \item \textbf{Integration Points:}
    \begin{itemize}
        \item Multiple cultivation units
        \item Livestock unit integration
        \item Enhanced waste recovery
    \end{itemize}
\end{itemize}

\subsubsection{Phase 3 (2028-2029)}
\begin{itemize}
    \item \textbf{Inputs:}
    \begin{itemize}
        \item Full-scale Azolla feedstock (30 tons annually)
        \item Diversified feedstock sources
        \item Advanced catalyst systems
        \item Maximum energy efficiency
    \end{itemize}
    \item \textbf{Outputs:}
    \begin{itemize}
        \item Peak biodiesel production (300,000 liters annually)
        \item Industrial glycerin products
        \item Maximum biochar production
        \item Significant carbon credits
    \end{itemize}
    \item \textbf{Integration Points:}
    \begin{itemize}
        \item All units: Resource cycling
        \item Complete waste recovery
        \item Carbon credit optimization
    \end{itemize}
\end{itemize}

\subsubsection{Phase 4 (2029-2030)}
\begin{itemize}
    \item \textbf{Inputs:}
    \begin{itemize}
        \item Maximum Azolla processing (50 tons annually)
        \item Optimized feedstock mix
        \item Advanced processing aids
        \item Smart energy systems
    \end{itemize}
    \item \textbf{Outputs:}
    \begin{itemize}
        \item Enhanced biodiesel (400,000 liters annually)
        \item Premium glycerin products
        \item Optimized by-product streams
        \item Maximum carbon credits
    \end{itemize}
    \item \textbf{Integration Points:}
    \begin{itemize}
        \item Complete system integration
        \item Value-added processing
        \item Enhanced sustainability
    \end{itemize}
\end{itemize}

\subsubsection{Phase 5 (2030-2031)}
\begin{itemize}
    \item \textbf{Inputs:}
    \begin{itemize}
        \item Full capacity Azolla (65 tons annually)
        \item Complete feedstock network
        \item Optimized processing systems
        \item Peak energy efficiency
    \end{itemize}
    \item \textbf{Outputs:}
    \begin{itemize}
        \item Maximum biodiesel (500,000 liters annually)
        \item Maximum value by-products
        \item Full carbon credit generation
        \item Complete system optimization
    \end{itemize}
    \item \textbf{Integration Points:}
    \begin{itemize}
        \item Full circular economy integration
        \item Complete resource optimization
        \item Maximum system efficiency
    \end{itemize}
\end{itemize}

% Arabic translation
\selectlanguage{arabic}
\section{خطة التكامل لإنتاج الديزل الحيوي}

\subsection{التكامل المرحلي (2026-2031)}

\subsubsection{المرحلة الأولى (2026-2027)}
\begin{itemize}
    \item \textbf{المدخلات:}
    \begin{itemize}
        \item مواد أولية أزولا أولية (5 أطنان سنوياً)
        \item جمع زيت الطهي المستعمل
        \item إمدادات أساسية من الميثانول والمحفز
        \item تكامل الطاقة الشمسية
    \end{itemize}
    \item \textbf{المخرجات:}
    \begin{itemize}
        \item إنتاج الديزل الحيوي (50,000 لتر سنوياً)
        \item جليسرين لصناعة الصابون
        \item حرارة العملية للوحدات الأخرى
        \item ائتمانات كربون أولية
    \end{itemize}
    \item \textbf{نقاط التكامل:}
    \begin{itemize}
        \item وحدة الأزولا: توريد المواد الأولية
        \item التسميد الدودي: مخلفات العملية
        \item الوحدات الزراعية: إمداد الوقود
    \end{itemize}
\end{itemize}

\subsubsection{المرحلة الثانية (2027-2028)}
\begin{itemize}
    \item \textbf{المدخلات:}
    \begin{itemize}
        \item توسيع مواد أولية الأزولا (15 طن سنوياً)
        \item تحسين جمع الزيت المستعمل
        \item مدخلات كيميائية محسنة
        \item تحسين كفاءة الطاقة
    \end{itemize}
    \item \textbf{المخرجات:}
    \begin{itemize}
        \item زيادة الديزل الحيوي (150,000 لتر سنوياً)
        \item منتجات جليسرين محسنة
        \item فحم حيوي من مخلفات المعالجة
        \item نمو ائتمانات الكربون
    \end{itemize}
    \item \textbf{نقاط التكامل:}
    \begin{itemize}
        \item وحدات زراعية متعددة
        \item تكامل وحدة الثروة الحيوانية
        \item تحسين استعادة النفايات
    \end{itemize}
\end{itemize}

\subsubsection{المرحلة الثالثة (2028-2029)}
\begin{itemize}
    \item \textbf{المدخلات:}
    \begin{itemize}
        \item مواد أولية أزولا كاملة النطاق (30 طن سنوياً)
        \item مصادر متنوعة للمواد الأولية
        \item أنظمة محفزات متقدمة
        \item كفاءة قصوى للطاقة
    \end{itemize}
    \item \textbf{المخرجات:}
    \begin{itemize}
        \item ذروة إنتاج الديزل الحيوي (300,000 لتر سنوياً)
        \item منتجات جليسرين صناعية
        \item أقصى إنتاج للفحم الحيوي
        \item ائتمانات كربون كبيرة
    \end{itemize}
    \item \textbf{نقاط التكامل:}
    \begin{itemize}
        \item جميع الوحدات: دورة الموارد
        \item استعادة كاملة للنفايات
        \item تحسين ائتمان الكربون
    \end{itemize}
\end{itemize}

\subsubsection{المرحلة الرابعة (2029-2030)}
\begin{itemize}
    \item \textbf{المدخلات:}
    \begin{itemize}
        \item معالجة قصوى للأزولا (50 طن سنوياً)
        \item مزيج محسن من المواد الأولية
        \item مساعدات معالجة متقدمة
        \item أنظمة طاقة ذكية
    \end{itemize}
    \item \textbf{المخرجات:}
    \begin{itemize}
        \item ديزل حيوي محسن (400,000 لتر سنوياً)
        \item منتجات جليسرين ممتازة
        \item تدفقات محسنة للمنتجات الثانوية
        \item أقصى ائتمانات كربون
    \end{itemize}
    \item \textbf{نقاط التكامل:}
    \begin{itemize}
        \item تكامل كامل للنظام
        \item معالجة ذات قيمة مضافة
        \item استدامة محسنة
    \end{itemize}
\end{itemize}

\subsubsection{المرحلة الخامسة (2030-2031)}
\begin{itemize}
    \item \textbf{المدخلات:}
    \begin{itemize}
        \item سعة كاملة للأزولا (65 طن سنوياً)
        \item شبكة كاملة للمواد الأولية
        \item أنظمة معالجة محسنة
        \item كفاءة قصوى للطاقة
    \end{itemize}
    \item \textbf{المخرجات:}
    \begin{itemize}
        \item أقصى إنتاج للديزل الحيوي (500,000 لتر سنوياً)
        \item أقصى قيمة للمنتجات الثانوية
        \item توليد كامل لائتمانات الكربون
        \item تحسين كامل للنظام
    \end{itemize}
    \item \textbf{نقاط التكامل:}
    \begin{itemize}
        \item تكامل كامل مع الاقتصاد الدائري
        \item تحسين كامل للموارد
        \item كفاءة قصوى للنظام
    \end{itemize}
\end{itemize}
