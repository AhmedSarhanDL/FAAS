\section{Governance Framework}

\subsection{Unified Risk Management Framework} \label{sec:unified_risk_management}

\subsubsection{Purpose and Scope}
This unified risk management framework establishes a standardized approach to identifying, assessing, mitigating, and monitoring risks across all units of the El Tor Circular Economy project. It aims to:

\begin{itemize}
    \item Ensure consistent risk assessment methodologies across all operational units
    \item Reduce redundancy in risk documentation while maintaining comprehensive coverage
    \item Facilitate cross-unit risk awareness and mitigation coordination
    \item Provide clear accountability and responsibility structures for risk management
    \item Enable effective reporting and escalation of critical risks
\end{itemize}

\subsubsection{Risk Categories} \label{sec:risk_categories}
The project utilizes the following standardized risk categories for classification:

\begin{table}[h]
\centering
\begin{tabular}{|p{3cm}|p{9cm}|}
\hline
\textbf{Risk Category} & \textbf{Description} \\
\hline
Strategic Risks & Risks affecting the achievement of organizational objectives related to mission, vision, and long-term goals \\
\hline
Operational Risks & Risks affecting day-to-day operations, including production processes, resource availability, and quality control \\
\hline
Financial Risks & Risks related to financial sustainability, including capital investments, cash flow, revenue generation, and cost management \\
\hline
Environmental Risks & Risks related to environmental impacts, climate change adaptation, resource conservation, and ecosystem health \\
\hline
Technical Risks & Risks related to technology implementation, performance, reliability, and compatibility \\
\hline
Market Risks & Risks related to market demand, competition, price volatility, and consumer preferences \\
\hline
Regulatory Risks & Risks related to compliance with laws, regulations, standards, and certification requirements \\
\hline
Social Risks & Risks related to community relations, labor practices, stakeholder engagement, and social acceptance \\
\hline
\end{tabular}
\caption{Standardized Risk Categories}
\end{table}

\subsubsection{Risk Assessment Methodology} \label{sec:risk_assessment_methodology}

All units must apply the following standardized risk assessment methodology:

\begin{enumerate}
    \item \textbf{Risk Identification:} Systematic identification of risks through structured processes (brainstorming, SWOT analysis, scenario planning, etc.)
    
    \item \textbf{Risk Analysis:} Evaluation of risk likelihood and impact using the standardized scales below:
    
    \begin{table}[h]
    \centering
    \begin{tabular}{|p{3cm}|p{9cm}|}
    \hline
    \textbf{Likelihood Score} & \textbf{Definition} \\
    \hline
    1 (Rare) & May occur only in exceptional circumstances (< 5\% probability) \\
    \hline
    2 (Unlikely) & Could occur at some time (5-25\% probability) \\
    \hline
    3 (Possible) & Might occur at some time (25-50\% probability) \\
    \hline
    4 (Likely) & Will probably occur in most circumstances (50-75\% probability) \\
    \hline
    5 (Almost Certain) & Expected to occur in most circumstances (> 75\% probability) \\
    \hline
    \end{tabular}
    \caption{Risk Likelihood Scale}
    \end{table}
    
    \begin{table}[h]
    \centering
    \begin{tabular}{|p{3cm}|p{9cm}|}
    \hline
    \textbf{Impact Score} & \textbf{Definition} \\
    \hline
    1 (Negligible) & Minimal impact that can be absorbed through normal activity \\
    \hline
    2 (Minor) & Minor impact with some adjustments to operational activities \\
    \hline
    3 (Moderate) & Significant impact requiring substantial adjustments to operations \\
    \hline
    4 (Major) & Major impact threatening program success and requiring significant resources \\
    \hline
    5 (Severe) & Catastrophic impact threatening project viability and sustainability \\
    \hline
    \end{tabular}
    \caption{Risk Impact Scale}
    \end{table}
    
    \item \textbf{Risk Evaluation and Prioritization:} Determination of risk significance using the Risk Priority Matrix:
    
    \begin{table}[h]
    \centering
    \begin{tabular}{|p{2cm}|p{2cm}|p{2cm}|p{2cm}|p{2cm}|p{2cm}|}
    \hline
    \multirow{2}{*}{\textbf{Likelihood}} & \multicolumn{5}{c|}{\textbf{Impact}} \\
    \cline{2-6}
    & \textbf{1} & \textbf{2} & \textbf{3} & \textbf{4} & \textbf{5} \\
    \hline
    \textbf{5} & 5 (Moderate) & 10 (High) & 15 (Extreme) & 20 (Extreme) & 25 (Extreme) \\
    \hline
    \textbf{4} & 4 (Moderate) & 8 (High) & 12 (High) & 16 (Extreme) & 20 (Extreme) \\
    \hline
    \textbf{3} & 3 (Low) & 6 (Moderate) & 9 (High) & 12 (High) & 15 (Extreme) \\
    \hline
    \textbf{2} & 2 (Low) & 4 (Moderate) & 6 (Moderate) & 8 (High) & 10 (High) \\
    \hline
    \textbf{1} & 1 (Low) & 2 (Low) & 3 (Low) & 4 (Moderate) & 5 (Moderate) \\
    \hline
    \end{tabular}
    \caption{Risk Priority Matrix}
    \end{table}
    
    \item \textbf{Risk Response Planning:} Development of appropriate response strategies:
    \begin{itemize}
        \item \textbf{Avoid:} Eliminating the threat by eliminating the cause
        \item \textbf{Mitigate:} Reducing the probability and/or impact of the risk
        \item \textbf{Transfer:} Shifting the risk impact and management to a third party
        \item \textbf{Accept:} Acknowledging the risk without taking action, with contingency plans if needed
    \end{itemize}
\end{enumerate}

\subsubsection{Risk Management Responsibilities} \label{sec:risk_responsibilities}

\begin{itemize}
    \item \textbf{Project Executive Committee:} Ultimate oversight of risk management, approval of risk thresholds, and review of extreme risks
    \item \textbf{Risk Management Coordinator:} Central coordination of risk management activities, maintenance of risk register, and facilitation of cross-unit risk assessment
    \item \textbf{Unit Managers:} Implementation of risk management within respective units, identification and assessment of unit-specific risks, and execution of mitigation measures
    \item \textbf{Technical Specialists:} Provision of expertise for specialized risk assessments in relevant domains
    \item \textbf{All Staff:} Ongoing identification and reporting of risks, implementation of control measures
\end{itemize}

\subsubsection{Risk Documentation and Reporting} \label{sec:risk_reporting}

\begin{itemize}
    \item \textbf{Master Risk Register:} Centralized database of all identified risks across the project
    \item \textbf{Unit Risk Registers:} Unit-specific register with detailed analysis of unit-level risks
    \item \textbf{Risk Response Plans:} Detailed plans for high and extreme risks
    \item \textbf{Risk Monitoring Reports:} Regular monitoring reports tracking risk status and effectiveness of controls
    \item \textbf{Risk Escalation Protocol:} Formal procedure for escalating risks that exceed defined thresholds
\end{itemize}

\subsubsection{Monitoring and Review Process} \label{sec:risk_monitoring}

\begin{itemize}
    \item \textbf{Regular Reviews:}
    \begin{itemize}
        \item Monthly review of risk registers at unit level
        \item Quarterly review of all high and extreme risks at project level
        \item Annual comprehensive review of entire risk management framework
    \end{itemize}
    
    \item \textbf{Continuous Monitoring:}
    \begin{itemize}
        \item Ongoing monitoring of early warning indicators
        \item Regular assessment of control effectiveness
        \item Horizon scanning for emerging risks
    \end{itemize}
    
    \item \textbf{Feedback Loop:}
    \begin{itemize}
        \item Documentation of lessons learned from risk events
        \item Integration of risk management improvements
        \item Knowledge sharing across units
    \end{itemize}
\end{itemize}

\subsubsection{Risk Management Integration} \label{sec:risk_integration}

Risk management is integrated with the following project management processes:

\begin{itemize}
    \item \textbf{Strategic Planning:} Risk considerations in long-term planning
    \item \textbf{Operational Planning:} Risk mitigation incorporated into annual plans
    \item \textbf{Budgeting:} Risk-based contingency allocation
    \item \textbf{Performance Measurement:} Risk management effectiveness metrics
    \item \textbf{Change Management:} Risk assessment of proposed changes
\end{itemize}

\subsubsection{Unit-Level Risk Management Requirements} \label{sec:unit_risk_requirements}

Each unit's risk management document should:

\begin{enumerate}
    \item Reference this unified framework (\verb|\ref{sec:unified_risk_management}|)
    \item Focus on unit-specific risks while avoiding repetition of common risks
    \item Categorize risks according to the standardized categories (\verb|\ref{sec:risk_categories}|)
    \item Apply the standardized assessment methodology (\verb|\ref{sec:risk_assessment_methodology}|)
    \item Clearly assign responsibilities for unit-level risk management (\verb|\ref{sec:risk_responsibilities}|)
    \item Specify unit-specific monitoring mechanisms (\verb|\ref{sec:risk_monitoring}|)
    \item Detail integration with unit operational processes (\verb|\ref{sec:risk_integration}|)
\end{enumerate}

\subsection{Decision-Making Structure}
// ... existing code ...

\subsection{Roles and Responsibilities}
// ... existing code ...

\subsection{Communication Protocols}
// ... existing code ...

\subsection{Performance Measurement}
// ... existing code ...
