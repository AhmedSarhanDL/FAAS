\section{خطة العمليات اللوجستية}

\subsection{إدارة تدفق المواد}

\subsubsection{لوجستيات الكتلة الحيوية للأزولا}
\begin{itemize}
    \item \textbf{عمليات الحصاد:}
    \begin{itemize}
        \item جدول الحصاد اليومي: 2-3 طن من الكتلة الحيوية الطازجة
        \item نشر معدات الحصاد الآلي
        \item تنسيق نقاط التجميع
        \item نقاط فحص الجودة
    \end{itemize}
    
    \item \textbf{تدفق المعالجة:}
    \begin{itemize}
        \item محطات إزالة المياه الأولية
        \item مرافق الفرز والتصنيف
        \item تخصيص وحدة المعالجة
        \item إدارة المخزون
    \end{itemize}
\end{itemize}

\subsection{شبكة النقل الداخلي}

\subsubsection{إدارة أسطول المركبات}
\begin{itemize}
    \item \textbf{معدات النقل:}
    \begin{itemize}
        \item مركبات خدمة كهربائية
        \item ناقلات الكتلة الحيوية المتخصصة
        \item أنظمة نقل المياه
        \item معدات مناولة المواد
    \end{itemize}
    
    \item \textbf{تحسين المسارات:}
    \begin{itemize}
        \item أنظمة التتبع في الوقت الفعلي
        \item تحسين الحمولة
        \item تنسيق الجداول
        \item تخطيط الصيانة
    \end{itemize}
\end{itemize}

\subsection{إدارة التخزين والمخزون}

\subsubsection{تخزين الكتلة الحيوية}
\begin{itemize}
    \item \textbf{مرافق التخزين:}
    \begin{itemize}
        \item مناطق تخزين الكتلة الحيوية الطازجة
        \item تخزين الكتلة الحيوية المعالجة
        \item وحدات متحكم في مناخها
        \item مناطق التخزين الاحتياطي
    \end{itemize}
    
    \item \textbf{مراقبة المخزون:}
    \begin{itemize}
        \item المراقبة في الوقت الفعلي
        \item بروتوكولات تدوير المخزون
        \item تدابير الحفاظ على الجودة
        \item التنبؤ بالطلب
    \end{itemize}
\end{itemize}

\subsection{التكامل بين الوحدات}

\subsubsection{أنظمة تبادل المواد}
\begin{itemize}
    \item \textbf{التكامل مع وحدة الديزل الحيوي:}
    \begin{itemize}
        \item جدولة تسليم الكتلة الحيوية
        \item تنسيق العمليات
        \item بروتوكولات مراقبة الجودة
        \item أنظمة التغذية الراجعة
    \end{itemize}
    
    \item \textbf{التكامل مع وحدة الثروة الحيوانية:}
    \begin{itemize}
        \item شبكة توزيع الأعلاف
        \item نظام جمع النفايات
        \item بروتوكولات مشاركة الموارد
        \item منع التلوث المتبادل
    \end{itemize}
\end{itemize}

\subsection{أنظمة مراقبة الجودة}

\subsubsection{ضمان الجودة}
\begin{itemize}
    \item \textbf{بروتوكولات الاختبار:}
    \begin{itemize}
        \item تقييم جودة الكتلة الحيوية
        \item مراقبة محتوى الرطوبة
        \item تحليل المغذيات
        \item فحوصات التلوث
    \end{itemize}
    
    \item \textbf{التوثيق:}
    \begin{itemize}
        \item نظام التتبع الرقمي
        \item تحديد الدفعات
        \item شهادات الجودة
        \item سجلات الامتثال
    \end{itemize}
\end{itemize}

\subsection{إجراءات الاستجابة للطوارئ}

\subsubsection{خطط الطوارئ}
\begin{itemize}
    \item \textbf{تعطل المعدات:}
    \begin{itemize}
        \item تفعيل الأنظمة الاحتياطية
        \item الصيانة الطارئة
        \item التوجيه البديل
        \item تعديل الإنتاج
    \end{itemize}
    
    \item \textbf{اضطراب سلسلة التوريد:}
    \begin{itemize}
        \item مصادر بديلة
        \item استخدام المخزون الاحتياطي
        \item تخصيص الأولويات
        \item بروتوكولات الاتصال
    \end{itemize}
\end{itemize}

\subsection{مراقبة الأداء}

\subsubsection{مؤشرات الأداء الرئيسية}
\begin{itemize}
    \item \textbf{مقاييس التشغيل:}
    \begin{itemize}
        \item كفاءة النقل
        \item استغلال التخزين
        \item معدل الإنتاج
        \item معدل الامتثال للجودة
    \end{itemize}
    
    \item \textbf{مقاييس الاستدامة:}
    \begin{itemize}
        \item البصمة الكربونية
        \item كفاءة الموارد
        \item تقليل النفايات
        \item استهلاك الطاقة
    \end{itemize}
\end{itemize}
