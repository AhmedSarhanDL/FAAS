\section{خطة إدارة الأراضي}

\subsection{نظرة عامة على المشروع}
\begin{itemize}
    \item \textbf{المساحة الإجمالية:} 200 فدان
    \item \textbf{الموقع:} سهل القاع، جنوب سيناء
    \item \textbf{مصدر المياه:} بئر مياه جوفية واحدة (سعة قانونية 1,800 متر مكعب/يوم)
    \item \textbf{مصدر الطاقة:} نظام الطاقة الشمسية
\end{itemize}

\subsection{التوزيع النهائي للأراضي}
\begin{itemize}
    \item \textbf{نخيل المجدول:} 60 فدان
    \item \textbf{أشجار الزيتون:} 45 فدان
    \item \textbf{أشجار الأكاسيا:} 45 فدان (مصدات رياح، أعلاف، تثبيت النيتروجين)
    \item \textbf{أنظمة الأزولا:} 50 فدان (مصدر للبروتين، علف، سماد طبيعي)
\end{itemize}

\subsection{خطة التنفيذ المرحلي}

\subsubsection{المرحلة الأولى (2026-2027) - 10 أفدنة}
\begin{itemize}
    \item \textbf{توزيع الأراضي:}
    \begin{itemize}
        \item نخيل: 5 فدادين (200 شجرة)
        \item زيتون: 3 فدادين (300 شجرة)
        \item أكاسيا: 2 فدان
    \end{itemize}
    
    \item \textbf{تطوير البنية التحتية:}
    \begin{itemize}
        \item إنشاء المشتل المحلي
        \item بركة أزولا تجريبية
        \item وحدة صغيرة لإنتاج الفحم الحيوي
        \item تركيب نظام الري بالتنقيط
    \end{itemize}
    
    \item \textbf{الأهداف الرئيسية:}
    \begin{itemize}
        \item تقييم أداء الري
        \item اختبار تكيف النباتات
        \item تأسيس إنتاج المشتل (2500 نخلة، 3000 زيتون، 3000 شتلة أكاسيا)
    \end{itemize}
\end{itemize}

\subsubsection{المرحلة الثانية (2027-2028) - إجمالي 30 فدان}
\begin{itemize}
    \item \textbf{التوسع:}
    \begin{itemize}
        \item نخيل: +10 فدادين
        \item زيتون: +6 فدادين
        \item أكاسيا: +4 فدادين
        \item أزولا: 3 فدادين
    \end{itemize}
    
    \item \textbf{دمج الثروة الحيوانية:}
    \begin{itemize}
        \item 5 أبقار
        \item 200 دجاجة
        \item 100 بطة
    \end{itemize}
    
    \item \textbf{مرافق المعالجة:}
    \begin{itemize}
        \item تركيب معصرة زيتون صغيرة
        \item توسيع وحدة الفحم الحيوي
    \end{itemize}
\end{itemize}

\subsubsection{المرحلة الثالثة (2028-2029) - إجمالي 60 فدان}
\begin{itemize}
    \item \textbf{التوسع:}
    \begin{itemize}
        \item نخيل: +15 فدان (600 شجرة)
        \item زيتون: +10 فدادين (1000 شجرة)
        \item أكاسيا: زيادة إلى 10 فدادين
        \item أزولا: 5 فدادين
    \end{itemize}
    
    \item \textbf{توسيع الثروة الحيوانية:}
    \begin{itemize}
        \item 15 بقرة
        \item 500 دجاجة
        \item 200 بطة
    \end{itemize}
\end{itemize}

\subsubsection{المرحلة الرابعة (2029-2030) - إجمالي 120 فدان}
\begin{itemize}
    \item \textbf{التوسع:}
    \begin{itemize}
        \item نخيل: +20 فدان (إجمالي 55 فدان)
        \item زيتون: +15 فدان (إجمالي 45 فدان)
        \item أكاسيا: زيادة إلى 25 فدان
        \item أزولا: 30 فدان
    \end{itemize}
    
    \item \textbf{تحسين البنية التحتية:}
    \begin{itemize}
        \item تطوير معصرة الزيتون
        \item تطوير وحدة معالجة التمور
        \item توسيع منشأة الفحم الحيوي
    \end{itemize}
\end{itemize}

\subsubsection{المرحلة الخامسة (2030-2031) - اكتمال المشروع 200 فدان}
\begin{itemize}
    \item \textbf{التوزيع النهائي:}
    \begin{itemize}
        \item نخيل: 60 فدان
        \item زيتون: 45 فدان
        \item أكاسيا: 45 فدان
        \item أزولا: 50 فدان
    \end{itemize}
    
    \item \textbf{الأعداد النهائية للثروة الحيوانية:}
    \begin{itemize}
        \item 25 بقرة
        \item 1000 دجاجة
        \item 300 بطة
    \end{itemize}
\end{itemize}

\subsection{تدابير الاستدامة}
\begin{itemize}
    \item \textbf{إدارة المياه:}
    \begin{itemize}
        \item أنظمة الري بالتنقيط
        \item تقنيات إعادة تدوير المياه
        \item مراقبة رطوبة التربة
        \item معالجة المياه باستخدام الأزولا
    \end{itemize}
    
    \item \textbf{تحسين التربة:}
    \begin{itemize}
        \item استخدام الفحم الحيوي
        \item إعادة تدوير المواد العضوية
        \item تثبيت النيتروجين من خلال الأكاسيا
        \item الإدارة المتكاملة للآفات
    \end{itemize}
    
    \item \textbf{الاستدامة الاقتصادية:}
    \begin{itemize}
        \item تنويع مصادر الدخل
        \item معالجة القيمة المضافة
        \item استخدام الطاقة الشمسية
        \item إدارة الموارد الدائرية
    \end{itemize}
\end{itemize}
