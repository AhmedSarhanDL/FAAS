\section{إطار الحوكمة}

\subsection{\RL{إطار موحد لإدارة المخاطر}} \label{sec:unified_risk_management_ar}

\subsubsection{\RL{الغرض والنطاق}}
\RL{يؤسس إطار إدارة المخاطر الموحد هذا نهجًا موحدًا لتحديد وتقييم وتخفيف ومراقبة المخاطر عبر جميع وحدات مشروع اقتصاد الطور الدائري. ويهدف إلى:}

\begin{itemize}
    \item \RL{ضمان منهجيات متسقة لتقييم المخاطر عبر جميع الوحدات التشغيلية}
    \item \RL{تقليل التكرار في توثيق المخاطر مع الحفاظ على تغطية شاملة}
    \item \RL{تسهيل الوعي بالمخاطر عبر الوحدات وتنسيق جهود التخفيف}
    \item \RL{توفير هياكل واضحة للمساءلة والمسؤولية في إدارة المخاطر}
    \item \RL{تمكين الإبلاغ الفعال والتصعيد للمخاطر الحرجة}
\end{itemize}

\subsubsection{\RL{فئات المخاطر}} \label{sec:risk_categories_ar}
\RL{يستخدم المشروع فئات المخاطر الموحدة التالية للتصنيف:}

\begin{table}[h]
\centering
\begin{tabular}{p{\dimexpr\linewidth/1-2\tabcolsep\relax}}m}|p{9cm}|}
\hline
\textbf{\RL{فئة المخاطر}} & \textbf{\RL{الوصف}} \\
\hline
\RL{المخاطر الاستراتيجية} & \RL{المخاطر التي تؤثر على تحقيق الأهداف التنظيمية المتعلقة بالرسالة والرؤية والأهداف طويلة المدى} \\
\hline
\RL{المخاطر التشغيلية} & \RL{المخاطر التي تؤثر على العمليات اليومية، بما في ذلك عمليات الإنتاج وتوافر الموارد ومراقبة الجودة} \\
\hline
\RL{المخاطر المالية} & \RL{المخاطر المتعلقة بالاستدامة المالية، بما في ذلك الاستثمارات الرأسمالية والتدفق النقدي وتوليد الإيرادات وإدارة التكاليف} \\
\hline
\RL{المخاطر البيئية} & \RL{المخاطر المتعلقة بالتأثيرات البيئية والتكيف مع تغير المناخ والحفاظ على الموارد وصحة النظام البيئي} \\
\hline
\RL{المخاطر التقنية} & \RL{المخاطر المتعلقة بتنفيذ التكنولوجيا وأدائها وموثوقيتها وتوافقها} \\
\hline
\RL{مخاطر السوق} & \RL{المخاطر المتعلقة بالطلب في السوق والمنافسة وتقلب الأسعار وتفضيلات المستهلكين} \\
\hline
\RL{المخاطر التنظيمية} & \RL{المخاطر المتعلقة بالامتثال للقوانين واللوائح والمعايير ومتطلبات الشهادات} \\
\hline
\RL{المخاطر الاجتماعية} & \RL{المخاطر المتعلقة بالعلاقات المجتمعية وممارسات العمل ومشاركة أصحاب المصلحة والقبول الاجتماعي} \\
\hline
\end{tabular}
\caption{\RL{فئات المخاطر الموحدة}}
\end{table}

\subsubsection{\RL{منهجية تقييم المخاطر}} \label{sec:risk_assessment_methodology_ar}

\RL{يجب على جميع الوحدات تطبيق منهجية تقييم المخاطر الموحدة التالية:}

\begin{enumerate}
    \item \textbf{\RL{تحديد المخاطر:}} \RL{التحديد المنهجي للمخاطر من خلال عمليات منظمة (العصف الذهني، تحليل SWOT، تخطيط السيناريوهات، إلخ)}
    
    \item \textbf{\RL{تحليل المخاطر:}} \RL{تقييم احتمالية وتأثير المخاطر باستخدام المقاييس الموحدة أدناه:}
    
    \begin{table}[h]
    \centering
    \begin{tabular}{p{\dimexpr\linewidth/1-2\tabcolsep\relax}}p{10p{\dimexpr\linewidth/7-2\tabcolsep\relax}m}|}
    \hline
    \textbf{\RL{درجة الاحتمالية}} & \textbf{\RL{التعريف}} \\
    \hline
    1 (\RL{نادر}) & \RL{قد يحدث فقط في ظروف استثنائية (احتمالية < 5\%)} \\
    \hline
    2 (\RL{غير مرجح}) & \RL{يمكن أن يحدث في وقت ما (احتمالية 5-25\%)} \\
    \hline
    3 (\RL{ممكن}) & \RL{قد يحدث في وقت ما (احتمالية 25-50\%)} \\
    \hline
    4 (\RL{مرجح}) & \RL{من المحتمل أن يحدث في معظم الظروف (احتمالية 50-75\%)} \\
    \hline
    5 (\RL{شبه مؤكد}) & \RL{من المتوقع أن يحدث في معظم الظروف (احتمالية > 75\%)} \\
    \hline
    \end{tabular}
    \caption{\RL{مقياس احتمالية المخاطر}}
    \end{table}
    
    \begin{table}[h]
    \centering
    \begin{tabular}{p{\dimexpr\linewidth/1-2\tabcolsep\relax}}p{10p{\dimexpr\linewidth/7-2\tabcolsep\relax}m}|}
    \hline
    \textbf{\RL{درجة التأثير}} & \textbf{\RL{التعريف}} \\
    \hline
    1 (\RL{ضئيل}) & \RL{تأثير ضئيل يمكن استيعابه من خلال النشاط العادي} \\
    \hline
    2 (\RL{طفيف}) & \RL{تأثير طفيف مع بعض التعديلات على الأنشطة التشغيلية} \\
    \hline
    3 (\RL{متوسط}) & \RL{تأثير كبير يتطلب تعديلات جوهرية على العمليات} \\
    \hline
    4 (\RL{كبير}) & \RL{تأثير كبير يهدد نجاح البرنامج ويتطلب موارد كبيرة} \\
    \hline
    5 (\RL{شديد}) & \RL{تأثير كارثي يهدد جدوى واستدامة المشروع} \\
    \hline
    \end{tabular}
    \caption{\RL{مقياس تأثير المخاطر}}
    \end{table}
    
    \item \textbf{\RL{تقييم المخاطر وتحديد الأولويات:}} \RL{تحديد أهمية المخاطر باستخدام مصفوفة أولوية المخاطر:}
    
    \begin{table}[h]
    \centering
    \begin{tabular}{p{\dimexpr\linewidth/1-2\tabcolsep\relax}}p{\dimexpr\linewidth/6-2\tabcolsep\relax}p{\dimexpr\linewidth/6-2\tabcolsep\relax}p{\dimexpr\linewidth/6-2\tabcolsep\relax}p{\dimexpr\linewidth/6-2\tabcolsep\relax}p{\dimexpr\linewidth/6-2\tabcolsep\relax}}
    \hline
    \multirow{2}{*}{\textbf{\RL{الاحتمالية}}} & \multicolumn{5}{c|}{\textbf{\RL{التأثير}}} \\
    \cline{2-6}
    & \textbf{1} & \textbf{2} & \textbf{3} & \textbf{4} & \textbf{5} \\
    \hline
    \textbf{5} & 5 (\RL{متوسط}) & 10 (\RL{عالي}) & 15 (\RL{قصوى}) & 20 (\RL{قصوى}) & 25 (\RL{قصوى}) \\
    \hline
    \textbf{4} & 4 (\RL{متوسط}) & 8 (\RL{عالي}) & 12 (\RL{عالي}) & 16 (\RL{قصوى}) & 20 (\RL{قصوى}) \\
    \hline
    \textbf{3} & 3 (\RL{منخفض}) & 6 (\RL{متوسط}) & 9 (\RL{عالي}) & 12 (\RL{عالي}) & 15 (\RL{قصوى}) \\
    \hline
    \textbf{2} & 2 (\RL{منخفض}) & 4 (\RL{متوسط}) & 6 (\RL{متوسط}) & 8 (\RL{عالي}) & 10 (\RL{عالي}) \\
    \hline
    \textbf{1} & 1 (\RL{منخفض}) & 2 (\RL{منخفض}) & 3 (\RL{منخفض}) & 4 (\RL{متوسط}) & 5 (\RL{متوسط}) \\
    \hline
    \end{tabular}
    \caption{\RL{مصفوفة أولوية المخاطر}}
    \end{table}
    
    \item \textbf{\RL{تخطيط الاستجابة للمخاطر:}} \RL{تطوير استراتيجيات الاستجابة المناسبة:}
    \begin{itemize}
        \item \textbf{\RL{التجنب:}} \RL{القضاء على التهديد من خلال القضاء على السبب}
        \item \textbf{\RL{التخفيف:}} \RL{تقليل احتمالية و/أو تأثير المخاطر}
        \item \textbf{\RL{النقل:}} \RL{نقل تأثير المخاطر وإدارتها إلى طرف ثالث}
        \item \textbf{\RL{القبول:}} \RL{الاعتراف بالمخاطر دون اتخاذ إجراء، مع خطط طوارئ إذا لزم الأمر}
    \end{itemize}
\end{enumerate}

\subsubsection{\RL{مسؤوليات إدارة المخاطر}} \label{sec:risk_responsibilities_ar}

\begin{itemize}
    \item \textbf{\RL{اللجنة التنفيذية للمشروع:}} \RL{الإشراف النهائي على إدارة المخاطر، والموافقة على عتبات المخاطر، ومراجعة المخاطر القصوى}
    \item \textbf{\RL{منسق إدارة المخاطر:}} \RL{التنسيق المركزي لأنشطة إدارة المخاطر، صيانة سجل المخاطر، وتسهيل تقييم المخاطر عبر الوحدات}
    \item \textbf{\RL{مديرو الوحدات:}} \RL{تنفيذ إدارة المخاطر داخل الوحدات المعنية، تحديد وتقييم المخاطر الخاصة بالوحدة، وتنفيذ تدابير التخفيف}
    \item \textbf{\RL{المتخصصون الفنيون:}} \RL{توفير الخبرة لتقييمات المخاطر المتخصصة في المجالات ذات الصلة}
    \item \textbf{\RL{جميع الموظفين:}} \RL{التحديد المستمر للمخاطر والإبلاغ عنها، تنفيذ تدابير المراقبة}
\end{itemize}

\subsubsection{\RL{توثيق المخاطر وإعداد التقارير}} \label{sec:risk_reporting_ar}

\begin{itemize}
    \item \textbf{\RL{سجل المخاطر الرئيسي:}} \RL{قاعدة بيانات مركزية لجميع المخاطر المحددة عبر المشروع}
    \item \textbf{\RL{سجلات مخاطر الوحدة:}} \RL{سجل خاص بالوحدة مع تحليل مفصل للمخاطر على مستوى الوحدة}
    \item \textbf{\RL{خطط الاستجابة للمخاطر:}} \RL{خطط مفصلة للمخاطر العالية والقصوى}
    \item \textbf{\RL{تقارير مراقبة المخاطر:}} \RL{تقارير مراقبة منتظمة تتبع حالة المخاطر وفعالية الضوابط}
    \item \textbf{\RL{بروتوكول تصعيد المخاطر:}} \RL{إجراء رسمي لتصعيد المخاطر التي تتجاوز العتبات المحددة}
\end{itemize}

\subsubsection{\RL{عملية المراقبة والمراجعة}} \label{sec:risk_monitoring_ar}

\begin{itemize}
    \item \textbf{\RL{المراجعات الدورية:}}
    \begin{itemize}
        \item \RL{مراجعة شهرية لسجلات المخاطر على مستوى الوحدة}
        \item \RL{مراجعة ربع سنوية لجميع المخاطر العالية والقصوى على مستوى المشروع}
        \item \RL{مراجعة شاملة سنوية لإطار إدارة المخاطر بأكمله}
    \end{itemize}
    
    \item \textbf{\RL{المراقبة المستمرة:}}
    \begin{itemize}
        \item \RL{المراقبة المستمرة لمؤشرات الإنذار المبكر}
        \item \RL{التقييم المنتظم لفعالية الضوابط}
        \item \RL{مسح الأفق للمخاطر الناشئة}
    \end{itemize}
    
    \item \textbf{\RL{حلقة التغذية الراجعة:}}
    \begin{itemize}
        \item \RL{توثيق الدروس المستفادة من أحداث المخاطر}
        \item \RL{دمج تحسينات إدارة المخاطر}
        \item \RL{تبادل المعرفة عبر الوحدات}
    \end{itemize}
\end{itemize}

\subsubsection{\RL{تكامل إدارة المخاطر}} \label{sec:risk_integration_ar}

\RL{يتم دمج إدارة المخاطر مع عمليات إدارة المشروع التالية:}

\begin{itemize}
    \item \textbf{\RL{التخطيط الاستراتيجي:}} \RL{اعتبارات المخاطر في التخطيط طويل المدى}
    \item \textbf{\RL{التخطيط التشغيلي:}} \RL{دمج تخفيف المخاطر في الخطط السنوية}
    \item \textbf{\RL{وضع الميزانية:}} \RL{تخصيص الطوارئ على أساس المخاطر}
    \item \textbf{\RL{قياس الأداء:}} \RL{مقاييس فعالية إدارة المخاطر}
    \item \textbf{\RL{إدارة التغيير:}} \RL{تقييم المخاطر للتغييرات المقترحة}
\end{itemize}

\subsubsection{\RL{متطلبات إدارة المخاطر على مستوى الوحدة}} \label{sec:unit_risk_requirements_ar}

\RL{يجب أن تقوم وثيقة إدارة المخاطر لكل وحدة بما يلي:}

\begin{enumerate}
    \item \RL{الإشارة إلى هذا الإطار الموحد} (\verb|\ref{sec:unified_risk_management_ar}|)
    \item \RL{التركيز على المخاطر الخاصة بالوحدة مع تجنب تكرار المخاطر المشتركة}
    \item \RL{تصنيف المخاطر وفقًا للفئات الموحدة} (\verb|\ref{sec:risk_categories_ar}|)
    \item \RL{تطبيق منهجية التقييم الموحدة} (\verb|\ref{sec:risk_assessment_methodology_ar}|)
    \item \RL{تعيين مسؤوليات واضحة لإدارة المخاطر على مستوى الوحدة} (\verb|\ref{sec:risk_responsibilities_ar}|)
    \item \RL{تحديد آليات المراقبة الخاصة بالوحدة} (\verb|\ref{sec:risk_monitoring_ar}|)
    \item \RL{تفصيل التكامل مع عمليات تشغيل الوحدة} (\verb|\ref{sec:risk_integration_ar}|)
\end{enumerate}

\subsection{\RL{هيكل اتخاذ القرار}}
// ... existing code ...

\subsection{\RL{الأدوار والمسؤوليات}}
// ... existing code ...

\subsection{\RL{بروتوكولات الاتصال}}
// ... existing code ...

\subsection{\RL{قياس الأداء}}
// ... existing code ...
