\section{خطة إدارة المياه}

\subsection{نظام إدارة المياه المتكامل}

\subsubsection{مصادر المياه والإطار القانوني}
\begin{itemize}
    \item \textbf{الحدود القانونية للاستخراج:}
    \begin{itemize}
        \item الحد الأقصى لمعدل الاستخراج: 1800 متر مكعب/يوم لكل بئر
        \item حفر آبار تحت إشراف حكومي
        \item مراقبة منتظمة لمعدلات الاستخراج
        \item الامتثال لإرشادات الإنتاج المستدام
    \end{itemize}
    
    \item \textbf{المصادر الرئيسية:}
    \begin{itemize}
        \item مياه الأمطار المحصودة (متوسط هطول 10-50 ملم/سنة)
        \item المياه الرمادية المعالجة
        \item مياه الصرف من وحدة الثروة الحيوانية
        \item مياه الصرف من نظام الاستزراع المائي
    \end{itemize}
    
    \item \textbf{فئات جودة المياه:}
    \begin{itemize}
        \item الفئة أ: مياه صالحة للشرب ومياه عمليات عالية النقاء (المواد الصلبة الذائبة < 500 جزء في المليون)
        \item الفئة ب: مياه الري وزراعة الأزولا (المواد الصلبة الذائبة 500-1000 جزء في المليون)
        \item الفئة ج: مياه الثروة الحيوانية وعمليات التنظيف (المواد الصلبة الذائبة 1000-1500 جزء في المليون)
        \item الفئة د: مياه غنية بالمغذيات لتطبيقات محددة (المواد الصلبة الذائبة 1500-2500 جزء في المليون)
    \end{itemize}
\end{itemize}

\subsection{نظام معالجة المياه القائم على الأزولا}

\subsubsection{قدرة وكفاءة المعالجة}
\begin{itemize}
    \item \textbf{أداء النظام:}
    \begin{itemize}
        \item حجم المعالجة: 5,000 متر مكعب يومياً
        \item كفاءة إزالة النيتروجين: 80-90\%
        \item كفاءة إزالة الفوسفور: 70-85\%
        \item تخفيض المعادن الثقيلة: 50-70\%
    \end{itemize}
    
    \item \textbf{التشغيل المستدام:}
    \begin{itemize}
        \item مراقبة منتظمة لكفاءة المعالجة
        \item تحسين جدول حصاد الكتلة الحيوية
        \item بروتوكولات مراقبة الجودة
        \item إجراءات صيانة النظام
    \end{itemize}
\end{itemize}

\subsection{تكامل تدفق المياه الدائري}

\subsubsection{نظام زراعة الأزولا}
\begin{itemize}
    \item \textbf{متطلبات المياه:}
    \begin{itemize}
        \item برك الزراعة: 500 متر مكعب/يوم
        \item عمليات المعالجة: 50 متر مكعب/يوم
        \item صيانة النظام: 20 متر مكعب/يوم
    \end{itemize}
    
    \item \textbf{إعادة تدوير المياه:}
    \begin{itemize}
        \item دورة تدوير مغلقة
        \item أنظمة استعادة المغذيات
        \item تدابير التحكم في التبخر
        \item مراقبة جودة المياه
    \end{itemize}
\end{itemize}

\subsubsection{تكامل الثروة الحيوانية}
\begin{itemize}
    \item \textbf{إدارة مياه الصرف:}
    \begin{itemize}
        \item أنظمة تجميع السماد ومياه الصرف
        \item عمليات المعالجة الأولية
        \item تثبيت المغذيات
        \item التطبيق المتحكم به في برك الأزولا
    \end{itemize}
    
    \item \textbf{الحفاظ على المياه:}
    \begin{itemize}
        \item أنظمة شرب فعالة
        \item إعادة تدوير مياه التنظيف
        \item بروتوكولات فصل النفايات
        \item المراقبة والصيانة
    \end{itemize}
\end{itemize}

\subsection{إدارة المياه الزراعية}

\subsubsection{أنظمة الري}
\begin{itemize}
    \item \textbf{توزيع المياه:}
    \begin{itemize}
        \item شبكات الري بالتنقيط
        \item أنظمة التطبيق الدقيق
        \item مراقبة رطوبة التربة
        \item جدولة قائمة على الطقس
    \end{itemize}
    
    \item \textbf{مصادر المياه:}
    \begin{itemize}
        \item مياه برك الأزولا المعالجة
        \item مياه الأمطار المحصودة
        \item مياه الجريان الزراعي المعاد تدويرها
        \item المياه الجوفية التكميلية
    \end{itemize}
\end{itemize}

\subsection{استراتيجيات الحفاظ على المياه}

\subsubsection{التحكم في التبخر}
\begin{itemize}
    \item \textbf{التغطية السطحية:}
    \begin{itemize}
        \item حصائر الأزولا العائمة
        \item هياكل التظليل
        \item حواجز الرياح
        \item الأغشية السطحية
    \end{itemize}
    
    \item \textbf{أنظمة التخزين:}
    \begin{itemize}
        \item خزانات مغطاة
        \item تخزين تحت الأرض
        \item خزانات معزولة
        \item أنظمة المراقبة
    \end{itemize}
\end{itemize}

\subsection{إدارة وحماية المياه الجوفية}

\subsubsection{تدابير حماية طبقة المياه الجوفية}
\begin{itemize}
    \item \textbf{منع تسرب المياه المالحة:}
    \begin{itemize}
        \item مراقبة منتظمة للآبار الساحلية
        \item الحفاظ على مسافات استخراج آمنة
        \item تنفيذ آبار حاجزة عند الحاجة
        \item نظام إنذار مبكر للتغيرات في الملوحة
    \end{itemize}
    
    \item \textbf{تعزيز إعادة التغذية:}
    \begin{itemize}
        \item بناء ثلاثة سدود استراتيجية
        \item الحصاد المتوقع: 790,000 متر مكعب/سنة
        \item صيانة أحواض الترشيح
        \item مراقبة فعالية إعادة التغذية
    \end{itemize}
\end{itemize}

\subsection{أنظمة المراقبة والتحكم}

\subsubsection{مراقبة جودة المياه}
\begin{itemize}
    \item \textbf{المعايير:}
    \begin{itemize}
        \item درجة الحموضة والتوصيل
        \item الأكسجين المذاب
        \item مستويات المغذيات
        \item تركيزات الملوثات
    \end{itemize}
    
    \item \textbf{أنظمة التحكم:}
    \begin{itemize}
        \item أجهزة استشعار آلية
        \item تسجيل البيانات في الوقت الفعلي
        \item أنظمة الإنذار
        \item بروتوكولات الاستجابة
    \end{itemize}
\end{itemize}

\subsection{خطة الاستجابة للطوارئ}

\subsubsection{بروتوكولات نقص المياه}
\begin{itemize}
    \item \textbf{أولوية التخصيص:}
    \begin{itemize}
        \item صيانة الأنظمة الحرجة
        \item إمداد المياه للثروة الحيوانية
        \item الري الأساسي للمحاصيل
        \item استقرار نظام الأزولا
    \end{itemize}
    
    \item \textbf{تدابير الحفاظ:}
    \begin{itemize}
        \item تعزيز إعادة التدوير
        \item تقليل الاستخدام غير الأساسي
        \item مصادر المياه البديلة
        \item التنسيق المجتمعي
    \end{itemize}
\end{itemize}
